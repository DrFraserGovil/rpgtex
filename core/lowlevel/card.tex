\tcbset{rpgcard/.style={}}
\newtcolorbox{__rpgcard}[1][]{
  rpgcard,
  left=0pt,
  right=0pt,
  top=\l__card_vmargin_tl,
  bottom=0pt,
  valign=top,
  #1
}

\msg_new:nnn { rpg } { card-dimensions}{~\\Dimension~error!\\The~#1~is~too~large~(#2).~It~cannot~exceed~#3~(half~the~#4).}

\tl_new:N \l__card_text_width_tl
\tl_new:N \l__card_text_height_tl

\bool_new:N \g__rpg_card_active_bool %%used to toggle the cardswitch environment
\tl_new:N \g_rpgcard_font_state_tl %%saves the font state in between cardbreaks

\box_new:N \l_rpgcard_content_box % The main box to hold ALL content
\box_new:N \l_rpgcard_chunk_box   % The box to hold one card's content
\tl_new:N \l__rpgcard_footnote_tl % container for footnote

\cs_new_protected:Npn \__rpg_build_multi_card:Nnnn#1#2#3#4
{
	\vbox_set:Nn #1{
        \hsize=#2
		\linewidth=\hsize

		
        #3

		\vfil %forces text to top of the box without weirdly stretching the paragraph gaps

		\tl_if_empty:eF{#4}
		{
			\vfil
			\hrule
			\vspace{0.5em}

			#4
		}
    }
}


\cs_new_protected:Npn \__rpg_single_card:nn#1#2
{
	% PROCESS THE CHUNK AS ITS OWN CARD
	\hbox{
		\begin{__rpgcard}
			%%Insert the previous font state (or blank, if this is the first card)
			\g_rpgcard_font_state_tl

				% \vfil

				%%box contents, centered within the card
				\hfil {\box_use:N #1}\hfil 
			
				% \vfil
			%%Save the current font state for the next iteration
			\tl_gset:Nx\g_rpgcard_font_state_tl{\the\font}
		\end{__rpgcard}
	}

	%%Insert some blank space
	\hskip #2
}
