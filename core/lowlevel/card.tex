\tcbset{rpgcard/.style={}}
\newtcolorbox{rpgcard}[1][]{
  rpgcard,
  #1
}

\msg_new:nnn { rpg } { card-dimensions}{~\\Dimension~error!\\The~#1~(#2)~cannot~be~larger~than~the~#3~(#4)}


\bool_new:N \g__rpg_card_active_bool %%used to toggle the cardswitch environment
\tl_new:N \g_rpgcard_font_state_tl %%saves the font state in between cardbreaks

\box_new:N \l_rpgcard_content_box % The main box to hold ALL content
\box_new:N \l_rpgcard_chunk_box   % The box to hold one card's content
\tl_new:N \l__rpgcard_footnote_tl % container for footnote


\cs_new_protected:Npn \__rpg_build_multi_card:Nnnn#1#2#3#4
{
	\vbox_set:Nn #1{
        \hsize=#2
		\linewidth=\hsize

		
        #3
		\tl_if_empty:eF{#4}
		{
			\vfil
			\hrule
			\vspace{0.5em}

			#4
		}
    }
}


\cs_new_protected:Npn \__rpg_single_card:nn#1#2
{
	% PROCESS THE CHUNK AS ITS OWN CARD
	\hbox{
		\begin{rpgcard}
			%%Insert the previous font state (or blank, if this is the first card)
			\g_rpgcard_font_state_tl

				%%box contents, centered within the card
				\hfil {\box_use:N #1}\hfil 
			
			%%Save the current font state for the next iteration
			\tl_gset:Nx\g_rpgcard_font_state_tl{\the\font}
		\end{rpgcard}
	}

	%%Insert some blank space
	\hskip #2
}
