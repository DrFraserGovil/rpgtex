
\frontmatter
\cover{example/img/typewriter}
\maketitle

\tableofcontents


\mainmatter{}

\part*{Commands}

\part{Basic}

\part[none]{Basic boo boo}
\section{test}
\part[engine_part]{Basic Commands}

	\chapter{Introduction}


	\RpgDropCap{H}{ello and welcome!} This package is designed to aid you in writing beautifully typeset documents for a variety of RPG documents. It is derived from {a D\&D-specific package}\footnote{\url{https://github.com/rpgtex/DND-5e-LaTeX-Template}}, but whilst we have retained much of the internal engine, we have adapted it to work with a variety of themes, thus making it a system-agnostic \verb|rpgtex|.

	This document - along with a few others - serves as a showcase of the different themes and classes available. We begin with the basic framework for the book class, and then move on to demonstrate the theme-specific options.

	\chapter{Sections}
		Sections follow the standard \LaTeX{} formatting; with the chapter formatting shown above.
		\section{Section}
			Sections break up chapters into large groups of associated text.

			\subsection{Subsection}
				Subsections further break down the information for the reader.

				\subsubsection{Subsubsection}
					Subsubsections are the furthest division of text that still have a block header. Below this level, headers are displayed inline.

					\paragraph{Paragraph}
					The paragraph format is seldom used in the core books, but is available if you prefer it to the ``normal'' style.

					\subparagraph{Subparagraph}
					The subparagraph format with the paragraph indent is likely going to be more familiar to the reader.
			\subsection{Parts}

				The rpgbook class also defines a \verb|part| (seen in the prior pages).

				Parts are often allowed to define a full-page image

				{\color{red} Write this bit!}
		\section{Map Regions}
			The map region functions \verb|\RpgArea| and \verb|\RpgSubArea| provide automatic numbering of areas.


			\RpgSetAreaOptions[area-label=village,area-num-depth=3]{}
			\RpgArea{Village of Hommlet}
			This is the village of hommlet.
			\RpgSetAreaOptions[area-label=villages,area-num-depth=3]{}
			\RpgSubArea{Inn of the Welcome Wench}
			Inside the village is the inn of the Welcome Wench.

			\RpgSubArea{Blacksmith's Forge}
			There's a blacksmith in town, too.

			\RpgArea{Foo's Castle}
			This is foo's home, a hovel of mud and sticks.

			\RpgSubArea{Moat}
			This ditch has a board spanning it.

			\RpgSubArea{Entrance}
			A five-foot hole reveals the dirt floor illuminated by a hole in the roof.




			\subsection{Resetting Counters}

				The counters can be reset
			\RpgSetAreaOptions[area-label=space]{}
			\RpgResetAreas{}
			\RpgArea{Nyxia Subsector}
				A dangerous region of space, accessed only by calling \verb|\RpgReset{}|
				\RpgSubArea{Crescent Station}
					A lawless hub of trade and crime
				\RpgSubArea{Raxxla Penal Colony}
					An ice planet populated with criminals

			\RpgArea{Great Expanse}
				A void of nothingness

			\subsection{Referencing Map Regions}

				Map regions are referenceable; so \verb|\RpgAreaRef[area-label=village]{Village of Hommlet}| is ``\RpgAreaRef[area-label=village]{Village of Hommlet}''

				When using reset-counters or otherwise ambiguous names, it is necessary to use \verb|\RpgSetAreaOptions[area-label={},sub-area-label={}]{}| to define prefixes to disambiguate.

			\subsection{Area Depths}

				Map headers can be set to different `levels', using either section/subsection (etc.) for display purposes. The default value is \verb|\RpgSetAreaOptions[area-num-depth=2]{}| (subsection). For display purposes here, we used \verb|area-num-depth=3|
		\section{RpgTable}

			\begin{RpgTable}[title=Box Colors]{lX}
  Color            &  Description \\
  |commentcolor|   & |DndComment| background \\
  |readaloudcolor| & |DndReadAloud| background \\
  |sidebarcolor|   & |DndSidebar| background \\
  |tablecolor|     & background of even |DndTable| rows \\
\end{RpgTable}
	\section{Text Boxes}

	The module has three environments for setting text apart so that it is drawn to the reader's attention. |DndReadAloud| is used for text that a game master would read aloud.
	\section{Tips \& Tricks}
	The other two environments are the |DndComment| and the |DndSidebar|. The |DndComment| is breakable and can safely be used inline in the text.

	\begin{RpgTip}{This Is a Tip Box!}
	A |RpgTip| is a box for minimal highlighting of text. It lacks the ornamentation of |DndSidebar|, but it can handle being broken over a column.
	\end{RpgTip}



	\section{Narration}

	\begin{RpgNarration}
		Narration doesn't get a title -- but that's ok, because you're reading this out loud!
	\end{RpgNarration}

	\begin{RpgSidebar}[]{Behold the DndSidebar!}
  The |DndSidebar| is used as a sidebar. It does not break over columns and is best used with a figure environment to float it to one corner of the page where the surrounding text can then flow around it.
\end{RpgSidebar}
