\documentclass[theme=dnd,size=10pt]{rpgbook}

\usepackage{hyperref}
\usepackage{blindtext}
\title{D\&D Book Test}
\subtitle{A Typesetting module for \newline the world's greatest roleplaying game!}
\author{Jack Fraser-Govil}

%  \RequirePackage{bookman}
    % \RequirePackage[type1]{gillius2}
    % \RequirePackage[notext,nomath,nott]{kpfonts}
    % \RequirePackage[T1]{fontenc}
    % \renewcommand{\sfdefault}{jkpss}

	% \newcommand{\gillius}{\gilliustwo}
	% \providecommand\kepler{\fontfamily{jkpss}\selectfont}
\begin{document}


	%%Load the generic book text (so you can see how dnd renders differently from scifi!)
   
\frontmatter
\cover{example/img/typewriter}
\maketitle

\tableofcontents


\mainmatter{}

\part*{Commands}

\part{Basic}

\part[none]{Basic boo boo}
\section{test}
\part[engine_part]{Basic Commands}

	\chapter{Introduction}


	\RpgDropCap{H}{ello and welcome!} This package is designed to aid you in writing beautifully typeset documents for a variety of RPG documents. It is derived from {a D\&D-specific package}\footnote{\url{https://github.com/rpgtex/DND-5e-LaTeX-Template}}, but whilst we have retained much of the internal engine, we have adapted it to work with a variety of themes, thus making it a system-agnostic \verb|rpgtex|.

	This document - along with a few others - serves as a showcase of the different themes and classes available. We begin with the basic framework for the book class, and then move on to demonstrate the theme-specific options.

	\chapter{Sections}
		Sections follow the standard \LaTeX{} formatting; with the chapter formatting shown above.
		\section{Section}
			Sections break up chapters into large groups of associated text.

			\subsection{Subsection}
				Subsections further break down the information for the reader.

				\subsubsection{Subsubsection}
					Subsubsections are the furthest division of text that still have a block header. Below this level, headers are displayed inline.

					\paragraph{Paragraph}
					The paragraph format is seldom used in the core books, but is available if you prefer it to the ``normal'' style.

					\subparagraph{Subparagraph}
					The subparagraph format with the paragraph indent is likely going to be more familiar to the reader.
			\subsection{Parts}

				The rpgbook class also defines a \verb|part| (seen in the prior pages).

				Parts are often allowed to define a full-page image

				{\color{red} Write this bit!}
		\section{Map Regions}
			The map region functions \verb|\RpgArea| and \verb|\RpgSubArea| provide automatic numbering of areas.


			\RpgSetAreaOptions[area-label=village,area-num-depth=3]{}
			\RpgArea{Village of Hommlet}
			This is the village of hommlet.
			\RpgSetAreaOptions[area-label=villages,area-num-depth=3]{}
			\RpgSubArea{Inn of the Welcome Wench}
			Inside the village is the inn of the Welcome Wench.

			\RpgSubArea{Blacksmith's Forge}
			There's a blacksmith in town, too.

			\RpgArea{Foo's Castle}
			This is foo's home, a hovel of mud and sticks.

			\RpgSubArea{Moat}
			This ditch has a board spanning it.

			\RpgSubArea{Entrance}
			A five-foot hole reveals the dirt floor illuminated by a hole in the roof.




			\subsection{Resetting Counters}

				The counters can be reset
			\RpgSetAreaOptions[area-label=space]{}
			\RpgResetAreas{}
			\RpgArea{Nyxia Subsector}
				A dangerous region of space, accessed only by calling \verb|\RpgReset{}|
				\RpgSubArea{Crescent Station}
					A lawless hub of trade and crime
				\RpgSubArea{Raxxla Penal Colony}
					An ice planet populated with criminals

			\RpgArea{Great Expanse}
				A void of nothingness

			\subsection{Referencing Map Regions}

				Map regions are referenceable; so \verb|\RpgAreaRef[area-label=village]{Village of Hommlet}| is ``\RpgAreaRef[area-label=village]{Village of Hommlet}''

				When using reset-counters or otherwise ambiguous names, it is necessary to use \verb|\RpgSetAreaOptions[area-label={},sub-area-label={}]{}| to define prefixes to disambiguate.

			\subsection{Area Depths}

				Map headers can be set to different `levels', using either section/subsection (etc.) for display purposes. The default value is \verb|\RpgSetAreaOptions[area-num-depth=2]{}| (subsection). For display purposes here, we used \verb|area-num-depth=3|
		\section{RpgTable}

			\begin{RpgTable}[title=Box Colors]{lX}
  Color            &  Description \\
  |commentcolor|   & |DndComment| background \\
  |readaloudcolor| & |DndReadAloud| background \\
  |sidebarcolor|   & |DndSidebar| background \\
  |tablecolor|     & background of even |DndTable| rows \\
\end{RpgTable}
\begin{RpgTable}{lX}
  Color            &  Description \\
  |commentcolor|   & |DndComment| background \\
  |readaloudcolor| & |DndReadAloud| background \\
  |sidebarcolor|   & |DndSidebar| background \\
  |tablecolor|     & background of even |DndTable| rows \\
\end{RpgTable}
	\section{Text Boxes}

	The module has three environments for setting text apart so that it is drawn to the reader's attention. |DndReadAloud| is used for text that a game master would read aloud.
	\section{Tips \& Tricks}
	The other two environments are the |DndComment| and the |DndSidebar|. The |DndComment| is breakable and can safely be used inline in the text.

	\begin{RpgTip}{This Is a Tip Box!}
	A |RpgTip| is a box for minimal highlighting of text. It lacks the ornamentation of |DndSidebar|, but it can handle being broken over a column.
	\end{RpgTip}



	\section{Narration}

	\begin{RpgNarration}
		Narration doesn't get a title -- but that's ok, because you're reading this out loud!
	\end{RpgNarration}

	\begin{RpgSidebar}[]{Behold the DndSidebar!}
  The |DndSidebar| is used as a sidebar. It does not break over columns and is best used with a figure environment to float it to one corner of the page where the surrounding text can then flow around it.
\end{RpgSidebar}




   \part[example/img/fantasy_title]{The D\&D Book Class}



		\chapter{Differences \& Overrides}

		\chapter{Monsters}

			\section{A Brief Overview}

			\begin{RpgMonster}{The Typesetter}
				 \RpgMonsterType{Medium aberration, lawful evil}

				  \RpgMonsterBasics[
						armor-class = {9 (12 with \emph{mage armor})},
						hit-points  = {\RpgDice{3d8 + 3}},
						speed       = {30 ft., fly 30 ft.},
						proficiency=2,
						str = 12,
						dex = 8,
						con = 13,
						int = 10,
						wis = 14,
						cha = 15,
						str_save,
						cha_save,
					]

				\RpgMonsterDetails[
					%saving-throws = {Str +0, Dex +0, Con +0, Int +0, Wis +0, Cha +0},
					%skills = {Acrobatics +0, Animal Handling +0, Arcana +0, Athletics +0, Deception +0, History +0, Insight +0, Intimidation +0, Investigation +0, Medicine +0, Nature +0, Perception +0, Performance +0, Persuasion +0, Religion +0, Sleight of Hand +0, Stealth +0, Survival +0},
					%damage-vulnerabilities = {cold},
					%damage-resistances = {bludgeoning, piercing, and slashing from nonmagical attacks},
					%damage-immunities = {poison},
					%condition-immunities = {poisoned},
					senses = {darkvision 60 ft., passive Perception 12},
					languages = {\LaTeX},
					challenge = 1,
					% proficiency-bonus, % = +2
				]
			\end{RpgMonster}


			\begin{RpgMonster*}[bh][]{Monster Foo}
    \RpgMonsterType{Medium aberration (metasyntactic variable), neutral evil}

    % If you want to use commas in the key values, enclose the values in braces.
    \RpgMonsterBasics[
        armor-class = {9 (12 with \emph{mage armor})},
        hit-points  = {\RpgDice{3d8 + 3}},
        speed       = {30 ft., fly 30 ft.},
		initiative  = {+5},
        str = 12,
        dex = 8,
        con = 13,
        int = 10,
        wis = 14,
        cha = 15,
		str_save,
		cha_save,
		proficiency=3
      ]

    \RpgMonsterDetails[
        %saving-throws = {Str +0, Dex +0, Con +0, Int +0, Wis +0, Cha +0},
        %skills = {Acrobatics +0, Animal Handling +0, Arcana +0, Athletics +0, Deception +0, History +0, Insight +0, Intimidation +0, Investigation +0, Medicine +0, Nature +0, Perception +0, Performance +0, Persuasion +0, Religion +0, Sleight of Hand +0, Stealth +0, Survival +0},
        %damage-vulnerabilities = {cold},
        %damage-resistances = {bludgeoning, piercing, and slashing from nonmagical attacks},
        %damage-immunities = {poison},
        %condition-immunities = {poisoned},
        senses = {darkvision 60 ft., passive Perception 10},
        languages = {Common, Goblin, Undercommon},
        challenge = 1,
      ]
    % Traits

    \begin{RpgMonsterSpells}[modifier=CHA,bonus={0}]
      \RpgSpellList{misty step}
      \RpgSpellList[3]{fog cloud, rope trick}
      \RpgSpellList[1]{identify}
	\RpgMonsterSpellSlots[1][3]{burning hands,mage armor,shield,}
    \end{RpgMonsterSpells}

    \RpgMonsterSection{Actions}
    \RpgMonsterAction{Multiattack}
    The foo makes two melee attacks.

    %Default values are shown commented out
    \RpgMonsterAttack[
      %distance=both, % valid options are in the set {both,melee,ranged},
      %type=weapon, %valid options are in the set {weapon,spell}
      modifier=STR,
      %reach=5,
      %range=20/60,
      %targets=one target,
      dmg=\RpgDice{1d4+1},
      dmg-type=piercing,
      %plus-dmg=,
      %plus-dmg-type=,
      %or-dmg=,
      %or-dmg-when=,
      %extra=,
    ]{Dagger}

    %\RpgMonsterMelee calls \RpgMonsterAttack with the melee option
    \RpgMonsterMelee[
      modifier=DEX,
      %reach=5,
      %targets=one target,
      dmg=\RpgDice{1d8+1},
      dmg-type=slashing,
      plus-dmg=\RpgDice{2d6},
      plus-dmg-type=fire,
      or-dmg=\RpgDice{1d10+1},
      or-dmg-when=if used with two hands,
      %extra=,
    ]{Flame Tongue Longsword}

    %\RpgMonsterRanged calls \RpgMonsterAttack with the ranged option
    \RpgMonsterRanged[
      modifier=CHA,
      range=80/320,
      dmg=\RpgDice{1d8},
      dmg-type=piercing,
      %plus-dmg=,
      %plus-dmg-type=,
      %or-dmg=,
      %or-dmg-when=,
      extra={, and the target must make a DC 15 Constitution saving throw, taking 24 (7d6) poison damage on a failed save, or half as much damage on a successful one}
    ]{Poisoned Crossbow}

    % Legendary Actions
    \begin{RpgMonsterLegendaryActions}[3]
      \RpgMonsterLegendaryAction{Move}{The foo moves up to its speed.}
      \RpgMonsterLegendaryAction{Dagger Attack}{The foo makes a dagger attack.}
      \RpgMonsterLegendaryAction[2]{Create Contract}{The foo presents a contract in a language it knows and waves it in the face of a creature within 10 feet. The creature must make a DC 10 Intelligence saving throw. On a failure, the creature is incapacitated until the start of the foo's next turn. A creature who cannot read the language in which the contract is written has advantage on this saving throw.}
	  \RpgLegendaryDefiance{}
    \end{RpgMonsterLegendaryActions}
\end{RpgMonster*}

		\section{Full documentation}

		The negative sign is (-)
\end{document}
