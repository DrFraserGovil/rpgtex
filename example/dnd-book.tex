\documentclass[theme=dnd,size=10pt]{rpgbook}

\usepackage{hyperref}
\usepackage{blindtext}
\title{D\&D Book Test}
\subtitle{A Typesetting module for \newline the world's greatest roleplaying game!}
\author{Jack Fraser-Govil}

%  \RequirePackage{bookman}
    % \RequirePackage[type1]{gillius2}
    % \RequirePackage[notext,nomath,nott]{kpfonts}
    % \RequirePackage[T1]{fontenc}
    % \renewcommand{\sfdefault}{jkpss}

	% \newcommand{\gillius}{\gilliustwo}
	% \providecommand\kepler{\fontfamily{jkpss}\selectfont}
\begin{document}


	%%Load the generic book text (so you can see how dnd renders differently from scifi!)
   \RpgDropCap{H}{ello} there young children! How are you today? \blindtext


{\Huge test1}


{\Huge \RpgContour{test2}}

{\RpgFontTableTitle title

\blindtext}

{\RpgFontTableHeader header

\blindtext}

{\RpgFontTableBody body

\blindtext}





% \Blindtext




   \part[example/img/fantasy_title]{The D\&D Book Class}



		\chapter{Differences \& Overrides}

		\chapter{Monsters}

			\section{A Brief Overview}

			\begin{RpgMonster}{The Typesetter}
				 \RpgMonsterType{Medium aberration, lawful evil}

				  \RpgMonsterBasics[
						armor-class = {9 (12 with \emph{mage armor})},
						hit-points  = {\RpgDice{3d8 + 3}},
						speed       = {30 ft., fly 30 ft.},
						proficiency=2,
						str = 12,
						dex = 8,
						con = 13,
						int = 10,
						wis = 14,
						cha = 15,
						str_save,
						cha_save,
					]

				\RpgMonsterDetails[
					%saving-throws = {Str +0, Dex +0, Con +0, Int +0, Wis +0, Cha +0},
					%skills = {Acrobatics +0, Animal Handling +0, Arcana +0, Athletics +0, Deception +0, History +0, Insight +0, Intimidation +0, Investigation +0, Medicine +0, Nature +0, Perception +0, Performance +0, Persuasion +0, Religion +0, Sleight of Hand +0, Stealth +0, Survival +0},
					%damage-vulnerabilities = {cold},
					%damage-resistances = {bludgeoning, piercing, and slashing from nonmagical attacks},
					%damage-immunities = {poison},
					%condition-immunities = {poisoned},
					senses = {darkvision 60 ft., passive Perception 12},
					languages = {\LaTeX},
					challenge = 1,
					% proficiency-bonus, % = +2
				]
			\end{RpgMonster}


			\begin{RpgMonster*}[bh][]{Monster Foo}
    \RpgMonsterType{Medium aberration (metasyntactic variable), neutral evil}

    % If you want to use commas in the key values, enclose the values in braces.
    \RpgMonsterBasics[
        armor-class = {9 (12 with \emph{mage armor})},
        hit-points  = {\RpgDice{3d8 + 3}},
        speed       = {30 ft., fly 30 ft.},
		initiative  = {+5},
        str = 12,
        dex = 8,
        con = 13,
        int = 10,
        wis = 14,
        cha = 15,
		str_save,
		cha_save,
		proficiency=3
      ]

    \RpgMonsterDetails[
        %saving-throws = {Str +0, Dex +0, Con +0, Int +0, Wis +0, Cha +0},
        %skills = {Acrobatics +0, Animal Handling +0, Arcana +0, Athletics +0, Deception +0, History +0, Insight +0, Intimidation +0, Investigation +0, Medicine +0, Nature +0, Perception +0, Performance +0, Persuasion +0, Religion +0, Sleight of Hand +0, Stealth +0, Survival +0},
        %damage-vulnerabilities = {cold},
        %damage-resistances = {bludgeoning, piercing, and slashing from nonmagical attacks},
        %damage-immunities = {poison},
        %condition-immunities = {poisoned},
        senses = {darkvision 60 ft., passive Perception 10},
        languages = {Common, Goblin, Undercommon},
        challenge = 1,
      ]
    % Traits

    \begin{RpgMonsterSpells}[modifier=CHA,bonus={0}]
      \RpgSpellList{misty step}
      \RpgSpellList[3]{fog cloud, rope trick}
      \RpgSpellList[1]{identify}
	\RpgMonsterSpellSlots[1][3]{burning hands,mage armor,shield,}
    \end{RpgMonsterSpells}

    \RpgMonsterSection{Actions}
    \RpgMonsterAction{Multiattack}
    The foo makes two melee attacks.

    %Default values are shown commented out
    \RpgMonsterAttack[
      %distance=both, % valid options are in the set {both,melee,ranged},
      %type=weapon, %valid options are in the set {weapon,spell}
      modifier=STR,
      %reach=5,
      %range=20/60,
      %targets=one target,
      dmg=\RpgDice{1d4+1},
      dmg-type=piercing,
      %plus-dmg=,
      %plus-dmg-type=,
      %or-dmg=,
      %or-dmg-when=,
      %extra=,
    ]{Dagger}

    %\RpgMonsterMelee calls \RpgMonsterAttack with the melee option
    \RpgMonsterMelee[
      modifier=DEX,
      %reach=5,
      %targets=one target,
      dmg=\RpgDice{1d8+1},
      dmg-type=slashing,
      plus-dmg=\RpgDice{2d6},
      plus-dmg-type=fire,
      or-dmg=\RpgDice{1d10+1},
      or-dmg-when=if used with two hands,
      %extra=,
    ]{Flame Tongue Longsword}

    %\RpgMonsterRanged calls \RpgMonsterAttack with the ranged option
    \RpgMonsterRanged[
      modifier=CHA,
      range=80/320,
      dmg=\RpgDice{1d8},
      dmg-type=piercing,
      %plus-dmg=,
      %plus-dmg-type=,
      %or-dmg=,
      %or-dmg-when=,
      extra={, and the target must make a DC 15 Constitution saving throw, taking 24 (7d6) poison damage on a failed save, or half as much damage on a successful one}
    ]{Poisoned Crossbow}

    % Legendary Actions
    \begin{RpgMonsterLegendaryActions}[3]
      \RpgMonsterLegendaryAction{Move}{The foo moves up to its speed.}
      \RpgMonsterLegendaryAction{Dagger Attack}{The foo makes a dagger attack.}
      \RpgMonsterLegendaryAction[2]{Create Contract}{The foo presents a contract in a language it knows and waves it in the face of a creature within 10 feet. The creature must make a DC 10 Intelligence saving throw. On a failure, the creature is incapacitated until the start of the foo's next turn. A creature who cannot read the language in which the contract is written has advantage on this saving throw.}
	  \RpgLegendaryDefiance{}
    \end{RpgMonsterLegendaryActions}
\end{RpgMonster*}

		\section{Full documentation}

		The negative sign is (-)
\end{document}
