\onecolumn{}
\chapter{Introduction}

\vfill{}

\begin{center}
\parbox[c]{0.8\linewidth}{
	\large
	\justifying

	\RpgDropCap{W}{elcome to the \rpgtex{} package}. This \LaTeX{} package is designed to allow users to flexibly typeset documents associated with Role Playing Games such as \textit{Dungeons \& Dragons} -- and many more besides. This packages defines a central engine: \texttt{rpgcore} which define a number of useful functions and classes, and a flexible set of \texttt{themes} which control how those commands are rendered in the final document.

	\vspace{2cm}
	\subsection*{Attribution \& License}

	This package would not have been possible without the team who developed its predecessor\footnote{}. That code was released under an MIT license, the text of which can be found in the LICENSE file. \texttt{rpgtex} is released under the same license.
	}
\end{center}
\footnotetext[\numexpr\value{footnote}\relax]{\url{{https://github.com/rpgtex/DND-5e-LaTeX-Template/tree/dev}}}
\vfill{}

\twocolumn{}
\chapter{Installation \& Usage}

	\section{Getting \rpgtex}

		There are a number of different ways to acquire \rpgtex{}. Once you have installed it, it is vital to ensure that it is \href{\ref{S:Configuration}}{properly configured} (see below).


		\subsection{texmf Installation}

			The simplest way to use \rpgtex{} is to install it on the \texttt{texmf} path, where the compiler can automatically find it:

			\begin{lstlisting}
			git clone https://github.com/DrFraserGovil/rpgtex.git "$(kpsewhich -var-value TEXMFHOME)/tex/latex/rpgtex"
			\end{lstlisting}

			This will clone the repository into your\LaTeX{} path.

		\subsection{Indirect Installation}

			If you want to tinker with \rpgtex{} -- such as by creating a new theme -- it is helpful to have it in a more accessible location. Clone the repository into a location of your choice:

			\begin{lstlisting}
			git clone https://github.com/DrFraserGovil/rpgtex.git ~/your/rpgtex/directory
			\end{lstlisting}

			You then have two options to make the package visible to the compiler:

			\subsubsection{Use TEXINPUTS}

			Setting the environment variable \verb|TEXINPUTS| allows the compiler access:
				\begin{lstlisting}
	TEXINPUTS=~/your/rpgtex/directory/::
				\end{lstlisting}
				(Or similar commands, depending on your shell -- in \texttt{fish} you would call \verb|set TEXINPUTS dir|).

			\subsubsection{Use Symlinks}

			You can symlink the install location to the texmf directory, allowing the compiler to act as if you had performed the texmf installation:

			\begin{lstlisting}
				ln -sf ~/your/rpgtex/directory "$(kpsewhich -var-value TEXMFHOME)/tex/latex/rpgtex"
			\end{lstlisting}

		\subsection{Overleaf (Not recommended!)}

			We do not recommend using Overleaf since the free-tier subscription has reduced compilation times drastically, making compiling documents using complex packages such as this one extremely difficult. Nevertheless:

			\begin{enumerate}
				\item  Download this GitHub repository as a ZIP archive using the Clone or download link above.
    			\item On Overleaf, click the New Project button and select Upload Project. Upload the ZIP archive you downloaded from this repository.
				\item Manually create the file \texttt{rpg-config.cfg} with the contents ``\verb|\edef\RpgPackagePath{../}|''. This replaces the configuration step described below.

			\end{enumerate}


	\section{Configuring \rpgtex{}} \label{S:Configuration}

		Wherever one installs \rpgtex{} from, it is vital that it is properly configured. From within the \rpgtex{}-root directory, call:

		\begin{lstlisting}
			./configure
		\end{lstlisting}
		Or -- if one is (reasonably!) wary about running arbitrary executables -- manually create the relevant file:
		\begin{lstlisting}
			cd <rpgtex root directory>
			cmd="\edef\RpgPackagePath{$(pwd)}"
			echo $cmd >> core/rpg-config.cfg
		\end{lstlisting}

		\begin{RpgTip}{Why is configuration necessary?}
			\TeX{} is generally set up so that when a file calls \verb|include| or \verb|input| it is possible to use filepaths relative to the package itself. \texttt{rpg.sty} can call \verb|\ExplSyntaxOn
%%%reactivate the original rm-font loaded
\providefontfamily{\lrfont}{Latin~Modern~Roman}
[
	SmallCapsFont=Latin~Modern~Roman~Caps
]

\RpgSetFont{
	main-body-family = \lrfont,
	main-body-style= {},
	title-family         = \normalfont,
    title-style         = \Huge,
	% Subtitle
    subtitle-family         = \normalfont,
    subtitle-style         = \Large,
	% Part
    part-family         = \normalfont,
    part-style         = \Huge,
    % Chapter
    chapter-family         = \normalfont,
    chapter-style         = \Huge \bfseries,
    % Section
    section-family         = \normalfont,
    section-style         = \huge \bfseries,
    % Subsection
    subsection-family         = \normalfont,
    subsection-style         = \Large \bfseries,
    % subsubsection
    subsubsection-family         = \normalfont,
    subsubsection-style         = \large \bfseries,
    % paragraph
	paragraph-family 			=\normalfont,
    paragraph-style         = \bfseries \slshape,
    % subparagraph
	subparagraph-family 			=\normalfont,
    subparagraph-style         = \slshape,
    %%%%%%%%%%%%%%%%%%%%%%%%%%%%%%%%%%%%%%%%%%%%%%%%%%%%%%%%%%%%%%%%%%%%%%%%%%%
    % Tables
    %%%%%%%%%%%%%%%%%%%%%%%%%%%%%%%%%%%%%%%%%%%%%%%%%%%%%%%%%%%%%%%%%%%%%%%%%%%
    % Table title
    table-title-family         = \sffamily,
    table-title-style         = \bfseries \large,
    % Table header
    table-header-family         = \sffamily,
    table-header-style         = \bfseries,
    % Table body
    table-body-family         = \normalfont,
    table-body-style         = \small,
    %%%%%%%%%%%%%%%%%%%%%%%%%%%%%%%%%%%%%%%%%%%%%%%%%%%%%%%%%%%%%%%%%%%%%%%%%%%
    % Tip boxes
    %%%%%%%%%%%%%%%%%%%%%%%%%%%%%%%%%%%%%%%%%%%%%%%%%%%%%%%%%%%%%%%%%%%%%%%%%%%
    % Tip title
    tip-title-family         = \sffamily,
    tip-title-style         = \bfseries,
    % Tip body
    tip-body-family         = \normalfont,
    tip-body-style         = \small,
    %%%%%%%%%%%%%%%%%%%%%%%%%%%%%%%%%%%%%%%%%%%%%%%%%%%%%%%%%%%%%%%%%%%%%%%%%%%
    % Sidebars
    %%%%%%%%%%%%%%%%%%%%%%%%%%%%%%%%%%%%%%%%%%%%%%%%%%%%%%%%%%%%%%%%%%%%%%%%%%%
    % Sidebar title
    sidebar-title-family         = \sffamily,
    sidebar-title-style         = \bfseries\normalsize,
    % Sidebar body
    sidebar-body-family         = \normalfont,
    sidebar-body-style         = \small,
    %%%%%%%%%%%%%%%%%%%%%%%%%%%%%%%%%%%%%%%%%%%%%%%%%%%%%%%%%%%%%%%%%%%%%%%%%%%
    % Read-aloud boxes
    %%%%%%%%%%%%%%%%%%%%%%%%%%%%%%%%%%%%%%%%%%%%%%%%%%%%%%%%%%%%%%%%%%%%%%%%%%%
    narration-family         = \normalfont,
    narration-style         = \small,
	%%%%%%%%%%%%%%%%%%%%%%%%%%%%%%%%%%%%%%%%%%%%%%%%%%%%%%%%%%%%%%%%%%%%%%%%%%%
    % Abstract
    %%%%%%%%%%%%%%%%%%%%%%%%%%%%%%%%%%%%%%%%%%%%%%%%%%%%%%%%%%%%%%%%%%%%%%%%%%%
	abstract-title-family = \Large,
	abstract-title-style = \scshape\bfseries,
	abstract-body-family = \normalfont,
	abstract-body-style = \small\slshape,
    %%%%%%%%%%%%%%%%%%%%%%%%%%%%%%%%%%%%%%%%%%%%%%%%%%%%%%%%%%%%%%%%%%%%%%%%%%%
    % Table of Contentss
    %%%%%%%%%%%%%%%%%%%%%%%%%%%%%%%%%%%%%%%%%%%%%%%%%%%%%%%%%%%%%%%%%%%%%%%%%%%
    % Part
    toc-part-family         = \normalfont,
    toc-part-style         = \Large,
    % Chapter
    toc-chapter-family         = \normalfont,
    toc-chapter-style         = \large,
    % Section
    toc-section-family         = \normalfont,
    toc-section-style         = \normalsize,
    %%%%%%%%%%%%%%%%%%%%%%%%%%%%%%%%%%%%%%%%%%%%%%%%%%%%%%%%%%%%%%%%%%%%%%%%%%%
    % Stat blocks
    %%%%%%%%%%%%%%%%%%%%%%%%%%%%%%%%%%%%%%%%%%%%%%%%%%%%%%%%%%%%%%%%%%%%%%%%%%%
    % Stat block title
    stat-block-title-family         = \normalfont,
    stat-block-title-style         = \bfseries \LARGE,
    % Stat block body
    stat-block-body-family         = \normalfont,
    stat-block-body-style         = \small,
    % Stat block section
    stat-block-section-family         = \sffamily,
    stat-block-section-style         = \large,
    %%%%%%%%%%%%%%%%%%%%%%%%%%%%%%%%%%%%%%%%%%%%%%%%%%%%%%%%%%%%%%%%%%%%%%%%%%%
    % Miscellaneous
    %%%%%%%%%%%%%%%%%%%%%%%%%%%%%%%%%%%%%%%%%%%%%%%%%%%%%%%%%%%%%%%%%%%%%%%%%%%
    % Footer
    footer-family         = \normalfont,
    footer-style         = \scriptsize,
    % Page number
    page-number-family         = \normalfont,
    page-number-style         = \scriptsize ,
    % Drop caps
    drop-cap-family         = \normalfont,
    drop-cap-internal-style         = \scshape,
	emph-family 		= \normalfont,
	emph-style			=  \bfseries\slshape,
}

| and it will know to first check for the file relative to rpg.sty; even if the package resides within the texmf path and the user has no idea where \texttt{rpgroot/rpg.sty}, or \texttt{rpgroot/core/font.sty}, are.

			An annoying exception to this is fonts and typefaces. \texttt{xelatex} searches for fonts based on \textit{filepaths relative to the current working directory} -- or from those installed in as system fonts.

			Since \texttt{rpgtex} includes several (license free) typefaces as part of the provided themes, this poses a problem. We must either require:
			\begin{enumerate}
				\item \rpgtex{} documents can only be prepared in restricted locations relative to the install location of \rpgtex{}.
				\item Users must identify and specify the \rpgtex{} root path when preparing a document
				\item \rpgtex{} must be configured to know `where it is', and so provide an absolute filepath to the internal fonts.
			\end{enumerate}
			The Configuration step is the easiest-to-use of these options.
		\end{RpgTip}


\chapter{Commands}
