\chapter{Environments}
	\def\tcref{\forcelink{https://ctan.org/pkg/tcolorbox?lang=en}{tcolorbox}}
	\newcommand\labelsection[1]
	{
		\section{#1}\label{S:#1}
	}

	
	\labelsection{Rpg Boxes}

		\rpgtex{} defines three `colorbox' environments, which inherit from \tcref{}.

		\begin{macrolist}
			\RpgMacro*[RpgNarration]{RpgNarration,{{o}}}{A \tcref{} wrapper designed for text that is read aloud to players}{
				\verb|\begin{RpgNarration}[color=<color>,<tcbox-options>]|

				~~~~\verb|<text>|
				
				\verb|\end{RpgNarration}|
			}{
				RpgNarration does not (by default) set a title, using only `body text', which is typeset using the \verb|RpgFontNarration| font. The optional \verb|<tcbox-options>| argument can be a list of all the basic tcolorbox options (see that documentation). The \verb|color| argument is an alias for \verb|colback| (\verb|colbacktitle| is also set, but is ignored as the title is empty). Due to the order of processing, if both \verb|color| and \verb|colback| are set, the value of \verb|colback| is used.

				Themes may alter the appearance of the narration block using the tcb interface, calling \verb|\tcbset{rpgnarration /.append~style={...}}| to overwrite the existing instructions.
			}
			\RpgMacro*[RpgSidebar]{RpgSidebar,{{o m}}}{A decorated \tcref{} wrapper designed for information which is set outside the main text.}{
				\verb|\begin{RpgSidebar}[color=<color>,<tcbox-options>]{<title>}|

				~~~~\verb|<text>|
				
				\verb|\end{RpgSidebar}|
			}{
				RpgSidebar requires a title (using \verb|RpgFontSidebarTitle|) as well as the body text (\verb|RpgFontSidebarBody|). 	RpgSidebar is typically more highly decorated than RpgTip, and does not have the \verb|breakable| flag set. It is usually best to use one of the `float' options.
				
				The optional \verb|<tcbox-options>| argument can be a list of all the basic tcolorbox options (see that documentation). The \verb|color=x| argument is equivalent to calling both \verb|colback=x| and \verb|colbacktitle=x|. Due to the order of processing, if both \verb|color| and \verb|colback| are set, the value of \verb|colback| is used.
		 

				Themes may alter the appearance of the sidebar using the tcb interface, calling \verb|\tcbset{rpgsidebar /.append~style={...}}| to overwrite the existing instructions.
			}
			\RpgMacro*[RpgTip]{RpgTip,{{o m}}}{A simple \tcref{} wrapper designed for information which is set outside the main text.}{
				\verb|\begin{RpgTip}[color=<color>,<tcbox-options>]{<title>}|

				~~~~\verb|<text>|
				
				\verb|\end{RpgTip}|
			}{
				RpgTip is similar to RpgSidebar, requiring a title (\verb|RpgFontTipTitle|) in addition to the body text (\verb|RpgFontTipBody|). However, it is generally simpler, enabling it to safely break over page boundaries. The optional \verb|<tcbox-options>| argument can be a list of all the basic tcolorbox options (see that documentation). The \verb|color=x| argument is equivalent to calling both \verb|colback=x| and \verb|colbacktitle=x|. Due to the order of processing, if both \verb|color| and \verb|colback| are set, the value of \verb|colback| is used.

				Themes may alter the appearance of the narration block using the tcb interface, calling \verb|\tcbset{rpgnarration /.append~style={...}}| to overwrite the existing instructions.
			}
		\end{macrolist}

		

		\subsection{Which Colorbox To Use?}

			The choice between RpgSidebar and RpgTip is somewhat arbitrary -- although they have a mechanical difference by default (one being breakable, the other not) -- this can be overridden by themes. Instead, the intention is that they serve slightly different purposes:

			\begin{description}
				\item[RpgSidebar] is used for `important information' -- key rules or summaries which readers \textit{should} pay attention to.
				\item[RpgTip] is for `helpful additions' -- tips, tricks and trivia that are not necessary, but which might be useful, and are too big to fit into a footnote or parenthetical.   
			\end{description}
		\subsection{Colorbox Examples}
		\DualExampleRender{narration1}
	
	\labelsection{RpgTable}



		\begin{macrolist}
			\RpgMacro*[RpgTable]{RpgTable,{{o m}}}{Begins an environment for creating visually appealing and consistent tables. }{
				\verb|\begin{RpgTable}[<options>]{<column-specifications}|

				~~~~\verb|<table-contents>|
				
				\verb|\end{RpgTable}|
			}{
				RpgTable is a wrapper for the \forcelink{https://ctan.org/pkg/tabularx}{tabularx} (or \forcelink{https://ctan.org/pkg/xltabular}{xltabular} -- see \verb|breakable|) environment, and so accepts the standard set of column specifications:\{c,l,r,p{width},\@,...\} and the extended set (i.e. X). It therefore acts almost identically to the standard tabular environment with a few stylistic differences.
				\macrosection{Stylistic Changes}
					The RpgTable environment makes the following changes:
					\begin{enumerate}
						\item \textbf{Title.} If the \verb|title| option is set, a title-heading is rendered above the tablular in the \verb|RpgFontTableTitle| font.
						\item \textbf{Auto-headings.} The first row of the tabular environment is automatically rendered in the \verb|RpgFontTableHeader|, allowing for trivial header labels.
						\item \textbf{Font Integration.} The main body of the table is rendered in \verb|RpgFontTableBody| font.
						\item \textbf{Auto-colouring.} The rows alternate between being transparent and being set to the \verb|tablecolor| variable \RpgPage[p]{S:Colors}. This is powered by \verb|rowcolors|.
					\end{enumerate}
				\macrosection{Optional Arguments}
				\begin{description}
					\item[width=<dimexpr>] Fixes the width of the tabular environment to the value of this argument. Default value is the current \verb|\linewidth|.
					\item[color=<color-name>] If set, uses this value instead of \verb|tablecolor| for the alternating coloration.
					\item[title=<text>] Sets the text to be rendered as the title of the table. 
					\item[breakable] If flag is present, renders using \verb|xltabular|, enabling the table to break over pages. \textbf{only available in 1-column mode (a fundamental limitation of xltabular).}
					\item[noheader] If flag is present, suppresses the autoformatting of the title. The first row is instead rendered in the body formatting.
				\end{description}
				% The available options are as follows:
			}
		\end{macrolist}
		\subsection{RpgTable Example}

		This is the standard usage of the table, showing automatic formatting of the header rows and the word-wrapping abilities of the X-column:

		\DualExampleRender{table1}

		This example adds a title, but suppresses the header formatting:

		\DualExampleRender{table2}
		

