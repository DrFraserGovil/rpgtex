\chapter{rpghandout Class}\label{S:handoutClass}

	The rpghandout class is designed for smaller documents where the full structure of a book is unnecessary, such as printouts for players, or individual adventure modules.
	
		
	\section*{Features}

		\subsection{Inherited Class}

			The rpghandout class inherits from \forcelink{https://ctan.org/pkg/extsizes}{the extarticle class}. This is an extension to the basic \texttt{article} class to allow more fontsizes to be accepted. Otherwise it behaves near-identically to the standard article class.

			The full list of sizes which extarticle can accept (and thus allowed inputs for the \texttt{size} option \RpgPage[p]{size}) is ``eight, nine, ten, eleven, twelve, fourteen, seventeen and twenty points''.

			\subsubsection{Special Commands}

				\RpgMacro*{abstract}{\param{O\param{Abstract}},summary}{Defines a block of text which (if in twocolumn mode) spans both columns, serving as a summary of the document.}
				{
					\cmd{begin\param{abstract}}[<abstract-name>] ~~~\% (or \cmd{begin\param{summary}})
					
					~~<abstract-contents>

					\cmd{end\param{abstract}}
				}
				{				
					We have redefined the abstract environment slightly so that it renders almost identically in both twocolumn and onecolumn mode:

					The optional argument determines the `header' text which is printed above the abstract text. This is centered and uses the \cmd{RpgFontAbstractTitle} font, whilst the body text uses \cmd{RpgFontAbstractBody}. The text is placed into a parbox which is 70\% the line width.

					The \texttt{summary} variant is identical, but uses the default header value of `Summary', which might be more familiar to non-technical writers.\cmdidx{summary}
				}
			
		\subsection{Options}

			The rpghandout interacts with all of the options detailed on \RpgPage{S:PackageOptions}. Note that there is no `forwarding' to the underlying class and that there is a slightly different syntax for, i.e., setting the global font size. 

		\subsection{Geometry}

			The default geometry for an rpghandout is:
			\begin{RpgTable}{ll}
				Element & Size
				\\
				Left and right margin & 0.65in
				\\
				Top margin & 0.4in
				\\
				Bottom margin (from main text to page bottom) & 0.75in
				\\
				Bottom margin (from main text to top of footer area) & 0.3in 
				\\
				Gap between columns in twocolumn mode & 0.25in
			\end{RpgTable}
		\subsection{Interactions}

			\begin{itemize}
				\item rpghandout set \cmdref{RpgUseCoverPage} to false
				\item The \texttt{layout} mode is activated
				\begin{enumerate}
					\item Unless print mode is active, the page background will use the image set by \cmdref{RpgSetPaper}
					\item Calling \cmd{RpgSetTheme} clears the page (so that the old theme may complete)
				\end{enumerate}
				\item A table of contents is available and formatted using the ToC-fonts
			\end{itemize}

