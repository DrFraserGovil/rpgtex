\RpgSetTheme{dnd}
\chapter{\texttt{dnd} Theme}


	\section{dnd RpgItem}\label{S:DndItem}\index{rpgitem@\texttt{RpgItem}!\texttt{dnd}}

awidj oaiwhd oiawdio aoiwdjoa ijwdijaiowj doi jaiowjd oija woijd ioajw d his is a test of \emph{My emphasis} adh ouawwhd iuahwiud hwaiudh iuawhdu iahwiudh iuwahdiu ahw iudhiuahw iudhaw iduh awliudh lauiwhdl iuahw idulha wiuldhiuhwiuh 

	\section{RpgMonster}\cmdidx{RpgMonster}

		The dnd theme defines a special command which mimics the appearance of a monster statblock - particularly those in the more modern D\&D 2024 iteration. 

	% 	\begin{multicols}{2}
	% 	\begin{RpgMonster}{Test}
			
	% 	\end{RpgMonster}
	% 	\clearpage
	% 	\Blindtext
	% \end{multicols}

	\labelsection{RpgSpell Environment}
	\RpgUseCards{true}

	\begin{RpgSpell}{Hocus Pocus}{
		level = 3,
		school=Transmutation,
		casting-time={1 bonus action\footnote{Taken whilst in waist-deep water}},
		source={Player's Handbook}
	}
		The body text of the spell

	\end{RpgSpell}

	\begin{RpgSpell}[]{Hocus Pocus 2}{
		% level = 1,
		% school=Conjuration
		casting-time={1 Reaction}
	}
		The body text of the \cardbreak spell
		% \Blindtext
	\end{RpgSpell}
	
	\begin{RpgSpell}[width=8cm]{Hocus Pocus 3}{
		% level = 1,
		% school=Conjuration
		casting-time={1 Reaction}
	}
		The body text of the \cardbreak spell
		% \Blindtext
	\end{RpgSpell}

	Now I test \footnote{That I didn't damage anything}.

	% \newenvironment{testit}[1]{\textit{#1}\newline}{}
	% \newenvironment{testbf}[1]{\textbf{#1}\newline}{}




	% \begin{RpgCardSwitch}{testit}{testbf}{test this!}
	% 		Some text
	% \end{RpgCardSwitch}

	% \RpgCardMode{true}

	% \begin{RpgCardSwitch}[test]{testit}{testbf}{test this!}
	% 		Some text
	% \end{RpgCardSwitch}
	
