\RpgSetTheme{dnd}
\chapter{\texttt{dnd} Theme}\label{Theme:dnd}

	The \texttt{dnd} theme is designed to mimic the Appearance of the \textit{Dungeons \& Dragons} source books. The specific parameters used to replicate the D\&D books are mostly derived from the original DnD-5e-Latex-Template package, with some additional updates in line with the changes made with the 2024 rules update.

    \section{Appearance}
        \subsection{Fonts}
            The dnd theme defines a number of font families which are used throughout the theme.
            
            \RpgMacro{bookman}{}{The {Bookman Old Style STd} typeface, used as the main body text.}{}{}{}
            \RpgMacro{keplerserif}{}{\keplerserif The {KpRoman} font, used for \textbf{emphasis}}{}{}{}
            \RpgMacro{kepler}{}{\kepler The {KpSans} font, primarily used in headers and titles}{}{}{}
            \RpgMacro{gillius}{}{\gillius The Gillius ADF No. 2 Font, used as a lighter sans-serif font, often in the body of environments where \cmd{kepler} was used as the title.}{}{}{}

             The values assigned to the font elements areas follows (note that \verb|\normalfont| is aliased to the value of RpgFontBody).

            \InsertFontTable{
                main-body-family			={\bookman},
				%Sections
				title-family=		{\normalfont},
				title-style			= {\bfseries\fontsize{30}{50}\selectfont\color{titlered}},
				subtitle-family		={\scshape},
				subtitle-style 		= {\LARGE\color{titlered} },
				part-style			= {\color{titlered}\fontsize{50}{30}\selectfont},
				chapter-style		= {\titlemode\fontsize{30}{50}\selectfont},
				section-family		={\keplerserif \scshape},
				section-style		= {\color{titlered} \Huge},
				subsection-family		={\keplerserif \scshape},
				subsection-style	= {\color{titlered} \huge},
				subsubsection-family		={\keplerserif \scshape},
				subsubsection-style	= {\color{titlered} \Large},
				paragraph-style	= {\bfseries \slshape},
				subparagraph-style	= {\bfseries \slshape},
				%Tables
				table-title-family	= {\kepler},
				table-title-style	= {\bfseries\large\scshape},
				table-header-family	= {\kepler},
				table-body-family	= {\gillius},
				%tip boxes
				tip-title-family= {\kepler},
				tip-title-style	= {\bfseries\scshape},
				tip-body-family	= {\gillius},
				tip-body-style	= {\small},
				%sidebars
				sidebar-title-family= {\kepler},
				sidebar-title-style	= {\bfseries\scshape},
				sidebar-body-family	= {\gillius},
				sidebar-body-style	= {\small},
				%narration
				narration-family	= {\gillius},
				narration-style		= {\small},
				% TOC
				toc-part-style		= {\LARGE \keplerserif \scshape \color{titlered}},
				toc-chapter-style	= {\Large \keplerserif \scshape \color{titlered}},
				toc-section-style	= {\normalsize},
				%stat block
				stat-block-title-style= {\bfseries\scshape \LARGE},
				stat-block-body-family={\gillius},
				stat-block-body-style={\small},
				stat-block-section-family={\kepler},
				stat-block-section-style={\color{titlered} \scshape \large},
				%%misc
				footer-style		= {\color{pagegold}\scriptsize},
				page-number-style	= {\color{pagegold}\scriptsize},
				drop-cap-family		= {\Royal},
				%%Spell card
				card-title-family = {\keplerserif},
				card-title-style = { \color{titlered}\large\bfseries},
				card-header-family = {\normalfont}, 
				card-header-style = {\normalfont \scshape \footnotesize}, 
				card-body-family = {\normalfont},
				card-body-style = {\footnotesize},
				emph-family = {\keplerserif},
				emph-style = {\bfseries\scshape\larger[1]},
            }
            
        \subsection{Colors}

            The theme provides a large number of colors:

            \noindent\DefaultSwatches{}

			\noindent\ignorespaces\Swatch[default narration]{bgtan}
			\Swatch[pagenumbers \& footer]{pagegold}
			\Swatch[title font \& default contour-inner]{titlered}
			\Swatch[default contour-outer]{contourgray}
			\Swatch[title rules]{titlegold}
			\Swatch[statblock triangles]{rulered}
			\Swatch{statblockoutline}
			\Swatch[top/bottom line (2014 only)]{statblockribbon}
			\Swatch{statblockbg}
			\Swatch[Basic Rules]{BrGreen}
			\Swatch[PHB Part 1]{PhbLightGreen}
			\Swatch[PHB Part 2]{PhbLightCyan}
			\Swatch[PHB Part 3]{PhbMauve}
			\Swatch[PHB Appendix]{PhbTan}
			\Swatch[DMG Part 1]{DmgLavender}
			\Swatch[DMG Part 2]{DmgCoral}
			\Swatch[DMG Part 3]{DmgSlateGray}
			\Swatch[DMG Appendix]{DmgLilac}

			The colors from BrGreen through DmgLilac are chosen so that, when set as the \texttt{themecolor} (using \cmdref{RpgSetThemeColor}), the appearance of the tables and sidebars is the same as the corresponding part in either the Basic Rules, the Players' Handbook, and the Dungeon Master's Guide.
        \subsection{Backgrounds \& Footers}

			The dnd theme uses a paper-like image as the background, and places a `scroll' on the footer\footnote{The scroll alternates direction in \texttt{twoside} documents; this document is written in oneside mode, so there is no alternating.}. The chapter name is placed in the footer (in \texttt{pagegold} color). The page number is positioned to lie in a divot of the scroll.

        \subsection{Text Boxes}

            The dnd theme modifies the text boxes to give them the following appearance:

            \begin{multicols}{3}
                
                \begin{RpgSidebar}{The RpgSidebar}
                   
					Horizontal `ribbons' and sharp corners differentiate it from RpgTip.
                \end{RpgSidebar}
                
                \begin{RpgTip}{The RpgTip}
                    Almost entirely undecorated, with rounded corners.
                \end{RpgTip}

                \begin{RpgNarration}
                    RpgNarration has `bars' on the side.
                \end{RpgNarration}
            \end{multicols}



        \subsection{Section Headers}

               As in the default mode, only chapters\footnote{If in an rpgbook / other class which support chapters} are numbered; all other section/subsections (etc.) are unnumbered. 

				Chapters are rendered using a \cmd{RpgContour} (with the default colors). The subsection environment is embellished with a \cmd{hrule} (in the \texttt{titlegold} color) which stretches across the page.

              \begin{ExampleBlock}{Section Headers}
    \section*{Example} %starred so not added to toc; otherwise the same
    \subsection{Another Example}
    \subsubsection{More Examples!}
            \end{ExampleBlock}

        \subsection{RpgDice}\index{RpgDice}
            
            Following the syntax found in D\&D monster statblocks, RpgDice shows the average of the roll:
            \begin{ExampleBlock}{Default Dice}
            \RpgDice{3d8 -5 +2}
            \end{ExampleBlock}

    \newpage
    \section{RpgItem}\label{S:DefaultItem}\index{RpgItem!Theme!default}

        The default configuration of the \envref{RpgItem} environment, used to describe physical objects and equipment. As with all FeatureForge environments, has the ability to switch between `text mode' and `card mode' depending on the value passed to \cmd{RpgItemShowCard} \RpgPage[p]{Macro:Rpg[X]ShowCard}. 
        
        The \texttt{default} theme defines two keys for the RpgItem object:
        \begin{RpgTable}{llX}
            Key & Default Value & Effect
            \\
            description & \param{} & An italicised byline used to summarise the item
            \\
            image & \param{} & If non-empty, define an image path which is used when in card-mode.
        \end{RpgTable}
    
    \begin{ExampleBlock}{Default RpgItem: Text Mode}
    \RpgItemShowCard{false}
    \begin{RpgItem}{Joyeuse}[
        description={The Sword Jewellous},
		rarity=Rare,
		type=weapon (sword),
		requires-attunement={by a paladin.},
        image={../example/img/joyeuse}
    ]
        The sword of Charlemagne; this jewelled sword gives you +3 to heroism checks.
    \end{RpgItem}
    \end{ExampleBlock}
    \begin{ExampleBlock}{Default RpgItem: Card Mode}
    \RpgItemShowCard{true}
    \begin{RpgItem}{Joyeuse}[
        description={The Sword Jewellous},
        image={../example/img/joyeuse}
    ]
        The sword of Charlemagne; this jewelled sword gives you +3 to heroism checks.
    \end{RpgItem}
    \end{ExampleBlock}

\newpage
\section{RpgFeat}\index{RpgFeat!Theme!default}
     The default configuration of the \envref{RpgFeat} environment, used to describe abilities and character features and choices. As with all FeatureForge environments, has the ability to switch between `text mode' and `card mode' depending on the value passed to \cmd{RpgItemShowCard} \RpgPage[p]{Macro:Rpg[X]ShowCard}. 
        
      The \texttt{default} theme defines two keys for the RpgFeat object:
        \begin{RpgTable}{llX}
            Key & Default Value & Effect
            \\
            prerequisite & \param{} & If non-empty, an italicised note is added, indicating the prerequisites for acquiring the ability.
        \end{RpgTable}
     \begin{ExampleBlock}{Default RpgFeat: Text Mode}
    \RpgFeatShowCard{false}
    \begin{RpgFeat}{Hyperattack}[
         prerequisite={Mega-attack},
    ]

        You can unleash superhuman speed against your enemies. Once per day, make \RpgDice{3d10} additional attacks.
    \end{RpgFeat}
    \end{ExampleBlock}
    \begin{ExampleBlock}{Default RpgFeat: Card Mode}
    \RpgFeatShowCard{true}
    \begin{RpgFeat}{Hyperattack}[
        prerequisite={Mega-attack},
    ]

        You can unleash superhuman speed against your enemies. Once per day, make \RpgDice{3d10} additional attacks.
    \end{RpgFeat}
    \end{ExampleBlock}
