\chapter{Fonts}
	\begin{RpgSidebar}{Family vs Style}
		When defining the Font for an element, the interface allows one to specify both a \texttt{family} and a \texttt{style}. Formally speaking, \texttt{family} defines the \textbf{typeface} used by the associated element, whilst the \texttt{style} determines the options passed to that typeface (bold, italics, size etc.).

		The distinction is largely irrelevant, as the construction of the final font object is often simply the concatenation of the two:
		\begin{lstlisting}
			\def\RpgFontX
			{
				\l__rpg_x_family \l__rpg_x_style
			}
		\end{lstlisting}
		The separate definitions is therefore largely a matter of clarity and readability. It is generally fine to place commands should be `family' in the `style' element.
	\end{RpgSidebar}
	\section{Font Names}

	\section{Contours}
		
		\begin{macrolist}
			\RpgMacro{\RpgContour,{{O{} m}}}
				{
					Renders text with a \RpgContour[inner=red,outer=black]{contour effect}. The color and style are set through key/value pairs.
				}
				{
					\RpgContour[inner=<color>,outer=<color>,style=<code>]{<text>}
				}
				{
					The \texttt{style} command is applied to the text, whilst the optional \texttt{inner} and \texttt{outer} commands set the base text colour and the external contour color respectively. If the colors are not set, the default values are the \verb|contourinnercolor| and \verb|contouroutercolor| values defined by the theme \RpgPage[p]{S:Colors}.
					
					The contour does not automatically linebreak, but can be controlled manually with a \verb|\newline| command (not \verb|\\| or \texttt{\textbackslash{}par})
					
					\begin{RpgTable}{lX}
						Example & Output \\
						\tabverbExample{\RpgContour[inner=red,outer=black]{example}}
						\tabverbExample{\RpgContour[style=\Huge\it]{example}}
						\tabverbExample{\RpgContour[]{multi\newline line\newline example}}
					\end{RpgTable}					
					~\macrosection{Quirks}

					Due to the tokenisation required for the line-splitting and space-preservation, the text inside the contour can exhibit some quirks if stylisation is applied within the \verb|<text>| argument. 

					Unbraced commands (such as \verb|\it| or \verb|\footnotesize|) will only apply to the first word in the text. Braced commands \textit{can} work, but will cause a compilation error if a \verb|\newline| is included. 

					
					\begin{RpgTable}{lX}
						\footnotesize\tabverbExample{\RpgContour[]{\Huge\it only first word changes}}
						\footnotesize\tabverbExample{\RpgContour[]{\textit{all words change}}}
						\footnotesize\verb|\RpgContour[]{\textit{all word \newline change}}| & (fails to compile)
					\end{RpgTable}
					For robustness, we therefore recommend that all stylisation be applied through the \verb|style| command, which is applied to each tokenised element, and therefore guaranteed to work as expected.
				}
		\end{macrolist}
