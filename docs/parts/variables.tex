\chapter{Variables}

	\rpgtex{} defines many dozens to hundreds of variables, most with the (expl3) syntax \verb|\l__rpg_[x]|. Most of these are used in the internal functioning of the macros, however a number of them are useful for a designer to understand.

	\section{Colo(u)rs}\label{S:Colors}

		\rpgtex{} by default defines a number of colors\footnote{Yes, I hate myself, but we're going with the code-based spelling.} which are used for different elements:
		\begin{description}
			\item[themecolor] A `basic color' which is (by default) equal to the following three colors:
			\begin{enumerate}
				\item \textbf{sidebarcolor} The background color of the \verb|RpgSidebar| environment
				\item \textbf{tablecolor} The background color of every other row in an \verb|RpgTable| 
				\item \textbf{tipcolor} The background color of the \verb|RpgTip| environment 
			\end{enumerate} 
			\item[narrationcolor]  The background color of the \verb|RpgTip| environment 
			\item[contourinnercolor]  The default color of the inner text within a \verb|RpgContour| command 
			\item[contouroutercolor]  The default color of the external contour drawn around text within a \verb|RpgContour| command.
		\end{description}

		Calling \verb|\RpgSetThemeColor| \RpgPage[p]{Macro:RpgSetThemeColor} updates the value of \verb|themecolor|, as well as the three `co-varying' colors (i.e. everything except \verb|narrationcolor|). Other colors are modified simply using the xcolors interface:

		\verb|\colorlet{narrationcolor}{html}{FFFFF}|



	\section{Command Line Interface}\label{S:CMD}

		By default, \LaTeX{} does not have a `command line interface' which allows a user to modify the document from within the command line: changes to the document have to be placed inside the file, and then compiled. However, we found that -- particularly with the \textit{print} option (which suppresses background images on the paper, reducing ink requirements for printing), it was convenient to be able to compile the same document in either `normal' mode, or `print mode', without modifying the text.

		To this end, we have provided a method for pseudo-`command line variables' to be inserted into the RpgOptions module. To do this, we exploit the fact that \TeX{} can read documents from an input stream, not just files.


		\begin{macrolist}
			\RpgMacro[RpgCMD]{\RpgCMD}{Holds key-value pairs to be inserted into RptOptions after the standard parsing is run, ideal for command line modification.}{
				xelatex ``\def\RpgCMD{<rpg-options>} \input{<document}''
			}{
				This will compile the \verb|<document>|, with the contents of \verb|RpgCMD| parsed as if they had been placed into \verb|\documentclass[<rpg-options>]{rpgclass}| or when invoking the package: \verb|\usepackage[rpg-options]{rpgtex}|.

				Values passed to \verb|RpgCMD| will override values passed to the package the standard way. 
			}
		\end{macrolist}

		The \verb|rpgtex| compiler which we have provided \RpgPage[p]{C:Compiler} performs this insertion by default for several predefined variables:

		\begin{lstlisting}
			rpgtex document.tex -p    (alias for  xelatex "\def\RpgCMD{bg=print} \input document.tex")
		\end{lstlisting}

		Thereby allowing the user to switch between \verb|print| and \verb|full| mode with a compiler switch.

		
