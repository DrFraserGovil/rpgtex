\chapter{Variables}
	\section{Colo(u)rs}\label{S:Colors}

		\rpgtex{} by default defines a number of colors\footnote{Yes, I hate myself, but we're going with the code-based spelling.} which are used for different elements:
		\begin{description}
			\item[themecolor] A `basic color' which is (by default) equal to the following three colors:
			\begin{enumerate}
				\item \textbf{sidebarcolor} The background color of the \verb|RpgSidebar| environment
				\item \textbf{tablecolor} The background color of every other row in an \verb|RpgTable| 
				\item \textbf{tipcolor} The background color of the \verb|RpgTip| environment 
			\end{enumerate} 
			\item[narrationcolor]  The background color of the \verb|RpgTip| environment 
			\item[contourinnercolor]  The default color of the inner text within a \verb|RpgContour| command 
			\item[contouroutercolor]  The default color of the external contour drawn around text within a \verb|RpgContour| command.
		\end{description}

		Calling \verb|\RpgSetThemeColor| \RpgPage[p]{Macro:SetThemeColor} updates the value of \verb|themecolor|, as well as the three `co-varying' colors (i.e. everything except \verb|narrationcolor|). When \verb|printmode| is active \verb|\RpgSetThemeColor{white}| is called, making environments transparent.
