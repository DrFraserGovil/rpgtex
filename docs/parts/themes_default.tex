\RpgSetTheme{default}
\chapter{\texttt{default} Theme}\label{Theme:Default}

    As the name suggests, the \texttt{default} theme is the theme which is loaded if no arguments are passed to \texttt{theme} when loading the \rpgtex{} package. As such, it is largely a re-statement of the default \LaTeX{} values. This is important, as most other themes begin by first re-inserting the default theme; thereby resetting everything into a `known state'. This means that they only need to define formatting which differs from the default. 
    
    The majority of this document has been typeset in a rpgbook class \RpgPage[p]{S:bookClass} using the default theme.

    \section{Appearance}
        \subsection{Fonts}
            The default format defines only a single font: the Latin Modern Roman font, the standard \LaTeX{} font. For consistency with the rest of the library (which uses fontspec-defined fonts), we have used an OTF version of LMR which is bundled with the package.
            
            \RpgMacro{lrfont}{}{The name given to the Latin Modern Roman font-family (extended with a small-caps font; Latin Modern Roman Caps) defined by the default theme}{}{}{}

             The values assigned to the font elements areas follows (note that \verb|\normalfont| is aliased to the value of RpgFontBody).

            \InsertFontTable{
                main-body-family = \lrfont,
                main-body-style= {},
                title-family         = \normalfont,
                title-style         = \Huge,
                % Subtitle
                subtitle-family         = \normalfont,
                subtitle-style         = \Large,
                % Part
                part-family         = \normalfont,
                part-style         = \Huge,
                % Chapter
                chapter-family         = \normalfont,
                chapter-style         = \Huge \bfseries,
                % Section
                section-family         = \normalfont,
                section-style         = \huge \bfseries,
                % Subsection
                subsection-family         = \normalfont,
                subsection-style         = \Large \bfseries,
                % subsubsection
                subsubsection-family         = \normalfont,
                subsubsection-style         = \large \bfseries,
                % paragraph
                paragraph-family 			=\normalfont,
                paragraph-style         = \bfseries \slshape,
                % subparagraph
                subparagraph-family 			=\normalfont,
                subparagraph-style         = \slshape,
                %%%%%%%%%%%%%%%%%%%%%%%%%%%%%%%%%%%%%%%%%%%%%%%%%%%%%%%%%%%%%%%%%%%%%%%%%%%
                % Tables
                %%%%%%%%%%%%%%%%%%%%%%%%%%%%%%%%%%%%%%%%%%%%%%%%%%%%%%%%%%%%%%%%%%%%%%%%%%%
                % Table title
                table-title-family         = \sffamily,
                table-title-style         = \bfseries \large,
                % Table header
                table-header-family         = \sffamily,
                table-header-style         = \bfseries,
                % Table body
                table-body-family         = \normalfont,
                table-body-style         = \small,
                %%%%%%%%%%%%%%%%%%%%%%%%%%%%%%%%%%%%%%%%%%%%%%%%%%%%%%%%%%%%%%%%%%%%%%%%%%%
                % Tip boxes
                %%%%%%%%%%%%%%%%%%%%%%%%%%%%%%%%%%%%%%%%%%%%%%%%%%%%%%%%%%%%%%%%%%%%%%%%%%%
                % Tip title
                tip-title-family         = \sffamily,
                tip-title-style         = \bfseries,
                % Tip body
                tip-body-family         = \normalfont,
                tip-body-style         = \small,
                %%%%%%%%%%%%%%%%%%%%%%%%%%%%%%%%%%%%%%%%%%%%%%%%%%%%%%%%%%%%%%%%%%%%%%%%%%%
                % Sidebars
                %%%%%%%%%%%%%%%%%%%%%%%%%%%%%%%%%%%%%%%%%%%%%%%%%%%%%%%%%%%%%%%%%%%%%%%%%%%
                % Sidebar title
                sidebar-title-family         = \sffamily,
                sidebar-title-style         = \bfseries\normalsize,
                % Sidebar body
                sidebar-body-family         = \normalfont,
                sidebar-body-style         = \small,
                %%%%%%%%%%%%%%%%%%%%%%%%%%%%%%%%%%%%%%%%%%%%%%%%%%%%%%%%%%%%%%%%%%%%%%%%%%%
                % Read-aloud boxes
                %%%%%%%%%%%%%%%%%%%%%%%%%%%%%%%%%%%%%%%%%%%%%%%%%%%%%%%%%%%%%%%%%%%%%%%%%%%
                narration-family         = \normalfont,
                narration-style         = \small,
                %%%%%%%%%%%%%%%%%%%%%%%%%%%%%%%%%%%%%%%%%%%%%%%%%%%%%%%%%%%%%%%%%%%%%%%%%%%
                % Abstract
                %%%%%%%%%%%%%%%%%%%%%%%%%%%%%%%%%%%%%%%%%%%%%%%%%%%%%%%%%%%%%%%%%%%%%%%%%%%
                abstract-title-family = \Large,
                abstract-title-style = \scshape\bfseries,
                abstract-body-family = \normalfont,
                abstract-body-style = \small\slshape,
                %%%%%%%%%%%%%%%%%%%%%%%%%%%%%%%%%%%%%%%%%%%%%%%%%%%%%%%%%%%%%%%%%%%%%%%%%%%
                % Table of Contentss
                %%%%%%%%%%%%%%%%%%%%%%%%%%%%%%%%%%%%%%%%%%%%%%%%%%%%%%%%%%%%%%%%%%%%%%%%%%%
                % Part
                toc-part-family         = \normalfont,
                toc-part-style         = \Large\bfseries,
                % Chapter
                toc-chapter-family         = \normalfont,
                toc-chapter-style         = \large,
                % Section
                toc-section-family         = \normalfont,
                toc-section-style         = \normalsize,
                %%%%%%%%%%%%%%%%%%%%%%%%%%%%%%%%%%%%%%%%%%%%%%%%%%%%%%%%%%%%%%%%%%%%%%%%%%%
                % Stat blocks
                %%%%%%%%%%%%%%%%%%%%%%%%%%%%%%%%%%%%%%%%%%%%%%%%%%%%%%%%%%%%%%%%%%%%%%%%%%%
                % Stat block title
                stat-block-title-family         = \normalfont,
                stat-block-title-style         = \bfseries \LARGE,
                % Stat block body
                stat-block-body-family         = \normalfont,
                stat-block-body-style         = \small,
                % Stat block section
                stat-block-section-family         = \sffamily,
                stat-block-section-style         = \large,
                %%%%%%%%%%%%%%%%%%%%%%%%%%%%%%%%%%%%%%%%%%%%%%%%%%%%%%%%%%%%%%%%%%%%%%%%%%%
                % Miscellaneous
                %%%%%%%%%%%%%%%%%%%%%%%%%%%%%%%%%%%%%%%%%%%%%%%%%%%%%%%%%%%%%%%%%%%%%%%%%%%
                % Footer
                footer-family         = \normalfont,
                footer-style         = \scriptsize,
                % Page number
                page-number-family         = \normalfont,
                page-number-style         = \scriptsize ,
                % Drop caps
                drop-cap-family         = \normalfont,
                drop-cap-first-line-style         = \scshape,
                emph-family 		= \normalfont,
                emph-style			=  \bfseries\slshape,
                 card-title-family          =\normalfont,
                card-title-style =          \textbf,
                card-header-family =        \normalfont,
                card-header-style= \textit,
                card-body-family=\normalfont,
                card-body-style=\small,
            }
            
        \subsection{Colors}

            The default theme colors are largely monochromatic:

            \DefaultSwatches{}

        \subsection{Backgrounds \& Footers}
            The default theme defines no background or headers (and clears any existing ones). The footer consists of the chapter name and a page number, using the appropriate fonts (\cmd{RpgFontFooter} and \cmd{RpgFontPageNumber}). 


        \subsection{Text Boxes}

            The default text boxes are very simple in design:

            \begin{multicols}{3}
                
                \begin{RpgSidebar}{The RpgSidebar}
                    Has minor decorations at the top and bottom to distinguish it from an RpgTip
                \end{RpgSidebar}
                
                \begin{RpgTip}{The RpgTip}
                    Almost entirely undecorated
                \end{RpgTip}

                \begin{RpgNarration}
                    RpgNarration is also simple; but uses the narration color
                \end{RpgNarration}
            \end{multicols}



        \subsection{Section Headers}

                Only chapters\footnote{If in an rpgbook / other class which support chapters} are numbered; all other section/subsections (etc.) are unnumbered. Otherwise there are no changes made to the titles aside from minor spacing changes, and the assignment of the relevant fonts.
          

              \begin{ExampleBlock}{Section Headers}
    \section*{Example} %starred so not added to toc; otherwise the same
    \subsection{Another Example}
    \subsubsection{More Examples!}
            \end{ExampleBlock}

        \subsection{RpgDice}\index{RpgDice}
            
            The default RpgDice format is $n\mathrm d x + m$:
            \begin{ExampleBlock}{Default Dice}
            \RpgDice{3d8 -5 +2}
            \end{ExampleBlock}

    \newpage
    \section{RpgItem}\label{S:DefaultItem}

		\RpgMacro[RpgItem!Theme!default]{RpgItem}{\param{\paramO m \paramO}}{The default configuration of the \envref{RpgItem} environment, used to describe physical objects and equipment.}
			{
				\cmd{RpgItemShowCard\param{true/false}} \% set the card mode

				\cmd{begin}\param{RpgItem}[card-opts]\param{Item Name}[key-values]

				~~<body text>

				\cmd{end}\param{RpgItem}
			}
			{
			As with all FeatureForge environments, has the ability to switch between `text mode' and `card mode' depending on the value passed to \cmd{RpgItemShowCard} \RpgPage[p]{Macro:Rpg[X]ShowCard}. 
			
			The \texttt{default} theme defines two keys for the RpgItem object:
			\begin{RpgTable}{llX}
				Key & Default Value & Effect
				\\
				description & \param{} & An italicised byline used to summarise the item
				\\
				image & \param{} & If non-empty, define an image path which is used when in card-mode.
			\end{RpgTable}
			}
    \begin{ExampleBlock}{Default RpgItem: Text Mode}
    \RpgItemShowCard{false}
    \begin{RpgItem}{Joyeuse}[
        description={The Sword Jewellous},
        image={../example/img/joyeuse}
    ]
        The sword of Charlemagne; this jewelled sword gives you +3 to heroism checks.
    \end{RpgItem}
    \end{ExampleBlock}
    \begin{ExampleBlock}{Default RpgItem: Card Mode}
    \RpgItemShowCard{true}
    \begin{RpgItem}{Joyeuse}[
        description={The Sword Jewellous},
        image={../example/img/joyeuse}
    ]
        The sword of Charlemagne; this jewelled sword gives you +3 to heroism checks.
    \end{RpgItem}
    \end{ExampleBlock}

\newpage
\section{RpgFeat}
	\RpgMacro*[RpgFeat!Theme!default]{RpgFeat}{\param{\paramO m \paramO}}{ The default configuration of the \envref{RpgFeat} environment, used to describe abilities and character features and choices.}
		{
			\cmd{RpgFeatShowCard\param{true/false}} \% set the card mode

			\cmd{begin}\param{RpgFeat}[card-opts]\param{Feat Name}[key-values]

			~~<body text>

			\cmd{end}\param{RpgFeat}
		}
		{As with all FeatureForge environments, has the ability to switch between `text mode' and `card mode' depending on the value passed to \cmd{RpgFeatShowCard} \RpgPage[p]{Macro:Rpg[X]ShowCard}. 
				
			The \texttt{default} theme defines only a single key for the RpgFeat object:
			\begin{RpgTable}{llX}
			Key & Default Value & Effect
			\\
			requires & \param{} & If non-empty, an italicised note is added, indicating the prerequisites for acquiring the ability.
			\end{RpgTable}
		}
    \begin{ExampleBlock}{Default RpgFeat: Text Mode}
    \RpgFeatShowCard{false}
    \begin{RpgFeat}{Hyperattack}[
         requires={Mega-attack},
    ]

        You can unleash superhuman speed against your enemies. Once per day, make \RpgDice{3d10} additional attacks.
    \end{RpgFeat}
    \end{ExampleBlock}
    \begin{ExampleBlock}{Default RpgFeat: Card Mode}
    \RpgFeatShowCard{true}
    \begin{RpgFeat}{Hyperattack}[
        requires={Mega-attack},
    ]

        You can unleash superhuman speed against your enemies. Once per day, make \RpgDice{3d10} additional attacks.
    \end{RpgFeat}
    \end{ExampleBlock}
\newpage
\section{RpgSpell}

	\RpgMacro*[RpgSpell!Theme!default]{RpgSpell}{\param{\paramO m \paramO}}{ The default configuration of the \envref{RpgSpell} environment, used to describe magical spells or spell-like abilities.}
		{
			\cmd{RpgSpellShowCard\param{true/false}} \% set the card mode

			\cmd{begin}\param{RpgSpell}[card-opts]\param{Spell Name}[key-values]

			~~<body text>

			\cmd{end}\param{RpgSpell}
		}
		{As with all FeatureForge environments, has the ability to switch between `text mode' and `card mode' depending on the value passed to \cmd{RpgSpellShowCard} \RpgPage[p]{Macro:Rpg[X]ShowCard}. 
				
			The \texttt{default} RpgSpell is a placeholder; it defines no keys, and defines no special formatting for RpgSpell; on the assumption that `spells' are so system dependent as to make an attempt to make a default would be pointless.
		}

\section{RpgStat}

	\RpgMacro*[RpgStat!Theme!default]{RpgStat}{\param{\paramO m \paramO}}{ The dnd configuration of the \envref{RpgStat} environment, used to describe monsters and enemies via a \textit{statblock}}
		{
			\cmd{RpgStatShowCard\param{true/false}} \% set the card mode

			\cmd{begin}\param{RpgStat}[card-opts]\param{Creature Name}[key-values]

			~~<body text>

			\cmd{end}\param{RpgStat}
		}
		{As with all FeatureForge environments, has the ability to switch between `text mode' and `card mode' depending on the value passed to \cmd{RpgStatShowCard} \RpgPage[p]{Macro:Rpg[X]ShowCard}. 
				
			The \texttt{default} RpgStat is a placeholder; it defines no keys, and defines no special formatting; on the assumption that `statblocks' are so system dependent as to make an attempt to make a default would be pointless.
		}
