\chapter{Cards}

    The RpgCard environment is designed to allow a writer to create a small, playing-card sized unit which is useful for handing out to players for game elements such as items, spells or abilities.

    We anticipate that users won't access RpgCard directly, but will instead access it through wrappers which utilise 
    
    % but instead a wrapper such as RpgSpell \RpgPage[p]{S:RpgSpell Environment} -- however these serve only to format the text within the card environment in a specific way -- the Card itself remains flexible.

    The main power of the RpgCard is the ability to automatically \textit{cardbreak}, splitting large internal elements across multiple cards. 

   
\labelsection{Basic RpgCard}
        \RpgMacro*{RpgCard}{\param{\paramO}}{Splits the contents across a number of playing-card sized units}{
            
                \cmd{begin}\param{RpgCard}[<key-value-opts>]

                ~~~~~~<contents>

                \cmd{end}\param{RpgCard}
            }{
                Creates a card environment with a height and width defined either by \texttt{<opts>}, or the global variables. Text is automatically broken if it exceeds this height, creating multiple cards to hold the text.

                The available options are:
                % \newcommand\kve[3]{\texttt{#1} & #2 & #3\\}
                \begin{RpgTable}{llX}
                    Key & Values & Effect
                    \\
                    \kve{width}{dimexpr}{The outer width of the RPG card}
                    \kve{hmargin}{dimexpr}{The horizontal margin between the card boundary and the inner text}
                    \kve{height}{dimexpr}{The outer height of the RPG card}
                    \kve{vmargin}{dimexpr}{The vertical margin between the card boundary and the inner text}
                    \kve{cardsep}{dimexpr}{The horizontal space inserted after a card is finished}
                    \kve{color}{color specifier}{The background color of the image. Replaces \textit{tcb/colback}}
                    \kve{opacity}{[0-1]}{The opacity of the background of the image. Replaces \textit{tcb/opacityback}}
                    \kve{under-img}{img/path}{An image to be used as the background of the card. Clipped to the card background so no overspill occurs.}
                    \kve{underlay}{img/path}{An alias for under-img, which replaces \textit{tcb/underlay} to prevent premature tikz expansion.}
                \end{RpgTable}

                The formatting of RpgCard otherwise follows that of a \tcref{}, and all other tcb options can be passed in as normal. However, since the title-frame causes issues with the height calculations, RpgCards cannot use the tcb title interface: any attempt to set a title will fail. We provide two aliases, \texttt{color} and \texttt{opacity}, to make setting the usual values, \texttt{colback} and \texttt{opacityback}, less verbose and obtuse.

                Options passed to the environment take precedence to the global variables.

				Note that the RpgCard also temporarily redefines the \cmd{footnote} command (see \cmd{footnote} below).

                }
                
        \RpgMacro{cardbreak}{\param{}}{Analogous to \cmd{pagebreak}, forces a card to break at the specified location.}
        {
            <before-text>

            \cmd{cardbreak}
            
            <after-text>
        }
        {
            The command inserts an infinite, but hidden, vertical space penalty, causing the card to break at its location, as if it had been `filled up'.

            \cmd{cardbreak} is any empty function outside of the \texttt{RpgCard} environment, so it will have no effect on RpgCardSwitch environments.
        }
		\RpgMacro{footnote}{\param{m}}{A redefinition of the standard footnote to account for the quirks of an RpgCard}
		{
			<before-text>\cmd{footnote}\param{foottext}<after-text>
		}
		{
			Due to the multiple processing passes required to ensure that the RpgCard contents fit inside the relevant space, and since footnotes must fit within that space, we had a great deal of difficulty getting footnotes to work as expected (despite them nominally functioning in a tcolorbox).

			The compromise is that \textbf{only a single footnote can appear in a card}, and the standard \cmd{footnote} command is redefined for the duration of the environment. If multiple footnotes are present, only the final one (in order of execution) will appear. The footnote appears at the bottom of the final card, separated from the main text by a \cmd{hrule}. 

			An RpgCardF-fotnote does not increment the global footnote counter.
		}
        \RpgMacro{RpgSetCard}{\param{m}}{Sets the card parameters as global values for all subsequent cards}
        {
            \cmd{RpgSetCard}\param{<key/values>}
        }
        {
            The \texttt{<opts>} can contain any of the valid inputs to the \texttt{RpgCard} environment: this acts to set them as global default values, but any locally set values will override them.
        }

        \RpgMacro{RpgResetCard}{\param{}}{Resets changes made to card parameters back to the defalt value.}{}{}
        
    \colorlet{ForestGreen}{green!70!black}
    \begin{figure}
 \begin{ExampleBlock}[RpgCard,footnote]{Simple RpgCard}

 \begin{tikzpicture}%wrap in tikz so we can draw dimensions

  \node[anchor=south west,inner sep=0pt,outer sep=0pt] at (0,0) 
  {
  \begin{RpgCard}[hmargin=0.5cm,vmargin=0.5cm]

    \subsubsection{Text goes here}
      It fits nicely into the card, wrapping over lines\footnote{We can also have a single footnote}.
  \end{RpgCard}\ignorespaces
  };
  %% Draw the dimenions on top
  %height and vmargin
  \draw[red,<->] (0.7,0)-- node [fill=white, right, rotate=90, anchor=north]{height}++(0,8.8cm);
  \draw[red,<->] (1.4,0)-- node [right]{vmargin}++(0,0.5cm);
  \draw[red,<->] (1.4,8.3)-- node [right]{vmargin}++(0,0.5cm);
  %width and hmargin
  \draw[ForestGreen,<->] (0.5,2)-- node[below]{width}++(4.35,0);
  \draw[ForestGreen,<->] (0,2.4)-- node[below, rotate=90, anchor=east]{hmargin}++(0.5,0);
  \draw[ForestGreen,<->] (4.85,2.4)-- node[below, rotate=90, anchor=east]{hmargin}++(0.5,0);
\end{tikzpicture}
 \end{ExampleBlock}
 \caption{An example of a simple RpgCard with the dimensions overlayed}
\end{figure}

\begin{figure}
 \begin{ExampleBlock}[RpgCard,footnote]{Large RpgCard}
 \begin{RpgCard}
 {And if we add lots of text\footnote{Normally RpgCards stack side-by-side before breaking over lines -- here they've not got enough room, so it's vertically stacked.}}
                 
 \blindtext
 \end{RpgCard}
 \end{ExampleBlock}
 \caption{An example of an RpgCard with the automatic cardbreaking.}
\end{figure}
