\chapter{Switchable Environments}\label{S:SwitchEnv}

	Often it is convenient to be able to toggle between two different environments depending on an external flag. In the context of an RPG this might be for a number of reasons: having a player version and a GM version, or having a screen-readable version versus a printable one. 

	Whilst it is obviously possible to build an environment which performs the switching for you, we provide a generic interface for switching between \textit{similar environments}.

	\begin{RpgSidebar}{'Similar' Environments}
		It is important to note that this system only works for switching between environments which are `similar', insofar as they permit the same number and order of arguments, and interpret their contents similarly.

		An itemize and an enumerate are `similar': an itemize and a figure are not.
	\end{RpgSidebar}

	\labelsection{RpgSwitchEnv}
	\RpgMacro*{RpgSwitchEnv}{\param{m o m o m}}
		{
			Acts as one of two similar environments based on the value of an input key.
		}{
			\cmd{begin}\param{RpgSwitchEnv}\param{<key>}[opt-1]\param{env-1}[opt-2]\param{env-2}

			~~<contents>

			\cmd{end}\param{RpgSwitchEnv}
		}{
			The \texttt{key} is an input token (a string) which should be in the global \textit{switch-registry} (see below). If the value associated with the key in the registry is false, then the environment acts as \texttt{env-1} (with optional arguments \texttt{[opt-1]}), whilst if the switch is true, the environment acts as \texttt{env-2[opt-2]}.

			Due to the way that token expansion works, it is possible to pass \textit{additional} arguments to this environment:
			\begin{Code}
				\cmd{begin}\param{RpgSwitchEnv}\param{<key>}[opt-1]\param{env-1}[opt-2]\param{env-2}\param{arg1}\param{arg2}
			\end{Code}
			Formally speaking, \texttt{arg1} and \texttt{arg2} are a part of the body of the environment; however if both env-1 and env-2 are expecting two arguments, then the token expansion captures them. It is also possible to use a shared optional argument, instead of the unique arguments:
			\begin{Code}
				\cmd{begin}\param{RpgSwitchEnv}\param{<key>}\param{env-1}\param{env-2}[shared-opt]
			\end{Code}
			If however, the environments are not similar, and take different numbers of arguments then any excess arguments are inserted into the body of the environment, which can cause unexpected behaviour.

			An error is thrown if \texttt{key} does not exist in the global registry.
		}
		\RpgMacro{RpgSetSwitch}{\param{m m}}
			{Change the value of a \textit{switch}, and therefore the behaviour of the associated RpgSwitchEnv}{\cmd{RpgSetSwitch}\param{<key>}\param{<value>}}
			{
				Sets the value of the \texttt{key} in the \textit{switch-registry} to \texttt{{value}}, which must be a 'bool-ish' text string\footnote{That is, either \param{true,True,1} or \param{false,False,0}}. If the entry does not exist in the registry, it is created.

				After the key is set, all subsequent \texttt{RpgSwitchEnv} calls which use that key will have their behaviour altered to match the new key.
			}
	\labelsection{RpgSecret}



	\labelsection{RpgItem}

	\labelsection{RpgAbility}