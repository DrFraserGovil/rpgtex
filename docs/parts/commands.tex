


\chapter{Commands \& Macros}
\def\COSS{\hyperref[S:ThemeCommands]{\textcolor{blue!40!black}{{CoSS Function}}}}


\section{Title \& Part Pages}
    \RpgMacro{cover}{\param{m},\cmd{@cover}}
		{
			Saves an image path to the variable \texttt{\bs{}@cover}, automatically used by \texttt{\bs{}maketitle} as the background image.
		}{
			\bs{}cover\{path/to/cover\} 
		}{
			If \cmdref{RpgUseCoverPage} has been set to true, then the image at this path will be used as a full-page image in the background of the page created by \texttt{maketitle}.

			The default value is empty (\texttt{\bs{}cover\param{}}), which draws no image.
		}
	\RpgMacro{maketitle}{\param{}}
		{	
			When called, creates theme-defined title pages using a custom format.
		}{
			\bs{}title\{A title\}

			\bs{}subtitle\{The subtitle\} \%optional

			\bs{}cover\{path/to/cover\} \%optional

			\bs{}author\{Dr. W. Riter\} \%optional

			~

			\bs{}begin\{document\}

			~~\bs{}maketitle{}
			
			~~(\ldots)
		}{
			\cmd{maketitle} has been completely defined, so commands such as \texttt{titlepage} won't work as expected. Instead, the appearance of the title page (or title header) are set by the theme, or user calls to either \cmdref{RpgSetCover} or \cmdref{RpgSetSimpleTitle}. Which of the two title `modes' is active is controlled by \cmd{RpgUseCoverPage}.

			If \texttt{RpgUseCoverPage} has been set to true, then the image stored in \texttt{\bs{}{@}cover} (if there is one) is automatically used as a full-page background image. This is independent of the drawing commands, and occurs before that function is called -- all subsequent drawing occurs over the top of the cover image.
		}
	
	\RpgMacro{part}{\cmd{part*},\param{{o m}}}
		{
			Defines a wrapper around the standard \texttt{part} command that allows for tikz-based custom page formatting
		}
		{
			\bs{}part(*)[<image>]\param{<part-name>}
		}
		{
			There are three distinct behaviours that can be exhibited, depending on the presence or absence of the \texttt{*}, and the presence and value of \texttt{<image>}.

			\begin{RpgTable}{XX}
				Command & Behaviour
				\\
				\parbox[t]{6cm}{\cmd{part*\param{partname}} \\ \cmd{part*[<any text>]\param{partname}} \\ \cmd{part[none]\param{partname}}} & Uses original \cmd{part} command defined by underlying class. 
				\\
				\cmd{part{partname}} & Calls the \cmd{RpgSetPartPage} control sequence on a blank background.
				\\
				\parbox[t]{6cm}{\cmd{part[path/to/image]\param{partname}}|} & Places the corresponding image as a full-page background, and then calls the \cmd{RpgSetPartPage} drawing command.
			\end{RpgTable}
			The `drawing command' is a control sequence set by \cmdref{RpgSetPartPage}, which defines a series of tikz functions to place the part title according to the theme specifications. 
		}
	\RpgMacro{RpgUseCoverPage}{\param{{m}}}
		{
			If true, \cmd{maketitle} creates a title page to populate, else the title is rendered as an article-like heading.
		}{
			\cmd{RpgUseCoverPage}\param{true/false}
		}
		{
			When true, \cmd{maketitle} attempts to use \cmd{@cover} and then calls \cmdref{RpgSetCover}. If false, it calls \cmdref{RpgSetSimpleTitle}.
		}
	\RpgMacro{subtitle}{\param{{m}},\cmd{@subtitle}}
		{
			Saves a string to the variable \cmd{@subtitle}. Themes may use this when defining their \cmd{RpgSetCover} and \cmd{RpgSimpleTitle}.
			\index{{"@}subtitle}
		}
		{
			\bs{}subtitle\param{<string>}
		}
		{
			This command has no effect on its own (unlike \cmd{cover} which is automatically included in the background).

			The default value is empty (\cmd{subtitle\param{}}).
		}

\section{Fonts \& Decorative Text}
    % In addition to the fundamental typeface alterations \rpgtex{} includes a number of commands to turn text into decorative elements.
	\RpgMacro{emph}{\param{m},\cmd{key},\param{m}}{Uses the \texttt{RpgFontEmphasis} font to emphasise text.\index{key}}
		{
			\cmd{emph}\param{text}

			\cmd{key}\param{text}
		}
		{
			The \cmd{emph} command is usually a `context aware' emphasis command: equal to \cmd{textit} normally, \cmd{textbf} when the surrounding text is italics etc. 

			However, for RPGs, it is convenient to be able to identify \emph{keywords} in a consistent fashion. The \cmd{emph} command has therefore been redefined to use the \texttt{RpgFontEmphasis} font which can be configured to give a desired `keyword formatting'.
			
			The command \cmd{key} has also been provided, as a direct alias for \cmd{emph}.
		}
	\RpgMacro{RpgContour}{\param{\paramO{} m}}
		{	
			Renders text with a \RpgContour[inner=red,outer=black]{contour effect}. The color and style are set through key/value pairs.
		}
		{
			\bs{}RpgContour[inner=<color>,outer=<color>,style=<code>]\param{<text>}
		}
		{
			The \texttt{style} command is applied to the text, whilst the optional \texttt{inner} and \texttt{outer} commands set the base text colour and the external contour color respectively. If the colors are not set, the default values are the \texttt{contourinnercolor} and \texttt{contouroutercolor} values defined by the theme \RpgPage[p]{S:Colors}.
			
			The contour does not automatically linebreak, but can be controlled manually with a \texttt{\newline} command (not \texttt{\\} or \texttt{\textbackslash{}par})
			
			\begin{RpgTable}{lX}
				Example & Output \\
				\tabverbExample{\RpgContour[inner=red,outer=black]{example}}
				\tabverbExample{\RpgContour[style=\Huge\it]{example}}
				\tabverbExample{\RpgContour[]{multi\newline line\newline example}}
			\end{RpgTable}					
			~\subsection{Quirks}

			Due to the tokenisation required for the line-splitting and space-preservation, the text inside the contour can exhibit some quirks if stylisation is applied within the \texttt{<text>} argument. 

			Unbraced commands (such as \texttt{\it} or \texttt{\footnotesize}) will only apply to the first word in the text. Braced commands \textit{can} work, but will cause a compilation error if a \texttt{\newline} is included. 

			
			\begin{RpgTable}{lX}
				\footnotesize\tabverbExample{\RpgContour[]{\Huge\it only first word changes}}
				\footnotesize\tabverbExample{\RpgContour[]{\textit{all words change}}}
				\footnotesize\texttt{\bs{}RpgContour[]{\bs{}textit\param{all word \bs{}newline change}}} & (fails to compile)
			\end{RpgTable}
			For robustness, we therefore recommend that all stylisation be applied through the \texttt{style} command, which is applied to each tokenised element, and therefore guaranteed to work as expected.
		}
	\RpgMacro{RpgDropCap}{\param{\paramO{} m m}}
		{	
			Creates a decorative `drop cap' letter to begin a new chapter with, and modifies the following text.
		}
		{
				\bs{}RpgDropCap[<lettrine-args>]{<letter>}{<text>}
		}{
			This command uses \forcelink{https://texdoc.org/serve/lettrine/0}{the lettrine package} and the \forcelink{https://ctan.org/pkg/magaz?lang=en}{magaz} package to create an easy-to-use environment in which the first letter is enlarged (and stylised in the \texttt{RpgFontDropCap} font). The second argument formats \textit{up to the first line} of text in the \texttt{RpgFontDropCapInternal} font (usually a simple \texttt{scshape} command).

			This command can be a little fragile -- lettrine does not usually play well with the `FirstLine' command provided by magaz -- and we've used a few workarounds to allow both linebreaking, and the formatting of only the first line of text. There may need to be a small amount of manual calibration, but it is better than the default.					

		}
 \begin{ExampleBlock}[RpgDropCap]{RpgDropCap}
 \raggedright \RpgDropCap{T}{he example: this text runs over the first line, and then revert back to the normal font. It almost works! However, because it's wrapped in a text box, it goes slightly over the edges.}
 \end{ExampleBlock}
        

    

\section{Theme Commands}

   
    
    \RpgMacro{RpgSetTheme}{\param{{m}}}
        {
            Activates a chosen theme.
        }
        {\bs{}RpgSetTheme\param{<theme-name>}}
        {
            Searches for the file \texttt{<theme-path>/<theme-name>/<theme-name>.cfg}, and inputs it. If this is a properly configured theme file, then it activates the chosen theme given the current global parameters. If the file does not exist, throws an error.

            If   \texttt{\textbackslash{}l\_\_rpg\_layout\_bool} is True, the command automatically inserts \texttt{\bs{}clearpage}, as required to ensure the old headers are not overwritten by the new theme.

            \texttt{<theme-path>} is modified via \cmdref{RpgSetThemePath}.
        }

    
    \RpgMacro{RpgSetThemePath}{\param{{m}}}
        {
            Changes the value of the theme path searched for by \cmd{RpgSetTheme}
        }
        {
            \bs{}RpgSetThemePath\param{<path-name>}
        }
        {
            Updates an internal variable to be equal to the input value; does not check if the theme path is valid or not. Useful if you wish to create a new theme outside of the \texttt{rpgtex} file structure.
        }
        
\section{Utility Commands}
 	 \RpgMacro{RpgDice}{\param{{m}}}
        {
            Evaluates expressions of the form $n\mathrm{d}x \pm m$, and outputs using a theme-dependent layout.
        }
        {
            \bs{}RpgDice{<dice-expression>}
        }{
            Uses regular expressions to extract and simplify the \texttt{dice-expression}, which must follow the following format:
            \begin{RpgSidebar}{Dice format}
                \begin{multicols}{2}
                \begin{enumerate}
                    \item It must contain either `d' or `D' (the `dice symbol')
                    \item The dice symbol must be immediately followed by a single number (the `dice size')
                    \item The dice symbol may optionally be prefixed by a single number (the `dice count')
                    \item The first (non-whitespace) character must be either the dice count (if present) or the dice symbol
                    \item The dice size must be followed by either a `+', '-', or the end of the expression.
                    \item After this, any number of standard numeric expressions may follow. This expression will be evaluated into a single `modifier'.
                \end{enumerate} 
                \end{multicols}
            \end{RpgSidebar}
            The dice ignores any whitespace before the beginning of the expression, and arbitrary whitespace within the `modifier' part of the exprssion.  
            \begin{RpgTable}{XX}
                Example & Output \\
                \tabverbExample{\RpgDice{  1d6-2}}
                \tabverbExample{\RpgDice{2D6 + 3*2^2}}
                \tabverbExample{\RpgDice{1d16}}
                \tabverbExample{\RpgDice{d8-3}}
                \texttt{\bs{}RpgDice\param{2*1d6}}, \texttt{\bs{}RpgDice\param{1 d6}}, \texttt{\bs{}RpgDice\param{3d 6 +3}} & (Fails to compile)
            \end{RpgTable}
        
            \cmd{RpgDice} is neat, but not necessarily impressive by itself. The true power of the expression is that it calls the control sequence set by \cmdref{RpgDiceSetFormat} to perform the output formatting (after performing the regular expression parsing), allowing designers to customise their dice formatting.
        }

    \RpgMacro{RpgFakeChapter}{\param{m}}{Mimics creating a new chapter with \cmd{chapter} (including adding in to the table of contents) without inserting a `chapter heading'}
    {
        \cmd{RpgFakeChapter}\param{fake-name}
    }
    {
        The value of \texttt{fake-name} is passed to the table of contents as a `true' chapter, and an update to \cmd{chaptermark} updates the Section Names \RpgPage[p]{S:SectionNames}, and thus the footer appearance.
    }

    \RpgMacro{RpgOrdinal}{\param{{o m}}}
        {
            Converts a numeric value to the corresponding ordinal.
        }
        {
            \bs{}RpgOrdinal[<command>]\param{<count>}
        }
        {
            The command outputs the \texttt{count} followed by the english abbreviations for the corresponding ordinal. The optional \texttt{command} argument is inserted between the numeral and the suffix, allowing for the customisation of appearances.
            \begin{RpgTable}{XX}
                Example & Output \\
                \tabverbExample{\RpgOrdinal{1}}
                \tabverbExample{\RpgOrdinal{2}}
                \tabverbExample{\RpgOrdinal{13}}
                \tabverbExample{\RpgOrdinal[\textsuperscript]{7}}
                \tabverbExample{\RpgOrdinal[\textbf]{133}}
                \tabverbExample{\RpgOrdinal[<arbitrary text>]{133}}
            \end{RpgTable}
            {\it Note that due to a lack of brace-capturing, it is not possible to chain multiple commands.}.
        }
    \RpgMacro{RpgPage}{\param{{O\param{t} m}}}
        {
            Outputs the current page reference for a label, with an option to enclose it in specific brackets or parentheses.
        }
        {
            \bs{}RpgPage[t/p/b/c]\param{<label-reference>}
        }
        {
            The optional arguments wrapping of the main reference. The options are:
            \begin{description}
                \item[t (default)] No wrapping
                \item[p] (parentheses)
                \item[b] [square brackets]
                \item[c] \{curly braces\}
            \end{description}
            An invalid input resolves to \texttt{?page~\bs{}pageref\param{<ref>}?}.\label{example:current page}
            
            \begin{RpgTable}{XX}
                Example & Output \\
                \tabverbExample{\RpgPage{example:current page}}
                \tabverbExample{\RpgPage[p]{example:current page}}
                \tabverbExample{\RpgPage[b]{example:current page}}
                \tabverbExample{\RpgPage[c]{example:current page}}
                \tabverbExample{\RpgPage[(error)]{example:current page}}
            \end{RpgTable}
        }
    \RpgMacro{RpgPlural}{\param{{o m m}}}
        {
            Generates grammatically correct plural forms of a word based on a given count.
        }
        {
            \bs{}RpgPlural[<custom-plural>]\param{count}\param{<text>}
        }
        {
            The command outputs the count followed by the value of \texttt{<text>}. For a count of 1, the command then finishes. For any other count, it appends an ``s'', pluralizing the text.

            The optional argument \texttt{[<custom-plural>]} overrides the default logic, allowing for irregular plurals.


            \begin{RpgTable}{XX}
                Example & Output \\
                \tabverbExample{\RpgPlural{1}{hat}}
                \tabverbExample{\RpgPlural{2}{hat}}
                \tabverbExample{\RpgPlural[octopodes]{1}{octopus}}
                \tabverbExample{\RpgPlural[octopodes]{359}{octopus}}
            \end{RpgTable}
        }

        

