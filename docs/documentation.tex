
\documentclass[size=10pt,theme=default]{rpgbook}

\usepackage{doc}
\usepackage{blindtext}
\usepackage{imakeidx}
\makeindex
\def\rpgtex{\texttt{rpgtex}}

\cover{../example/img/typewriter}

\title{The rpgtex Package}
\subtitle{A package for generating beautiful RPG documents}
\author{Jack Fraser-Govil}

\newcommand{\lindex}[1]{%
  \lowercase{\def\temp{#1}}%
  \expandafter\index\expandafter{\temp}%
}

\NewDocumentCommand{\cmdidx}{ o m }{
	\foreach \m in {#2}
	{
		% \lindex{\texttt{\textbackslash{}\lowercase{\m}}@{\lowercase{\m}}}
		\IfNoValueTF{#1}
		{
			 \lowercase{\def\temp{\m}}
			\expandafter\index\expandafter{\temp@\texttt{\textbackslash\m}}
		 }
		 {
			% \index{\m}
			 \lowercase{\def\temp{\m}}
			\expandafter\index\expandafter{#1!\temp@\texttt{\textbackslash\m}}
		 }
	}
}

\begin{document}
	\frontmatter
	\maketitle{}

	%%% fancy intro/abstract
		\onecolumn{}
		~\vfill{}
		\newcommand\forcelink[2]
		{
			\href{#1}{{\color{blue!70!black}#2}}
		}
		\begin{center}
		\parbox[c]{0.8\linewidth}{
			\large
			\justifying{}

			\RpgDropCap{W}{elcome to the \rpgtex{} package}. This \LaTeX{} package is designed to allow users to flexibly typeset documents associated with Role Playing Games such as \textit{Dungeons \& Dragons} -- and many more besides. This packages defines a central engine: \texttt{rpgcore} which define a number of useful functions and classes, and a flexible set of \texttt{themes} which control how those commands are rendered in the final document.

			\vspace{2cm}
			\subsection*{Attribution \& License}

			This package would not have been possible without the team who developed \forcelink{https://github.com/rpgtex/DND-5e-LaTeX-Template/tree/dev}{its predecessor, the `DND 5e LateX Template'}. That code was released under an MIT license, the text of which can be found in the LICENSE file. \texttt{rpgtex} is released under the same license.
			}
		\end{center}

		\vfill{}
		\twocolumn{}

	\tableofcontents
	\mainmatter{}


		
	\part{rpgtex Core}
		\chapter{Installation \& Usage}

	\section{Getting \rpgtex}

		There are a number of different ways to acquire \rpgtex{}. Once you have installed it, it is vital to ensure that it is \href{\ref{S:Configuration}}{properly configured} (see below).


		\subsection{texmf Installation}

			The simplest way to use \rpgtex{} is to install it on the \texttt{texmf} path, where the compiler can automatically find it:

			\begin{lstlisting}
git clone https://github.com/DrFraserGovil/rpgtex.git "$(kpsewhich -var-value TEXMFHOME)/tex/latex/rpgtex"
			\end{lstlisting}

			This will clone the repository into your\LaTeX{} path.

		\subsection{Indirect Installation}

			If you want to tinker with \rpgtex{} -- such as by creating a new theme -- it is helpful to have it in a more accessible location. Clone the repository into a location of your choice:

			\begin{lstlisting}
git clone https://github.com/DrFraserGovil/rpgtex.git ~/your/rpgtex/directory
			\end{lstlisting}

			You then have two options to make the package visible to the compiler:

			\subsubsection{Use TEXINPUTS}

			Setting the environment variable \texttt{TEXINPUTS} allows the compiler access:
				\begin{lstlisting}
	TEXINPUTS=~/your/rpgtex/directory/::
				\end{lstlisting}
				(Or similar commands, depending on your shell -- in \texttt{fish} you would call \texttt{set TEXINPUTS dir}).

			\subsubsection{Use Symlinks}

			You can symlink the install location to the texmf directory, allowing the compiler to act as if you had performed the texmf installation:

			\begin{lstlisting}
				ln -sf ~/your/rpgtex/directory "$(kpsewhich -var-value TEXMFHOME)/tex/latex/rpgtex"
			\end{lstlisting}

		\subsection{Overleaf (Not recommended!)}

			We do not recommend using Overleaf since the free-tier subscription has reduced compilation times drastically, making compiling documents using complex packages such as this one extremely difficult. Nevertheless:

			\begin{enumerate}
				\item  Download this GitHub repository as a ZIP archive using the Clone or download link above.
    			\item On Overleaf, click the New Project button and select Upload Project. Upload the ZIP archive you downloaded from this repository.
				\item Manually create the file \texttt{rpg-config.cfg} with the contents ``\texttt{\edef\RpgPackagePath{../}}''. This replaces the configuration step described below.
			\end{enumerate}


	\section{Configuring \rpgtex{}}\label{S:Configuration}

		Wherever one installs \rpgtex{} from, it is vital that it is properly configured. From within the \rpgtex{}-root directory, call:

		\begin{lstlisting}
			./configure
		\end{lstlisting}
		Or -- if one is (reasonably!) wary about running arbitrary executables -- manually create the relevant file:
		\begin{lstlisting}
			cd <rpgtex root directory>
			cmd="\edef\RpgPackagePath{$(pwd)}"
			echo $cmd >> core/rpg-config.cfg
		\end{lstlisting}

		\begin{RpgTip}{Why is configuration necessary?}
			\TeX{} is generally set up so that when a file calls \texttt{include} or \texttt{input} it is possible to use filepaths relative to the package itself. \texttt{rpg.sty} can call \texttt{\cmd{input}{core/font.sty}} and it will know to first check for the file relative to rpg.sty; even if the package resides within the texmf path and the user has no idea where \texttt{rpgroot/rpg.sty}, or \texttt{rpgroot/core/font.sty}, are.

			An annoying exception to this is fonts and typefaces. \texttt{xelatex} searches for fonts based on \textit{filepaths relative to the current working directory} -- or from those installed in as system fonts.

			Since \texttt{rpgtex} includes several (license free) typefaces as part of the provided themes, this poses a problem. We must either require that:
			\begin{enumerate}
				\item \rpgtex{} documents can only be prepared in restricted locations relative to the install location of \rpgtex{}.
				\item Users must identify and specify the \rpgtex{} root path when preparing a document
				\item Users must install the provided fonts to the system path
				\item \rpgtex{} must be configured to know `where it is', and so provide an absolute filepath to the internal fonts.
			\end{enumerate}
			The Configuration step is the most portable and easiest-to-use of these options.
		\end{RpgTip}

		Without a \texttt{core/rpg-config.cfg} file, any document which includes \rpgtex{} will fail to compile.
	\newpage

	\section{Package \& Class Usage}

		\rpgtex{} can be used either as a standalone package, or as part of a number of classes

		\subsection{Standalone Package}
			The standalone package can be used directly by including the \rpgtex{} package:
			\begin{lstlisting}
				\documentclass{arbitrary-class}

				\usepackage[options]{rpgtex}

				\begin{document}
				....
			\end{lstlisting}

			This will load only the core commands into the document, and (unless called explicitly) no themes will be imported. Using the package in this way does not activate any of the commands which change the overall geometry, background or headers of the document.

		\subsection{Classes}

			\rpgtex{} can also be loaded through a number of classes which drastically alter the appearance of the document, defining new geometries backgrounds and adding headers.

			The provided classes are:
			\begin{enumerate}
				\item \texttt{rpgbook} (\RpgPage{S:bookClass}). Based on the standard book class, this is designed for larger RPG documents.
				\item \texttt{rpghandout} (\RpgPage{S:handoutClass}). Based on the article class, this is designed for shorter documents
				\item \texttt{rpgcard} (\RpgPage{S:cardClass}). A small-document class designed for creating modular `handout' cards for items, spells or abilities.
			\end{enumerate}
	

		
\BeginMacroFormat{}
		
	\section{Compiling rpgtex Documents}\label{C:Compiler}


		\rpgtex{} uses the \texttt{fontspec} package to allow custom fonts, and therefore requires compiling with \texttt{xelatex} or \texttt{luatex}:

		\begin{lstlisting}
			xelatex main.tex		#works
			luatex main.tex			#works
			pdflatex main.tex #fails
		\end{lstlisting}
		So long as \rpgtex{} is on the user's latex path, and the package properly configured \RpgPage[p]{S:Configuration} no further compilation steps are required. However, for ease of use, we provide the \texttt{rpglatex} compiler as part of the \rpgtex{} distribution.

		\subsection{The \texttt{rpglatex} compiler}

		\rpgtex{} is shipped with a special compiler, \texttt{rpglatex}. This is simply a python3 script which acts as a wrapper around either xelatex or luatex, but includes several quality-of-life changes to the interface to make it easier to use with \texttt{rpgtex}.


		\RpgMacro*{rpglatex}{\param{m}}{Compiles latex documents using either xelatex or luatex}
		{
			> rpglatex [options] <file>
		}
		{
			\texttt{rpglatex} has the following features:

			\newcommand\feature[3]
			{
				\textbf{#1} & #2 & \texttt{#3} \\
			}
			\begin{RpgTable}[width=\linewidth]{lXl}
				Feature & Description & Options \\
				\feature{Compiler Selection}{The \texttt{xelatex} compiler is selected by default, but the \texttt{-l, --luatex} flags set it to use luatex instead.}{-l, --luatex}
				\feature{Build Directory}{Compilation files (.aux, .log etc.) are stored in a build directory. The default is \texttt{.build} in the calling location, but can be changed with the \texttt{-b} flag}{-b <build dir>}
				\feature{Output Naming}{The name of the output file can be changed from the default (equal to the input tex name)}{-o <output name>}
				\feature{Multi-pass Compiling}{By default, the compiler runs twice in a row to enable references and \texttt{tikz[remember]} commands to function. A full three-compilation suite (necessary for very complex or reference-heavy documents) can be activated with the \texttt{-f, --full} flag}{-f, --full}
				\feature{Volume Control}{latex is notoriously noisy, producing copius output. By default, this is suppressed and only a summary is printed. The summary can be removed (rendering it completely silent) with the \texttt{-q} command, or the original output recovered in verbose mode; \texttt{-v}.
				
				These outputs are always overriden if a compilation error occurs, in which case the full trace is output to the console.
				}{-q, -v}
				\feature{Auto-bibtex}{If the \texttt{-r} or \texttt{--ref} flag is set, \texttt{bibtex} is automatically called in between the multi-compilation steps}{-r, --ref}
				\feature{Auto-visualisation}{If the \texttt{--show 1} option is set (which it is by default), the compiler will call \texttt{xdg-open <output-file>} upon completion of the compilation; automatically opening or context-switching to the document. This can be turned off by calling \texttt{--show 0}}{--show}
				\feature{Print Mode}{A special interface for \rpgtex{}, this uses the \texttt{\RpgCMD} interface \RpgPage[p]{S:CMD} to inject code into the latex document, setting the \texttt{bg=print} mode and suppressing the background output.}{-p, --print}
			\end{RpgTable}
		}

	

		


\chapter{Commands \& Macros}
\def\backendCommand{\hyperref[S:ThemeCommands]{\textcolor{blue!40!black}{{Backend Command}}}}
\def\placeholderCommand{\hyperref[S:ThemeCommands]{\textcolor{blue!40!black}{{Placeholder Command}}}}
\begin{RpgSidebar}{Theme Commands}\label{S:ThemeCommands}

	Several commands in this documentation are described as \textbf{Theme Commands}. These are commands that the user is \textit{not expected to call}, but which are executed by the internal engine in the process of rendering the page, or as a result of other commands that the user has called. 

	\begin{center}
	\large \textbf{A user who wishes to simply write documents using an unmodified \rpgtex{} need only concern themselves with the User-Facing Commands}. 
	\end{center}
	On the other hand, these Theme Commands have been designed to provide a convenient interface for creating custom Themes -- and so their documentation allows for designers to create powerful and flexible themes from within \rpgtex{}. 
	
	Theme Commands can be split into two groups:
	
	\begin{enumerate}
		\item \textbf{Backend Commands} These are commands which are executed within a theme (or a class) to modify internal values, such as fonts and colors. A designer interacts with these commands by calling them.
		\item \textbf{Placeholder Commands} These are virtual commands which are designed to be overwritten with completely custom code, which is executed when the core engine runs the command. A user interacts with these commands by redefining them  (usually with \texttt{RenewDocumentCommand}).
	\end{enumerate}

	A `theme' is therefore a collection of Backend Commands (to configure the `core engine') and redefinitions of Placeholder Commands to provide their own unique functionality.
	
\end{RpgSidebar}

\section{Title Pages}
	\subsection{User-Facing Commands}
		\RpgMacro{maketitle}{\param{}}
			{	
				When called, creates theme-defined title pages using a custom format.
			}{
				\bs{}title\{A title\}

				\bs{}subtitle\{The subtitle\} \%optional

				\bs{}cover\{path/to/cover\} \%optional

				\bs{}author\{Dr. W. Riter\} \%optional

				~

				\bs{}begin\{document\}

				~~\bs{}maketitle{}
				
				~~(\ldots)
			}{
				Calls either \texttt{\bs{}RpgDrawCover} or \texttt{\bs{}RpgSimpleTitle} depending on the value passed to \texttt{\bs{}RpgUseCoverPage}.

				If \texttt{RpgUseCoverPage} has been set to true (usually by a class such as \texttt{rpgbook.cls}), then the image stored in \texttt{\bs{}{@}cover} (if there is one) is automatically used as a full-page background image. This is independent of the theme definition of \texttt{RpgDrawCover}, and occurs before that function is called -- all subsequent drawing occurs over the top of the cover image.
			}
		\RpgMacro{cover}{\param{m},\cmd{@cover}}
			{
				Saves an image path to the variable \texttt{\bs{}@cover}, automatically used by \texttt{\bs{}maketitle} as the background image.
			}{
				\bs{}cover\{path/to/cover\} 
			}{
				If \cmdref{RpgUseCoverPage} has been set to true, then the image at this path will be used as a full-page image in the background of the page created by \texttt{maketitle}.

				The default value is empty (\texttt{\bs{}cover\param{}}), which draws no image.
			}


		\RpgMacro{subtitle}{\param{{m}},\cmd{@subtitle}}
			{
				Saves a string to the variable \cmd{@subtitle}. Themes may use this when defining their \cmd{RpgDrawCover} and \cmd{RpgSimpleTitle}.
				\cmdidx{{"@}subtitle}
			}
			{
				\bs{}subtitle\param{<string>}
			}
			{
				This command has no effect on its own (unlike \cmd{cover} which is automatically included in the background).

				The default value is empty (\cmd{subtitle\param{}}).
			}
	\subsection{Theme Commands}
		\RpgMacro{RpgUseCoverPage}{\param{{m}}}
			{
				If true, \cmd{maketitle} creates a title page to populate, else the title is rendered as a heading.
			}{
				\cmd{RpgUseCoverPage}\param{true/false}
			}
			{
				This is a \backendCommand{}. When true, \cmd{maketitle} attempts to use \cmd{@cover} and then calls \cmdref{RpgDrawCover}. If false, it calls \cmdref{RpgSimpleTitle}.
			}
		\RpgMacro{RpgDrawCover}{\param{{}}}
			{
				Executes over the top of the \cmd{@cover}-image to render a front cover.
			}{}
			{
				This is a \placeholderCommand{}, used by themes to customise the appearance of the title page which appears in \texttt{rpgbook} class. The default value renders a single node at the centre of the page containing \cmd{@title}, \cmd{@subtitle}, \cmd{@author} and \cmd{@date} variables in the centre. More advanced themes (such as dnd or scifi) add decorative embellishments and place the text at custom locations.

				This command is executed by \cmd{maketitle} if \cmd{RpgUseCoverPage{true}} has been set by the theme, class or directly by the user. 
				The command is called from within an existing tikz environment with the \texttt{remember,overlay} options active, allowing for page coordinates (i.e. current page.north) to be used.

				If a \cmd{@cover} has been defined, this command is executed after the image is placed, drawing on top of it.
			}
		\RpgMacro{RpgSimpleTitle}{\param{{}}}
			{
				Renders a `header' title - a simple text-only title at the top of the page.
			}{}
			{
				This is a \placeholderCommand{}, used by themes to customise the appearance of the title header which appears in \texttt{rpghandout} class. The default value places the title, subtitle and author at the top of the page. More advanced themes (such as dnd or scifi) add decorative embellishments and place the text at custom locations.

				The Simple Title is configured so that, in a twocolumn document, it occupies the full page width; calling \texttt{centering} with the simple title therefore centers the text above both columns. 
			}

\section{Part Pages}
		\RpgMacro{part}{\cmd{part*},\param{{o m}}}
			{
				Defines a wrapper around the standard \texttt{part} command that allows for tikz-based custom page formatting
			}
			{
				\bs{}part(*)[<image>]\param{<part-name>}
			}
			{
				There are three distinct behaviours that can be exhibited, depending on the presence or absence of the \texttt{*}, and the presence and value of \texttt{<image>}.

				\begin{RpgTable}{XX}
					Command & Behaviour
					\\
					\parbox[t]{6cm}{\cmd{part*\param{partname}} \\ \cmd{part*[<any text>]\param{partname}} \\ \cmd{part[none]\param{partname}}} & Uses original \cmd{part} command defined by underlying class. 
					\\
					\cmd{part{partname}} & Calls \cmd{RpgDrawPartPage} on a blank background.
					\\
					\parbox[t]{6cm}{\cmd{part[path/to/image]\param{partname}}|} & Places the corresponding image as a full-page background, and then calls \cmd{RpgDrawPartPage}.
				\end{RpgTable}
				\cmdref{RpgDrawPartPage} is a Theme Function, which executes a series of tikz functions to place the part title according to the theme specifications. 
			}
		
		\RpgMacro{RpgDrawPartPage}{\param{{m}}}{Uses Tikz to draw a custom part page when activated by \cmdref{part}. 
			}
			{
				\bs{}RpgDrawPartPage\param{<part title>}
			}
			{
				This is a \placeholderCommand{}, allowing the designed to determine where to place the part name on the page, and what embellishments accompany it. The command is called from within an existing tikz environment with the \texttt{remember,overlay} options active, allowing for page coordinates (i.e. current page.north) to be used.

				The default \cmd{part} command allows a user to specify a background image for their part page -- it is not necessary to provide one within the drawing command.
			}
		


\section{Fonts \& Decorative Text}
	% In addition to the fundamental typeface alterations \rpgtex{} includes a number of commands to turn text into decorative elements.
		\RpgMacro{RpgContour}{\param{\param\paramO{} m}}
			{	
				Renders text with a \RpgContour[inner=red,outer=black]{contour effect}. The color and style are set through key/value pairs.
			}
			{
				\bs{}RpgContour[inner=<color>,outer=<color>,style=<code>]\param{<text>}
			}
			{
				The \texttt{style} command is applied to the text, whilst the optional \texttt{inner} and \texttt{outer} commands set the base text colour and the external contour color respectively. If the colors are not set, the default values are the \texttt{contourinnercolor} and \texttt{contouroutercolor} values defined by the theme \RpgPage[p]{S:Colors}.
				
				The contour does not automatically linebreak, but can be controlled manually with a \texttt{\newline} command (not \texttt{\\} or \texttt{\textbackslash{}par})
				
				\begin{RpgTable}{lX}
					Example & Output \\
					\tabverbExample{\RpgContour[inner=red,outer=black]{example}}
					\tabverbExample{\RpgContour[style=\Huge\it]{example}}
					\tabverbExample{\RpgContour[]{multi\newline line\newline example}}
				\end{RpgTable}					
				~\subsection{Quirks}

				Due to the tokenisation required for the line-splitting and space-preservation, the text inside the contour can exhibit some quirks if stylisation is applied within the \texttt{<text>} argument. 

				Unbraced commands (such as \texttt{\it} or \texttt{\footnotesize}) will only apply to the first word in the text. Braced commands \textit{can} work, but will cause a compilation error if a \texttt{\newline} is included. 

				
				\begin{RpgTable}{lX}
					\footnotesize\tabverbExample{\RpgContour[]{\Huge\it only first word changes}}
					\footnotesize\tabverbExample{\RpgContour[]{\textit{all words change}}}
					\footnotesize\texttt{\bs{}RpgContour[]{\bs{}textit\param{all word \bs{}newline change}}} & (fails to compile)
				\end{RpgTable}
				For robustness, we therefore recommend that all stylisation be applied through the \texttt{style} command, which is applied to each tokenised element, and therefore guaranteed to work as expected.
			}
		\RpgMacro{RpgDropCap}{\param{\paramO{}, m m}}
			{	
				Creates a decorative `drop cap' letter to begin a new chapter with, and modifies the following text.
			}
			{
					\bs{}RpgDropCap[<lettrine-args>]{<letter>}{<text>}
			}{
				This command uses \forcelink{https://texdoc.org/serve/lettrine/0}{the lettrine package} and the \forcelink{https://ctan.org/pkg/magaz?lang=en}{magaz} package to create an easy-to-use environment in which the first letter is enlarged (and stylised in the \texttt{RpgFontDropCap} font). The second argument formats \textit{up to the first line} of text in the \texttt{RpgFontDropCapInternal} font (usually a simple \texttt{scshape} command).

				This command can be a little fragile -- lettrine does not usually play well with the `FirstLine' command provided by magaz -- and we've used a few workarounds to allow both linebreaking, and the formatting of only the first line of text. There may need to be a small amount of manual calibration, but it is better than the default.					

			}
 \begin{ExampleBlock}[RpgDropCap]{RpgDropCap}
 \raggedright \RpgDropCap{T}{he example: this text runs over the first line, and then revert back to the normal font. It almost works! However, because it's wrapped in a text box, it goes slightly over the edges - and would require manual calibration.}
 \end{ExampleBlock}
		\RpgMacro{RpgSetFont}{\param{{m}}}
			{
				Saves new font values and styles to the internal RpgFont[X] variables, which are then available for themes to use.			
			}
			{
				\bs{}RpgSetFont{<key-value-pairs>}
			}{
				See \RpgPage{S:Fonts} for documentation of the available font families.

				The values changed by this command are local, and so persist only within a local group.
			}


\section{Dice Commands}
	Dice are a mainstay of RPGs, and so it is important to have a standard way to report and simplify their expressions. We provide an interface for a standard `dice + modifier' expression.
	\RpgMacro{RpgDice}{\param{{m}}}
		{
			Evaluates expressions of the form $n\mathrm{d}x \pm m$, and outputs using a theme-dependent layout.
		}
		{
			\bs{}RpgDice{<dice-expression>}
		}{
			Uses regular expressions to extract and simplify the \texttt{dice-expression}, which must follow the following format:
			\begin{RpgSidebar}{Dice format}
				\begin{multicols}{2}
				\begin{enumerate}
					\item It must contain either `d' or `D' (the `dice symbol')
					\item The dice symbol must be immediately followed by a single number (the `dice size')
					\item The dice symbol may optionally be prefixed by a single number (the `dice count')
					\item The first (non-whitespace) character must be either the dice count (if present) or the dice symbol
					\item The dice size must be followed by either a `+', '-', or the end of the expression.
					\item After this, any number of standard numeric expressions may follow. This expression will be evaluated into a single `modifier'.
				\end{enumerate} 
				\end{multicols}
			\end{RpgSidebar}
			The dice ignores any whitespace before the beginning of the expression, and arbitrary whitespace within the `modifier' part of the exprssion.  
			\begin{RpgTable}{XX}
				Example & Output \\
				\tabverbExample{\RpgDice{  1d6-2}}
				\tabverbExample{\RpgDice{2D6 + 3*2^2}}
				\tabverbExample{\RpgDice{1d16}}
				\tabverbExample{\RpgDice{d8-3}}
				\texttt{\bs{}RpgDice\param{2*1d6}}, \texttt{\bs{}RpgDice\param{1 d6}}, \texttt{\bs{}RpgDice\param{3d 6 +3}} & (Fails to compile)
			\end{RpgTable}
		
			\cmd{RpgDice} is neat, but not necessarily impressive by itself. The true power of the expression is that it calls \cmd{RpgDiceFormat} to perform the output formatting (after performing the regular expression parsing), allowing designers to customise their dice formatting.
		}

	\RpgMacro{RpgDiceFormat}{\param{{m m m}}}
		{Prints the values computed by \cmd{RpgDice}
		}
		{
			\bs{}RpgDiceFormat\param{<dice-count>}\param{<dice-size>}\param{<added bonus>}
		}
		{
			This is a \placeholderCommand{}, used by theme designers to determine how \cmd{RpgDice} is rendered. The default option is:
			\cmd{RpgDiceFormat\param{m m m}\param{\#1d\#2 \#3}}, such that \cmd{RpgDice\param{ndx + a + b}} gives ``ndx + c'', where c is the numerical value of a+b, with an additional check to see if \texttt{\#3} is equal to 0 (to avoid `1d6 + 0'). 
			
			The dnd implementation performs a more advanced operation, computing the average value of the roll, and formatting that first, to replicate the format used by monster stat blocks. 

			\RenewDocumentCommand{\RpgDiceFormat}{m m m}{\DndTempDiceFormat{#1}{#2}{#3}}	
			\begin{RpgTable}{XX}
				Example (with \texttt{\bs{}RpgSetTheme{dnd}}) & Output \\
				\tabverbExample{\RpgDice{  1d6-2}}
				\tabverbExample{\RpgDice{2D6 + 3*2^2}}
				\tabverbExample{\RpgDice{1d16}}
				\tabverbExample{\RpgDice{d8-3}}
			\end{RpgTable}
		}

\section{Theme Commands}

	\RpgMacro{RpgLayoutOnly}{\param{{m}}}
		{	
			Executes the contents of the command if \texttt{layout} mode is active.
		}
		{
			\bs{}RpgLayoutOnly\param{<content-to-execute>}
		}
		{
			If the internal value \texttt{\textbackslash{}l\_\_rpg\_layout\_bool} is True, then \texttt{content-to-execute} is run, otherwise it is ignored.

			This command is primarily used by theme developers and document class files to conditionally load or activate modules based on whether the package was loaded via a document class (layout mode active) or directly via \texttt{\bs{}usepackage\param{rpgtex}}.
		}
	\RpgMacro{RpgSetFooterDecoration}{\param{{o m}}}{Configures an image to be displayed along the bottom of a page as a `footer scroll'.}
		{
			\bs{}RpgSetFooterDecortation[<opts>]\param{path/to/img}
		}
		{
			When placed within a footer, (i.e. with fancypage), places the image in a node with parameters:
			
			\texttt{\textbackslash{}node[inner sep=0pt,anchor=south,nearly opaque] at (current page.south) \{\textbackslash{}includegraphics[width=\textbackslash{}paperwidth]\{path/to/img\}\};}

			If the package option \texttt{bg=none} has been passed, then the image is suppressed.

			The following options modify that code as follows:
			\begin{description}
				\item[reverse] adds \texttt{xscale=-1} to the node arguments, reversing the image (useful for right/left page differences)
				\item[tikz-insert={code}] inserts the code within the tikz environment after the footer scroll. This is not suppressed with \texttt{bg=none} and can be used to place chaptermarks / page numbers more precisely than the standard interface allows.
				\item[height=<dimexpr>] adds \texttt{height=dimexpr} to the includegraphics arguments
				\item[keepaspectratio] adds \texttt{keepaspectratio} to the includegraphics arguments    
			\end{description}
		}
	\RpgMacro{RpgSetPaper}{\param{}}
		{Sets a background image to be used as the `paper' image.}
		{
			\bs{}RpgSetPaper\param{path/to/image}
		}
		{
			If \texttt{layout} mode is active, then this configures \rpgtex{} to use the image as the `background image' of every page with \texttt{fancy, plain} or \texttt{clear} pagestyle. This allows for custom `paper textures' to be loaded in in the background. 

			The pagestyle \texttt{clear} is equal to \texttt{empty}, with the exception of the page texture.
		}
	\RpgMacro{RpgSetTheme}{\param{{m}}}
		{
			Activates a chosen theme.
		}
		{\bs{}RpgSetTheme\param{<theme-name>}}
		{
			Searches for the file \texttt{<theme-path>/<theme-name>/<theme-name>.cfg}, and inputs it. If this is a properly configured theme file, then it activates the chosen theme given the current global parameters. If the file does not exist, throws an error.

			If   \texttt{\textbackslash{}l\_\_rpg\_layout\_bool} is True, the command automatically inserts \texttt{\bs{}clearpage}, as required to ensure the old headers are not overwritten by the new theme.

			\texttt{<theme-path>} is modified via \cmdref{RpgSetThemePath}.
		}

	\RpgMacro{RpgSetThemeColor}{\param{{m}}}
		{
			Sets the \texttt{themecolor}, and simultaneously updates the co-varying colors \RpgPage[p]{S:Colors}.
		}{
			\bs{}RpgSetThemeColor\param{color-name}
		}{
			If \texttt{color-name} specifies a valid color, then the value of \texttt{themecolor} is updated, as well as a number of other colors (\texttt{tipcolor}, \texttt{sidebarcolor} and \texttt{tablecolor}) which are set to be equal to the themecolor by default.

			Of the rpg-provided colors, only \texttt{narrationcolor} is unaffected by this command.
		}
	\RpgMacro{RpgSetThemePath}{\param{{m}}}
		{
			Changes the value of the theme path searched for by \cmd{RpgSetTheme}
		}
		{
			\bs{}RpgSetThemePath\param{<path-name>}
		}
		{
			Updates an internal variable to be equal to the input value; does not check if the theme path is valid or not. Useful if you wish to create a new theme outside of the \texttt{rpgtex} file structure.
		}
		
\section{Utility Commands}

	\RpgMacro{RpgOrdinal}{\param{{o m}}}
		{
			Converts a numeric value to the corresponding ordinal.
		}
		{
			\bs{}RpgOrdinal[<command>]\param{<count>}
		}
		{
			The command outputs the \texttt{count} followed by the english abbreviations for the corresponding ordinal. The optional \texttt{command} argument is inserted between the numeral and the suffix, allowing for the customisation of appearances.
			\begin{RpgTable}{XX}
				Example & Output \\
				\tabverbExample{\RpgOrdinal{1}}
				\tabverbExample{\RpgOrdinal{2}}
				\tabverbExample{\RpgOrdinal{13}}
				\tabverbExample{\RpgOrdinal[\textsuperscript]{7}}
				\tabverbExample{\RpgOrdinal[\textbf]{133}}
				\tabverbExample{\RpgOrdinal[<arbitrary text>]{133}}
			\end{RpgTable}
			{\it Note that due to a lack of brace-capturing, it is not possible to chain multiple commands.}.
		}
	\RpgMacro{RpgPage}{\param{{O{t} m}}}
		{
			Outputs the current page reference for a label, with an option to enclose it in specific brackets or parentheses.
		}
		{
			\bs{}RpgPage[t/p/b/c]\param{<label-reference>}
		}
		{
			The optional arguments wrapping of the main reference. The options are:
			\begin{description}
				\item[t (default)] No wrapping
				\item[p] (parentheses)
				\item[b] [square brackets]
				\item[c] \{curly braces\}
			\end{description}
			An invalid input resolves to \texttt{?page~\bs{}pageref\param{<ref>}?}.\label{example:current page}
			
			\begin{RpgTable}{XX}
				Example & Output \\
				\tabverbExample{\RpgPage{example:current page}}
				\tabverbExample{\RpgPage[p]{example:current page}}
				\tabverbExample{\RpgPage[b]{example:current page}}
				\tabverbExample{\RpgPage[c]{example:current page}}
				\tabverbExample{\RpgPage[(error)]{example:current page}}
			\end{RpgTable}
		}
	\RpgMacro{RpgPlural}{\param{{o m m}}}
		{
			Generates grammatically correct plural forms of a word based on a given count.
		}
		{
			\bs{}RpgPlural[<custom-plural>]\param{count}\param{<text>}
		}
		{
			The command outputs the count followed by the value of \texttt{<text>}. For a count of 1, the command then finishes. For any other count, it appends an ``s'', pluralizing the text.

			The optional argument \texttt{[<custom-plural>]} overrides the default logic, allowing for irregular plurals.


			\begin{RpgTable}{XX}
				Example & Output \\
				\tabverbExample{\RpgPlural{1}{hat}}
				\tabverbExample{\RpgPlural{2}{hat}}
				\tabverbExample{\RpgPlural[octopodes]{1}{octopus}}
				\tabverbExample{\RpgPlural[octopodes]{359}{octopus}}
			\end{RpgTable}
		}

		


		\chapter{Environments}
	\newcommand\labelsection[1]
	{
		\section{#1}\label{S:#1}
	}

	
	\labelsection{RpgNarration}
	\labelsection{RpgSidebar}
	\labelsection{RpgTable}
	\labelsection{RpgTip}

		\chapter{Variables}
	\section{Colo(u)rs}\label{S:Colors}

		\rpgtex{} by default defines five colors\footnote{Yes, I hate myself, but we're going with the code-based spelling.} which are used for different elements:
		\begin{description}
			\item[themecolor] A `basic color' which is (by default) equal to the following three colors:
			\begin{enumerate}
				\item \textbf{sidebarcolor} The background color of the \verb|RpgSidebar| environment
				\item \textbf{tablecolor} The background color of every other row in an \verb|RpgTable| 
				\item \textbf{tipcolor} The background color of the \verb|RpgTip| environment 
			\end{enumerate} 
			\item[narrationcolor]  The background color of the \verb|RpgTip| environment 
		\end{description}

		Calling \verb|\RpgSetThemeColor| \RpgPage[p]{Macro:SetThemeColor} updates the value of \verb|themecolor|, as well as the three `co-varying' colors (i.e. everything except \verb|narrationcolor|). When \verb|printmode| is active \verb|\RpgSetThemeColor{white}| is called, making environments transparent.

		\chapter{Fonts}
	\begin{RpgSidebar}{Family vs Style}
		When defining the Font for an element, the interface allows one to specify both a \texttt{family} and a \texttt{style}. Formally speaking, \texttt{family} defines the \textbf{typeface} used by the associated element, whilst the \texttt{style} determines the options passed to that typeface (bold, italics, size etc.).

		The distinction is largely irrelevant, as the construction of the final font object is often simply the concatenation of the two:
		\begin{lstlisting}
			\def\RpgFontX
			{
				\l__rpg_x_family \l__rpg_x_style
			}
		\end{lstlisting}
		The separate definitions is therefore largely a matter of clarity and readability. It is generally fine to place commands should be `family' in the `style' element.
	\end{RpgSidebar}
	\section{Font Names}

	\section{Contours}
		
		\begin{macrolist}
			\RpgMacro{\RpgContour,{{O{} m}}}
				{
					Renders text with a \RpgContour[inner=red,outer=black]{contour effect}. The color and style are set through key/value pairs.
				}
				{
					\RpgContour[inner=<color>,outer=<color>,style=<code>]{<text>}
				}
				{
					The \texttt{style} command is applied to the text, whilst the optional \texttt{inner} and \texttt{outer} commands set the base text colour and the external contour color respectively. If the colors are not set, the default values are the \verb|contourinnercolor| and \verb|contouroutercolor| values defined by the theme \RpgPage[p]{S:Colors}.
					
					The contour does not automatically linebreak, but can be controlled manually with a \verb|\newline| command (not \verb|\\| or \texttt{\textbackslash{}par})
					
					\begin{RpgTable}{lX}
						Example & Output \\
						\tabverbExample{\RpgContour[inner=red,outer=black]{example}}
						\tabverbExample{\RpgContour[style=\Huge\it]{example}}
						\tabverbExample{\RpgContour[]{multi\newline line\newline example}}
					\end{RpgTable}					
					~\macrosection{Quirks}

					Due to the tokenisation required for the line-splitting and space-preservation, the text inside the contour can exhibit some quirks if stylisation is applied within the \verb|<text>| argument. 

					Unbraced commands (such as \verb|\it| or \verb|\footnotesize|) will only apply to the first word in the text. Braced commands \textit{can} work, but will cause a compilation error if a \verb|\newline| is included. 

					
					\begin{RpgTable}{lX}
						\footnotesize\tabverbExample{\RpgContour[]{\Huge\it only first word changes}}
						\footnotesize\tabverbExample{\RpgContour[]{\textit{all words change}}}
						\footnotesize\verb|\RpgContour[]{\textit{all word \newline change}}| & (fails to compile)
					\end{RpgTable}
					For robustness, we therefore recommend that all stylisation be applied through the \verb|style| command, which is applied to each tokenised element, and therefore guaranteed to work as expected.
				}
		\end{macrolist}

		\chapter{rpglatex Compiler}\label{C:Compiler}


	\rpgtex{} is shipped with a special compiler, \verb|rpglatex|. This is simply a python3 script which acts as a wrapper around either xelatex or luatex, but includes several quality-of-life changes to the interface to make it easier to use with \verb|rpgtex|.

	\begin{macrolist}
		\RpgMacro[rpglatex]{rpglatex}{Compiles latex documents using either xelatex or luatex}
		{
			rpglatex [options] <file> 
		}{
			\verb|rpglatex| has the following features:

			\newcommand\feature[3]
			{
				\textbf{#1} & #2 & \texttt{#3} \\
			}
			\begin{RpgTable}[width=\linewidth]{lXl}
				Feature & Description & Options \\
				\feature{Compiler Selection}{The \verb|xelatex| compiler is selected by default, but the \verb|-l, --luatex| flags set it to use luatex instead.}{-l, --luatex}
				\feature{Build Directory}{Compilation files (.aux, .log etc.) are stored in a build directory. The default is \verb|.build| in the calling location, but can be changed with the \verb|-b| flag}{-b <build dir>}
				\feature{Output Naming}{The name of the output file can be changed from the default (equal to the input tex name)}{-o <output name>}
				\feature{Multi-pass Compiling}{By default, the compiler runs twice in a row to enable references and \texttt{tikz[remember]} commands to function. A full three-compilation suite (necessary for very complex or reference-heavy documents) can be activated with the \verb|-f, --full| flag}{-f, --full}
				\feature{Volume Control}{latex is notoriously noisy, producing copius output. By default, this is suppressed and only a summary is printed. The summary can be removed (rendering it completely silent) with the \verb|-q| command, or the original output recovered in verbose mode; \verb|-v|.
				
				These outputs are always overriden if a compilation error occurs, in which case the full trace is output to the console.
				}{-q, -v}
				\feature{Auto-bibtex}{If the \verb|-r| or \verb|--ref| flag is set, \verb|bibtex| is automatically called in between the multi-compilation steps}{-r, --ref}
				\feature{Auto-visualisation}{If the \verb|--show 1| option is set (which it is by default), the compiler will call \verb|xdg-open <output-file>| upon completion of the compilation; automatically opening or context-switching to the document. This can be turned off by calling \verb|--show 0|}{--show}
				\feature{Print Mode}{A special interface for \rpgtex{}, this uses the \verb|\RpgCMD| interface \RpgPage[p]{S:CMD} to inject code into the latex document, setting the \verb|bg=print| mode and suppressing the background output.}{-p, --print}
			\end{RpgTable}
		}
	\end{macrolist}

	
	\part{rpgtex Classes}
		\chapter{rpgbook Class}\label{S:bookClass}

		\chapter{rpghandout Class}\label{S:handoutClass}


		\chapter{rpgcard Class}\label{S:cardClass}

	\part{Themes}
		\RpgSetTheme{default}
\chapter{\texttt{default} Theme}\label{Theme:Default}


	\section{RpgItem: default}\label{S:DefaultItem}\index{RpgItem!Theme!default}

		The default value of the \envref{RpgItem}
	
	\section{RpgFeat: default}\index{RpgFeat!Theme!default}
	
	\section{RpgSpell: default}\index{RpgSpell!Theme!default}

		\RpgSetTheme{dnd}
\chapter{\texttt{dnd} Theme}\label{Theme:dnd}

    The \texttt{dnd} theme is designed to mimic the Appearance of the \textit{Dungeons \& Dragons} source books. The specific parameters used to replicate the D\&D books are mostly derived from the original DnD-5e-Latex-Template package, with some additional updates in line with the changes made with the 2024 rules update.

    \section{Appearance}
        \subsection{Fonts}
            The dnd theme defines a number of font families which are used throughout the theme.
            
            \RpgMacro{bookman}{}{The {Bookman Old Style STd} typeface, used as the main body text.}{}{}{}
            \RpgMacro{keplerserif}{}{\keplerserif The {KpRoman} font, used for \textbf{emphasis}}{}{}{}
            \RpgMacro{kepler}{}{\kepler The {KpSans} font, primarily used in headers and titles}{}{}{}
            \RpgMacro{gillius}{}{\gillius The Gillius ADF No. 2 Font, used as a lighter sans-serif font, often in the body of environments where \cmd{kepler} was used as the title.}{}{}{}
            \RpgMacro{Royal}{}{{\Royal A DECORATIVE FONT}, unsuited for blocks of text. Used for drop-caps}{}{}

             The values assigned to the font elements areas follows (note that \verb|\normalfont| is aliased to the value of RpgFontBody).

            \InsertFontTable{}
            
        \subsection{Colors}\label{S:DndColor}

            The theme provides a large number of colors:

            \noindent\DefaultSwatches{}

            \noindent\ignorespaces\Swatch[default narration]{bgtan}
            \Swatch[pagenumbers \& footer]{pagegold}
            \Swatch[title font \& default contour-inner]{titlered}
            \Swatch[default contour-outer]{contourgray}
            \Swatch[title rules]{titlegold}
            \Swatch[statblock triangles]{rulecolor}
            \Swatch{statblockribbon}
            \Swatch{statblockbg}
            \Swatch[Basic Rules]{BrGreen}
            \Swatch[PHB Part 1]{PhbLightGreen}
            \Swatch[PHB Part 2]{PhbLightCyan}
            \Swatch[PHB Part 3]{PhbMauve}
            \Swatch[PHB Appendix]{PhbTan}
            \Swatch[DMG Part 1]{DmgLavender}
            \Swatch[DMG Part 2]{DmgCoral}
            \Swatch[DMG Part 3]{DmgSlateGray}
            \Swatch[DMG Appendix]{DmgLilac}

            The colors from BrGreen through DmgLilac are chosen so that, when set as the \texttt{themecolor} (using \cmdref{RpgSetThemeColor}), the appearance of the tables and sidebars is the same as the corresponding part in either the Basic Rules, the Players' Handbook, and the Dungeon Master's Guide.
        \subsection{Backgrounds \& Footers}

            The dnd theme uses a paper-like image as the background, and places a `scroll' on the footer\footnote{The scroll alternates direction in \texttt{twoside} documents; this document is written in oneside mode, so there is no alternating.}. The chapter name is placed in the footer (in \texttt{pagegold} color). The page number is positioned to lie in a divot of the scroll.

        \subsection{Text Boxes}

            The dnd theme modifies the text boxes to give them the following appearance:

            \begin{multicols}{3}
                
                \begin{RpgSidebar}{The RpgSidebar}
                   
                    Horizontal `ribbons' and sharp corners differentiate it from RpgTip.
                \end{RpgSidebar}
                
                \begin{RpgTip}{The RpgTip}
                    Almost entirely undecorated, with rounded corners.
                \end{RpgTip}

                \begin{RpgNarration}
                    RpgNarration has `bars' on the side.
                \end{RpgNarration}
            \end{multicols}



        \subsection{Section Headers}

            As in the default mode, only chapters\footnote{If in an rpgbook / other class which support chapters} are numbered; all other section/subsections (etc.) are unnumbered. 

            Chapters are rendered using a \cmd{RpgContour} (with the default colors). The subsection environment is embellished with a \cmd{hrule} (in the \texttt{titlegold} color) which stretches across the page.

            \RpgGetExample{dnd-section-headers}

        \subsection{RpgDice}\index{RpgDice}
            
            Following the syntax found in D\&D monster statblocks, RpgDice shows the average of the roll:
            \begin{ExampleBlock}{Dnd Dice}
            \RpgDice{3d8 -5 +2}
            \end{ExampleBlock}

        \subsection{Filigree}\label{S:DndFiligree}

            The dnd implemenation of \envref{RpgFiligree} is designed to mimic the decorative elements seen around the class tables in the Player's Handbook:
            
\begin{ExampleBlock}{D\&D Filigree}
    \begin{RpgFiligree}
        Some decorative text
    \end{RpgFiligree}

    \vspace{1cm}

    \begin{RpgTable}[filigree]{lX}
        Header 1 & Header 2
        \\
        Body 1 & Body 2
        \\
        Body 3 & Body 4
    \end{RpgTable}
\end{ExampleBlock}

    \newpage
    \section{RpgItem}\label{S:DndItem}

        \RpgMacro[RpgItem!Theme!dnd]{RpgItem}{\param{\paramO m \paramO}}{The dnd-specialisation of the \envref{RpgItem} environment, used to describe physical objects and equipment.}
        {
            \cmd{RpgItemShowCard\param{true/false}} \% set the card mode

            \cmd{begin}\param{RpgItem}[card-opts]\param{Item Name}[key-values]

            ~~<body text>

            \cmd{end}\param{RpgItem}

        }
        {As with all FeatureForge environments, RpgItem has the ability to switch between `text mode' and `card mode' depending on the value passed to \cmd{RpgItemShowCard} \RpgPage[p]{Macro:Rpg[X]ShowCard}. 
        
        The \texttt{dnd} theme defines the following keys for the RpgItem object:
        \begin{RpgTable}{llX}
            Key & Default Value & Effect
            \\
            rarity & \param{} & Common / Uncommon / Rare etc.
            \\
            type & \param{} & A descriptor such as `weapon' or `wondrous item'
            \\
             requires-attunement & \texttt{none} & Can either be used as a flag (i.e. a key with no value) in which case the phrase `Requires attunement' is added. If a value is assigned, it is appended immediately afterwards.
             \\
            image & \param{} & If non-empty, define an image path which is used when in card-mode.
        \end{RpgTable}}
    The card-variant  renders the title in a 'flag ribbon' which expands to fit the title (and shrinks the font size if it would spill over the card boundary)
    \begin{ExampleBlock}{Default RpgItem}
    \RpgItemShowCard{false}
    \begin{RpgItem}{Joyeuse}[
        rarity=Rare,
        type=weapon (sword),
        %just a flag, no value
        requires-attunement, 
        image={../example/img/joyeuse}
    ]
        The sword of Charlemagne; this jewelled sword gives you +3 to heroism checks.
    \end{RpgItem}

    \vspace{0.5cm}\hrule\vspace{0.5cm}

    %now the card version
    \RpgItemShowCard{true}
    \begin{RpgItem}{Joyeuse}[
        rarity=Rare,
        type=weapon (sword),
        %now add a value...
        requires-attunement={by a Paladin of France}, 
        image={../example/img/joyeuse}
    ]
        The sword of Charlemagne; this jewelled sword gives you +3 to heroism checks.
    \end{RpgItem}
    \end{ExampleBlock}
    
    
\section{RpgFeat}\label{S:DndFeat}\index{RpgFeat!Theme!default}

    \RpgMacro[RpgFeat!Theme!dnd]{RpgFeat}{\param{\paramO m \paramO}}{ The dnd-specialisation of the \envref{RpgFeat} environment, used to describe abilities and character features and choices.}
        {
            \cmd{RpgFeatShowCard\param{true/false}} \% set the card mode

            \cmd{begin}\param{RpgFeat}[card-opts]\param{Feat Name}[key-values]

            ~~<body text>

            \cmd{end}\param{RpgFeat}

        }
        {
        As with all FeatureForge environments, has the ability to switch between `text mode' and `card mode' depending on the value passed to \cmd{RpgFeatShowCard} \RpgPage[p]{Macro:Rpg[X]ShowCard}. The \texttt{dnd} RpgFeat is largely the same as the default, using the same `requires' key, and adding an alias:
            \begin{RpgTable}{llX}
                Key & Default Value & Effect
                \\
                requires & \param{} & If non-empty, an italicised note is added, indicating the prerequisites for acquiring the ability.
                \\
                prerequisite & \param{} & An alias for 'requires' (since this is what appears on screen).
            \end{RpgTable}
        }

        \def\grapplerText{  You've developed the skills necessary to hold your own in close--quarters grappling. You gain the following benefits:
     \begin{itemize}
       \item You have advantage on attack rolls against a creature you are grappling.
       \item You can use your action to try to pin a creature grappled by you. To do so, make another grapple check. If you succeed, you and the creature are both restrained until the grapple ends.
     \end{itemize}
}
     \begin{ExampleBlock}{Dnd RpgFeat}
    \RpgFeatShowCard{false}
    \begin{RpgFeat}{Grappler}[
         requires={Strength 13 or higher},
              ]
        \grapplerText %predefined to save space. Just plain text!
    \end{RpgFeat}

    \vspace{0.25cm}\hrule\vspace{0.25cm}

    %now the card version
    \RpgFeatShowCard{true}
    %demonstrate card-opts
    \begin{RpgFeat}[color=DmgLilac]{Grappler}[
        %%alias in action; could use 'requires' for same effect
         prerequisite={Strength 13 or higher},
            ]
       \grapplerText
    \end{RpgFeat}
    \end{ExampleBlock}
\section{RpgSpell}

    \RpgMacro*[RpgSpell!Theme!dnd]{RpgSpell}{\param{\paramO m \paramO}}{ The dnd configuration of the \envref{RpgSpell} environment, used to describe magical spells or spell-like abilities.}
        {
            \cmd{RpgSpellShowCard\param{true/false}} \% set the card mode

            \cmd{begin}\param{RpgSpell}[card-opts]\param{Spell Name}[key-values]

            ~~<body text>

            \cmd{end}\param{RpgSpell}
        }
        {As with all FeatureForge environments, has the ability to switch between `text mode' and `card mode' depending on the value passed to \cmd{RpgSpellShowCard} \RpgPage[p]{Macro:Rpg[X]ShowCard}. The RpgSpell formats the components of a D\&D spell by defining the following keys:
               \begin{RpgTable}{llX}
                Key & Default Value & Effect
                \\
                school & \param{} & The spell school (conjuration etc.). Can be empty.
                \\
                level & 0 & The level of the spell, expressed as a single integer. 0 is converted into `Cantrip'.
                \\
                casting-time & 1 action & A plaintext field for the casting time (1 action, 1 reaction etc.) 
                \\
                range & self & The range of the spell.
                \\
                components & VSM & The Verbal/Somatic/Material components for casting the spell
                \\
                duration & Instantaneous & The duration of the spell (and its concentration status)
                \\
                source & \param{} & An optional field for declaring the source of the spell (i.e. the PHB).
            \end{RpgTable}
            All these fields support a \cmd{footnote}, but note that per the RpgCard documentation \RpgPage[p]{RpgCard}, only a single footnote can be included when in card-mode; successive footnotes override each other.

            In addition to these keys, we also provide some commands for typesetting some `boilerplate language' found in D\&D spells:
        }
    \RpgMacro{RpgCantripScaling}{\param{O\param{Damage} m m m m}}{Provides a convenient interface for typesetting the `level scaling' of cantrips in D\&D}
    {
        \cmd{RpgCantripScaling}[benefit-name]\param{1st-lvl}\param{5th-lvl}\param{11th-level}\param{17th-level}
    }
    {
        The command is typeset with some boilerplate text, followed by a \envref{RpgTable}, with each argument as an entry used as the corresponding level in the table.
    }
    
    \RpgMacro{RpgSpellUpcast}{\param{O{} m O{}}}{Provides a convenient interface for typesetting the `upcasting' (increased benefits when using higher level spell slots) of spells in D\&D}
    {
        \cmd{RpgSpellUpcast}[benefit-of-upcasting]\param{benefit-per-increase}[slots-per-increase]
    }
    {
        The default `benefit' text is ``\textit{The damage increases by}''; with the expectation that the mandatory argument is a damage increase (such as 1d6). The slots-per-increase is an integer value (default 1), indicating the number of level gains per increase.
    }
    \ExplSyntaxOn
    \tl_set:Nn\l__rpg_dnd_spell_level_tl{3}
    \ExplSyntaxOff
    \begin{ExampleBlock}{Spell Helper Functions}
     \RpgCantripScaling[\# of Beams]{1} {3} {5}{9d10}

     \vspace{0.5cm}

     \RpgSpellUpcast[The duration increases by]{1 minute}[2]
    \end{ExampleBlock}
\begin{ExampleBlock}[RpgSpell]{Example Spells}
    \RpgSpellShowCard{false}
    \begin{RpgSpell}{Firebolt}[
        school=Evocation,
        level = 0,
        casting-time=1 action,
        range=120ft,
        components = VS,
        duration=Instantaneous,
        source=Free Basic Rules (2014)
       ]
    You hurl a mote of fire at a creature or object within range. Make a ranged spell attack against the target. On a hit, the target takes 1d10 fire damage. A flammable object hit by this spell ignites if it isn't being worn or carried.

        \RpgCantripScaling{1d10}{2d10} {3d10}{4d10}
    \end{RpgSpell}

    \vspace{1cm}

    \RpgSpellShowCard{true}
    \begin{RpgSpell}{Cone of Cold}[
        school=Evocation,
        level = 5,
        casting-time=1 action,
        range=self (60ft cone),
        components = VSM\footnote{A small crystal or glass cone},
        duration=Instantaneous,
        source=Free Basic Rules (2014)
       ]
     A blast of cold air erupts from your hands. Each creature in a 60-foot cone must make a Constitution saving throw. A creature takes 8d8 cold damage on a failed save, or half as much damage on a successful one. A creature killed by this spell becomes a frozen statue until it thaws.

        \RpgSpellUpcast{1d8}
    \end{RpgSpell}
\end{ExampleBlock}
\clearpage
\section{RpgStat: D\&D Monsters}

    Internally, the monster statblock is one of the most complex environments in the \rpgtex{} library; the upside being that it makes it quick and easy to generate a monster which follows the standard D\&D monster assumptions.

    \RpgMacroExample*[RpgStat!Theme!dnd]{RpgStat}{RpgStat*,\param{\paramO m \paramO}}{ The dnd configuration of the \envref{RpgStat} environment, used to describe monsters and enemies, using the rules and parameters of spells in D\&D 5e. As with all FeatureForge environments, RpgStat has the ability to switch between `text mode' and `card mode' depending on the value passed to \cmd{RpgStatShowCard} \RpgPage[p]{Macro:Rpg[X]ShowCard}. }
        {dnd-stat-card}
        {	
             \begin{RpgSidebar}{RpgStat vs RpgStat*}
                 \texttt{RpgStat*} is identical to the unstarred version, except it automatically activates the \texttt{twocolumn} mode, and sets \texttt{float-type = figure*} (see below). This makes it suitable for rendering 'boss' (i.e. large) stat blocks whilst writing twocolumn documents. 
            \end{RpgSidebar}
            We semantically split the standard \forcelink{https://media.wizards.com/2018/dnd/downloads/DnD_BasicRules_2018.pdf\#page=110}{D\&D statblock} into two parts:
            \begin{enumerate}
                \item The \textit{statistics} - hit points, armour class, stat modifiers, skill proficiencies and so on. These are highly structured and amount to 'form filling'. 
                
                \textbf{These are defined in the key-value options}.
                \item The \textit{abilities} - including the traits, actions, bonus actions and so on. These are freeform and largely consist of plain text.
                
                \textbf{These are defined in the body of the environment}.
            \end{enumerate}    
            
         
        }
        \begin{center}
            \begin{tikzpicture}
                \node[text width=0.45\linewidth] at (0,0) {
                    \begin{RpgStat}{Commoner}[
                        type={Medium or Small Humanoid; typically neutral},
                        languages=Common
                    ]
                        \section{Traits}
                            \action{Mob Mentality} If a Commoner is subjected to a mind-altering effect (such as the frightened condition, or a charm spell), all Commoners within 10ft are also subjected to this effect.
                        \section{Actions}
                            \melee{Club}
                    \end{RpgStat}
                };
                \draw[decorate,decoration={brace,amplitude=10pt,raise=-7pt},ultra thick] (4.5,4.25)--++(0,-4.75) node[midway,anchor=west]{\bf ~~Statistics (key/values)};
                \draw[decorate,decoration={brace,amplitude=10pt,raise=-7pt},ultra thick] (4.5,-0.6)--++(0,-3.75) node[midway,anchor=west]{\bf ~~Abilities (body text)};
            \end{tikzpicture}
        \end{center}
          
        \clearpage
        \subsection{Statistics Options}

            There are a large number of options that can be passed to the statistics. Where multiple key names are presented, these are aliased to the same value and behave identically. As with all expl3 keys, repeated values are sequentially overwritten, with only the final value retained.
            \newcommand{\statkey}[4]{\\\parbox[t]{2.5cm}{\raggedright #1} & #2 & #3 & #4}
            \newcommand{\statflag}[2]{\\\parbox[t]{2cm}{\raggedright #1} & (flag) & ~ & #2}
            \newcommand{\catbreak}[1]{\\\multicolumn{4}{l}{\textit{#1}}}
            \newcommand{\tablesection}[2]
            {
                \subsubsection{#1}
                \begin{RpgTable}[breakable,vskip=2pt plus 2pt minus 2pt]{lllX}
                Key & Value & Default Value &Effect 
                #2
                \end{RpgTable}
            }
            \tablesection{Appearance}
            {
                    \statkey{color}{tikz-color code}{rulered \RpgPage[p]{S:DndColor}}{The base color used by the environment. Several colors are automatically derived from this:
                        \begin{itemize}
                            \item The `triangle dividers' use this color
                            \item The 'header text' color is 50\% darker\footnote{In the sense that it is defined as \texttt{color!50!black}}
                            \item The border color is 20\% lighter (unless otherwise specified)
                            \item The \texttt{themecolor} is temporarily set to a value 75\% lighter.
                        \end{itemize}
                        }
                    \statkey{outline-color}{tikz-color code}{--empty--}{If non-empty, this color is used for the border of the environment, intstead of the one derived from 'color'. No effect when in Card mode.}
                    \statflag{filigree}{If this flag is present, replaces the default border with filigree \RpgPage[p]{S:DndFiligree}, using the current \texttt{filigreecolor}. No effect when in Card mode.}
                    \statflag{filigree-match}{If this flag is present, replaces the default border with filigree \RpgPage[p]{S:DndFiligree} and locally changes \texttt{filigreecolor} to be equal to the border color. If both \texttt{filigree-match} and \texttt{filigree} present, \texttt{filigree-match} takes priority. No effect when in card mode.}
                    \statflag{twocolumn}{If this flag is present, the environment is rendered in twocolumns, suitable for breaking up statblocks which are full-page width (either when in a onecolumn document, or placed inside a full-width float). Automatically activated by RpgStat*. No effect when in card mode.}
            }
            \tablesection{Positioning}{
                    \statkey{float}{!/h/t/b}{--empty--}{If non-empty, wraps the statblock in a float (of a type determined by \texttt{float-type}), with the float command equal to this value.
                    
                    \begin{RpgSidebar}{IMPORTANT!}
                        Normally the `h' option does not work for \texttt{figure*} when in a twocolumn environment. We have implemented a workaround which uses the \texttt{strip} environment to simulate a ``immediate full width image''. \textbf{This is very fragile!} It is likely to throw compilation errors when on the same page as other floats, or when there is insufficient text before or after the environment begins. Use with extreme caution!
                    \end{RpgSidebar}

                    No effect when in card mode.
                    }
                    \statkey{float-type}{environment name}{figure}{The environment which wraps the statblock if a float command is given; setting this to \texttt{figure*} creates a full-width statblock when in a twocolumn document.}
            }
            \clearpage
            \tablesection{Basic Description}{
                    \statkey{nickname}{text}{<title value>}{A shorter or alternative version of the `main name' which appears at the top of the statblock; this is the value returned by \cmdref{RpgStatName}, and used automatically in several places. Useful when you want to give the statblock a grand title, but don't want ``\textit{Gorgenhar Haluavin, the Devourer of Worlds} makes three attacks''.}
                    \statkey{type}{text}{(empty)}{This text is placed in italics underneath the main title. Usually used for size and alignment declarations.}
                    \statkey{armor-class / ac / AC}{text}{10}{The value given as the armour class of the monster. Can accept strings explaining the armour such as `15 (Natural Armour)' or `10 (13 with \textit{mage armor})'}
                    \statkey{hit-points / hp / HP}{text}{\cmd{RpgDice\param{1d8}}}{The value given as the total HP of the monster. The value given is usually as a \cmdref{RpgDice} command. This value is expanded after the modifiers are computed, so it is possible to use (i.e.) the consitution modifier as a variable: \texttt{hp=\cmd{RpgDice}\param{8d10 + 8*\cmd{stat}\param{con}}} will be correctly computed.}
                    \statkey{speed}{text}{30ft}{Value given as the monster's movement speed. If the substring `ft' is not found in the input, then it is appended to the end; i.e. \texttt{speed=30} will render as '30ft'.}
                    \statkey{initiative}{text}{}{If this value is non-empty, it is used (without further formatting) as the initiative of the creature. If the value is empty, the computed dexterity modifier is used instead.}

            }
            \tablesection{Abilities \& Saves}{
                    \statkey{proficiency-bonus / proficiency / pb}{integer}{(empty)}{If a value is provided, this is used as the proficiency bonus for all save bonus, skill bonuses and to-hit values. If not provided, a value is computed based on the challenge rating.}
                    \statkey{str}{integer}{10}{Sets the value of the Strength Score. The associated modifier is automatically computed} 
                    \statkey{dex}{integer}{10}{Sets the value of the Dexterity Score. The associated modifier is automatically computed} 
                    \statkey{con}{integer}{10}{Sets the value of the Constitution Score. The associated modifier is automatically computed} 
                    \statkey{int}{integer}{10}{Sets the value of the Intelligence Score. The associated modifier is automatically computed} 
                    \statkey{wis}{integer}{10}{Sets the value of the Wisdom Score. The associated modifier is automatically computed} 
                    \statkey{cha}{integer}{10}{Sets the value of the Charisma Score. The associated modifier is automatically computed} 
                    \statflag{str-save}{If present, this flag indicates the monster is proficient in Strength saving throws.}
                    \statflag{dex-save}{If present, this flag indicates the monster is proficient in Dexterity saving throws.}
                    \statflag{con-save}{If present, this flag indicates the monster is proficient in Constitution saving throws.}
                    \statflag{int-save}{If present, this flag indicates the monster is proficient in Intelligence saving throws.}
                    \statflag{wis-save}{If present, this flag indicates the monster is proficient in Wisdom saving throws.}
                    \statflag{cha-save}{If present, this flag indicates the monster is proficient in Charisma saving throws.}
            }
            \clearpage
            \tablesection{Skills \& Details}{
                \statkey{skills}{comma-separated list}{(empty)}{Each skill in this list is iterated over - if the value is one of the 18 D\&D skills, it is matched with its usual ability modifier, and a skill bonus is computed in the normal fashion. This bonus is appended to the skill name in the list when displayed.
                
                \begin{RpgSidebar}{Manual Skills}
                    The automated computing assumes that skills will always be paired with their normal attribute. If a monster uses a variant pairing (i.e. Strength (Intimidation)) then this will need to be manually declared. 

                    The automation only occurs if the string exactly matches\footnote{Whitespace and variant capitalisation count as `exact matches'} the names of the skills; so \texttt{athletics, Acrobatics (+25)} will cause the `athletics' to be computed automatically, but the Acrobatics will not count as a match, so will keep its manual bonus declaration.  
                \end{RpgSidebar}
                }
                \statkey{skills-expertise}{comma-separated list}{(empty)}{Exactly as with the \texttt{skills} key, except the automated bonus adds twice the proficiency bonus. Expertise-skills are listed before other skills in the rendered environemnt.}
                \statkey{languages}{text}{(empty)}{Any languages spoken or understood by the creature. This is used as a simple string (no manipulation)}
                \statkey{senses}{text}{(empty)}{Any special senses possed by the creature. Passive perception is handled separately.}
                \statkey{passive-perception}{integer}{(empty)}{If empty, the passive perception is equal to the Wisdom (Perception) bonus (computed from the Wisdom modifier, and if the perception skill was declared as proficient or expertise). If this value is non-empty, override that value with the given one.}
                \statkey{challenge / cr / CR}{number}{(special)}{If a value is provided, this sets the display value of the CR. If no value was provided to the CR, one is computed from the proficiency bonus using the usual formula. If no CR was provided and no PB, then the CR defaults to 0. 
                
                If both a CR and a proficiency bonus are manually provided, then no checks are performed to ensure they `make sense'. It is possible to have a CR0 creature with a +90 proficiency bonus, if the user manually sets those values.

                Non-integer values are accepted (i.e. 1/4).
                }
                \statkey{xp}{integer}{(empty)}{If this value is non-empty, it is used as the XP gain defeat. 
                
                If left empty, the XP is manually computed from an approximate formula which is accurate to $\pm 100$xp up to CR16:
                
                $$\text{XP} = \begin{cases} 100 + 50\Big(\text{CR}^2 + \text{CR} + 2 \text{CR} \lfloor \frac{\text{CR}}{7}\rfloor  &  \text{CR} \geq 1 
                    \\
                    \quad\quad- 7  \lfloor \frac{\text{CR}}{7}\rfloor^2 - 5  \lfloor \frac{\text{CR}}{7}\rfloor\Big) &
                    \\ 100 \times \text{CR} & \text{else}\end{cases}$$
                

                (This is a modified and analytically simplified version of the formula found on \forcelink{https://rpg.stackexchange.com/questions/192412/is-there-any-generalised-formula-for-converting-monster-cr-into-xp-in-dd-5e}{a helpful discussion online.})}

                \statkey{condition-immunities}{text}{(empty)}{Any condition immunities possessed by the creature.}
                \statkey{damage-immunities}{text}{(empty)}{Any damage immunities possessed by the creature.}
                \statkey{damage-resistances}{text}{(empty)}{Any damage resistances possessed by the creature.}
                \statkey{damage-vulnerabilities}{text}{(empty)}{Any damage vulnerabilities possessed by the creature.}
            }
    \subsection{Abilities, Attacks \& Spellcasting}

        The body of the environment is where the user places the traits, abilities and attacks that the creature can use. In order to format these, we provide the following commands.

        \subsubsection{Ability Interface}

            \begin{RpgSidebar}{Global \& Local Names}
                All of the commands we will introduce below follow the established syntax for the \rpgtex{} library: \cmd{RpgStat[Command]}. These are placed in the global namespace, and are accessible anywhere - a user may invoke RpgStat elements outside of the RpgStat environment.

                This naming convention - though consistent and avoiding 'common name collisions' - can become cumbersome when used in a dense environment like the RpgStat. We therefore provide several \textit{local aliases}. These are macros which can only be used inside the RpgStat environment. 

                With only one exception\footnote{\cmd{section}, because that will obviously already be defined!}, these aliases are created when the environment begins, using the \cmd{ProvideDocumentCommand} interface -- this means that the alias will only be created if a macro with that name \textit{doesn't exist}.

                The local alias is given in \textcolor{red}{red text}.
            \end{RpgSidebar}
            \newcommand\statalias[1]
            {
                \textcolor{red}{\cmd{#1}}\index{#1}
            }

            \RpgMacroExample[RpgStat!Section]{RpgStatSection}{\statalias{section},\param{m}}{Creates a section header}{dnd-stat-section}
            {
                Although called a `section', the visual is most similar to the \cmd{subsection}, though with the text rendered in the \cmd{RpgFontStatBlockSection} font.
            }

            
            
            \RpgMacroExample[RpgStat!Action]{RpgStatAction}{\statalias{action},\statalias{trait},\param{\paramO{} m}}{Creates an action `paragraph' for declaring traits and actions}{dnd-stat-action}
            {
                If a non-empty optional argument is passed, it inserts ``[X] actions'' into the action name. This is useful for Legendary Actions which might have a variable cost.
            }


            \RpgMacroExample[RpgStat!Reaction]{RpgStatReaction}{\statalias{reaction},\param{m m}}{Creates an action `paragraph' with an additional `trigger', indicating when the reaction may be used.}{dnd-stat-reaction}
            {}


        \subsubsection{Attacks}
            A thing!

        \subsubsection{Spellcasting}

        \subsubsection{Saving Throws}
            
        \RpgStatShowCard{false}
\RpgGetExample{dnd-stat-main-example}

    \subsection{Legacy Appearance}

        The appearance and structure of the RpgStat block is based on the improvements made by the 2024 rules release (with a few modifications of our own). A version that closely aligns with the original 2014 version of the rules can be activated:

        \RpgMacro{RpgStatLegacyMode}{}{Overrides the appearance of the RpgStat environment with 2014-style ones.}
        {
            \% call outside the environment


            \cmd{RpgStatLegacyMode}\{\}

    \% subsequent stats use the old visual style

            \cmd{begin\param{RpgStat}} \dots 
        }
        {
            The overrides are purely stylistic; the underlying computation remains the same. The changes persist within a local group.
        }

    \RpgStatShowCard{false}
    \RpgGetExample{dnd-stat-legacy}
		\RpgSetTheme{scifi}
\chapter{\texttt{scifi} Theme}

	
	\section{scifi RpgItem}\label{S:ScifiItem}\index{rpgitem@\texttt{RpgItem}!\texttt{scifi}}

This is a test of \emph{My emphasis} adh ouawwhd iuahwiud hwaiudh iuawhdu iahwiudh iuwahdiu ahw iudhiuahw iudhaw iduh awliudh lauiwhdl iuahw idulha wiuldhiuhwiuh 
\Blindtext
\clearpage
\Blindtext


	\RpgSetTheme{default}
	\printindex
\end{document}
