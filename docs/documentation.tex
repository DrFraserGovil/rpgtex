
\documentclass[size=10pt,theme=default]{rpgbook}

\usepackage{doc}
\usepackage{blindtext}
\usepackage{imakeidx}
\makeindex
\def\rpgtex{\texttt{rpgtex}}

\cover{../example/img/typewriter}

\title{The rpgtex Package}
\subtitle{A package for generating beautiful RPG documents}
\author{Jack Fraser-Govil}

\newcommand{\lindex}[1]{%
  \lowercase{\def\temp{#1}}%
  \expandafter\index\expandafter{\temp}%
}

\NewDocumentCommand{\cmdidx}{ o m }{
	\foreach \m in {#2}
	{
		% \lindex{\texttt{\textbackslash{}\lowercase{\m}}@{\lowercase{\m}}}
		\IfNoValueTF{#1}
		{
			 \lowercase{\def\temp{\m}}
			\expandafter\index\expandafter{\temp@\texttt{\textbackslash\m}}
		 }
		 {
			% \index{\m}
			 \lowercase{\def\temp{\m}}
			\expandafter\index\expandafter{#1!\temp@\texttt{\textbackslash\m}}
		 }
	}
}

\begin{document}
	\frontmatter
	\maketitle{}

	%%% fancy intro/abstract
		\onecolumn{}
		~\vfill{}
		\newcommand\forcelink[2]
		{
			\href{#1}{{\color{blue!70!black}#2}}
		}
		\begin{center}
		\parbox[c]{0.8\linewidth}{
			\large
			\justifying{}

			\RpgDropCap{W}{elcome to the \rpgtex{} package}. This \LaTeX{} package is designed to allow users to flexibly typeset documents associated with Role Playing Games such as \textit{Dungeons \& Dragons} -- and many more besides. This packages defines a central engine: \texttt{rpgcore} which define a number of useful functions and classes, and a flexible set of \texttt{themes} which control how those commands are rendered in the final document.

			\vspace{2cm}
			\subsection*{Attribution \& License}

			This package would not have been possible without the team who developed \forcelink{https://github.com/rpgtex/DND-5e-LaTeX-Template/tree/dev}{its predecessor, the `DND 5e LateX Template'}. That code was released under an MIT license, the text of which can be found in the LICENSE file. \texttt{rpgtex} is released under the same license.
			}
		\end{center}

		\vfill{}
		\twocolumn{}

	\tableofcontents
	\mainmatter{}


		
	\part{rpgtex Core}
		\chapter{Installation \& Usage}

	\section{Getting \rpgtex}

		There are a number of different ways to acquire \rpgtex{}. Once you have installed it, it is vital to ensure that it is \href{\ref{S:Configuration}}{properly configured} (see below).


		\subsection{texmf Installation}

			The simplest way to use \rpgtex{} is to install it on the \texttt{texmf} path, where the compiler can automatically find it:

			\begin{lstlisting}
git clone https://github.com/DrFraserGovil/rpgtex.git "$(kpsewhich -var-value TEXMFHOME)/tex/latex/rpgtex"
			\end{lstlisting}

			This will clone the repository into your\LaTeX{} path.

		\subsection{Indirect Installation}

			If you want to tinker with \rpgtex{} -- such as by creating a new theme -- it is helpful to have it in a more accessible location. Clone the repository into a location of your choice:

			\begin{lstlisting}
git clone https://github.com/DrFraserGovil/rpgtex.git ~/your/rpgtex/directory
			\end{lstlisting}

			You then have two options to make the package visible to the compiler:

			\subsubsection{Use TEXINPUTS}

			Setting the environment variable \verb|TEXINPUTS| allows the compiler access:
				\begin{lstlisting}
	TEXINPUTS=~/your/rpgtex/directory/::
				\end{lstlisting}
				(Or similar commands, depending on your shell -- in \texttt{fish} you would call \verb|set TEXINPUTS dir|).

			\subsubsection{Use Symlinks}

			You can symlink the install location to the texmf directory, allowing the compiler to act as if you had performed the texmf installation:

			\begin{lstlisting}
				ln -sf ~/your/rpgtex/directory "$(kpsewhich -var-value TEXMFHOME)/tex/latex/rpgtex"
			\end{lstlisting}

		\subsection{Overleaf (Not recommended!)}

			We do not recommend using Overleaf since the free-tier subscription has reduced compilation times drastically, making compiling documents using complex packages such as this one extremely difficult. Nevertheless:

			\begin{enumerate}
				\item  Download this GitHub repository as a ZIP archive using the Clone or download link above.
    			\item On Overleaf, click the New Project button and select Upload Project. Upload the ZIP archive you downloaded from this repository.
				\item Manually create the file \texttt{rpg-config.cfg} with the contents ``\verb|\edef\RpgPackagePath{../}|''. This replaces the configuration step described below.
			\end{enumerate}


	\section{Configuring \rpgtex{}}\label{S:Configuration}

		Wherever one installs \rpgtex{} from, it is vital that it is properly configured. From within the \rpgtex{}-root directory, call:

		\begin{lstlisting}
			./configure
		\end{lstlisting}
		Or -- if one is (reasonably!) wary about running arbitrary executables -- manually create the relevant file:
		\begin{lstlisting}
			cd <rpgtex root directory>
			cmd="\edef\RpgPackagePath{$(pwd)}"
			echo $cmd >> core/rpg-config.cfg
		\end{lstlisting}

		\begin{RpgTip}{Why is configuration necessary?}
			\TeX{} is generally set up so that when a file calls \verb|include| or \verb|input| it is possible to use filepaths relative to the package itself. \texttt{rpg.sty} can call \verb|%%Set the D&D specific fonts

\sys_if_engine_xetex:T
{
  % Spit out a warning if XeLaTeX is detected for the PDFLaTeX-based D&D theme
  \msg_new:nnn { rpg } { dnd-xelatex-warning }{#1}
%   { #1 }
  \msg_warning:nnx { rpg } { dnd-xelatex-warning }
    {
      The~dnd~theme~is~designed~for~pdfLaTeX.~Compiling~with~XeLaTeX~may~cause
      font~and~character~issues~due~to~reliance~on~the~legacy~NFSS~system.
    }
}


\RequirePackage{Royal}
\RequirePackage[T1]{fontenc} %%Even in xelatex, this allows T1 fonts to be loaded in
\RequirePackage{bookman} %% Sets the default font to bookman
\RequirePackage[type1]{gillius2} %%loads in the gilliustwo font
\RequirePackage[notext,nomath,nott]{kpfonts}

\newcommand\jkfont{\fontfamily{jkpss}\selectfont}


\newcommand\titlemode{\linespread{.9} \color{titlered} \scshape}

\RpgSetFont[
	%Sections
	part-style			= {\color{titlered}\Huge\scshape},
	chapter-style		= {\titlemode \Huge},
	section-style		= {\titlemode \huge},,
	subsection-style	= {\titlemode \Large},,
	subsubsection-style	= {\titlemode \large},,
	paragraph-style	= {\bfseries \slshape},
	subparagraph-style	= {\bfseries \slshape},
	%Tables
	table-title-family	= {\jkfont},
	table-title-style	= {\bfseries\large\scshape},
	table-header-family	= {\jkfont},
	table-body-family	= {\gilliustwo},
	%tip boxes
	tip-title-family= {\jkfont},
	tip-title-style	= {\bfseries\scshape},
	tip-body-family	= {\gilliustwo},
	tip-body-style	= {\small},
	%sidebars
	sidebar-title-family= {\jkfont},
	sidebar-title-style	= {\bfseries\scshape},
	sidebar-body-family	= {\gilliustwo},
	sidebar-body-style	= {\small},
	%narration
	narration-family	= {\gilliustwo},
	narration-style		= {\small},
	% TOC
	toc-part-style		= {\Large \scshape \color{titlered}},
	toc-chapter-style	= {\large \scshape \color{titlered}},
	toc-section-style	= {\normalsize},
	%stat block
	stat-block-title-style= {\bfseries\scshape \LARGE},
	stat-block-body-family={\gilliustwo},
	stat-block-body-style={\small},
	stat-block-section-family={\jkfont},
	stat-block-section-style={\color{titlered} \scshape \large},
	%%misc
	footer-style		= {\textcolor{pagegold}\scriptsize},
	page-number-style	= {\textcolor{pagegold}\scriptsize},
	drop-cap-family		= {\Royal},
]{}
| and it will know to first check for the file relative to rpg.sty; even if the package resides within the texmf path and the user has no idea where \texttt{rpgroot/rpg.sty}, or \texttt{rpgroot/core/font.sty}, are.

			An annoying exception to this is fonts and typefaces. \texttt{xelatex} searches for fonts based on \textit{filepaths relative to the current working directory} -- or from those installed in as system fonts.

			Since \texttt{rpgtex} includes several (license free) typefaces as part of the provided themes, this poses a problem. We must either require that:
			\begin{enumerate}
				\item \rpgtex{} documents can only be prepared in restricted locations relative to the install location of \rpgtex{}.
				\item Users must identify and specify the \rpgtex{} root path when preparing a document
				\item Users must install the provided fonts to the system path
				\item \rpgtex{} must be configured to know `where it is', and so provide an absolute filepath to the internal fonts.
			\end{enumerate}
			The Configuration step is the most portable and easiest-to-use of these options.
		\end{RpgTip}

		Without a \texttt{core/rpg-config.cfg} file, any document which includes \rpgtex{} will fail to compile.
	\newpage

	\section{Package \& Class Usage}

		\rpgtex{} can be used either as a standalone package, or as part of a number of classes

		\subsection{Standalone Package}
			The standalone package can be used directly by including the \rpgtex{} package:
			\begin{lstlisting}
				\documentclass{arbitrary-class}

				\usepackage[options]{rpgtex}

				\begin{document}
				....
			\end{lstlisting}

			This will load only the core commands into the document, and (unless called explicitly) no themes will be imported. Using the package in this way does not activate any of the commands which change the overall geometry, background or headers of the document.

		\subsection{Classes}

			\rpgtex{} can also be loaded through a number of classes which drastically alter the appearance of the document, defining new geometries backgrounds and adding headers.

			The provided classes are:
			\begin{enumerate}
				\item \texttt{rpgbook} (\RpgPage{S:bookClass}). Based on the standard book class, this is designed for larger RPG documents.
				\item \texttt{rpghandout} (\RpgPage{S:handoutClass}). Based on the article class, this is designed for shorter documents
				\item \texttt{rpgcard} (\RpgPage{S:cardClass}). A small-document class designed for creating modular `handout' cards for items, spells or abilities.
			\end{enumerate}
	\section{Compiling}

		\rpgtex{} uses the \texttt{fontspec} package to allow custom fonts, and therefore requires compiling with \texttt{xelatex} or \texttt{luatex}:

		\begin{lstlisting}
			xelatex main.tex		#works
			luatex main.tex			#works
			pdflatex main.tex #fails
		\end{lstlisting}

		
		\onecolumn
		
	

		


\chapter{Commands \& Macros}
	\def\backendCommand{\hyperref[S:ThemeCommands]{\textcolor{blue!40!black}{{Backend Command}}}}
		\def\placeholderCommand{\hyperref[S:ThemeCommands]{\textcolor{blue!40!black}{{Placeholder Command}}}}
	\begin{RpgSidebar}{Theme Commands}\label{S:ThemeCommands}

		Several commands in this documentation are described as \textbf{Theme Commands}. These are commands that the user is \textit{not expected to call}, but which are executed by the internal engine in the process of rendering the page, or as a result of other commands that the user has called. 

		\begin{center}
		\large \textbf{A user who wishes to simply write documents using an unmodified \rpgtex{} need only concern themselves with the User-Facing Commands}. 
		\end{center}
		On the other hand, these Theme Commands have been designed to provide a convenient interface for creating custom Themes -- and so their documentation allows for designers to create powerful and flexible themes from within \rpgtex{}. 
		
		Theme Commands can be split into two groups:
		
		\begin{enumerate}
			\item \textbf{Backend Commands} These are commands which are executed within a theme (or a class) to modify internal values, such as fonts and colors. A designer interacts with these commands by calling them.
			\item \textbf{Placeholder Commands} These are virtual commands which are designed to be overwritten with completely custom code, which is executed when the core engine runs the command. A user interacts with these commands by redefining them  (usually with \verb|RenewDocumentCommand|).
		\end{enumerate}

		A `theme' is therefore a collection of Backend Commands (to configure the `core engine') and redefinitions of Placeholder Commands to provide their own unique functionality.
		
	\end{RpgSidebar}
	\section{Title Pages}
		\subsection{User-Facing Commands}
			\begin{macrolist}
				\RpgMacro*[maketitle]{\maketitle,{{}}}
					{	
						When called, creates theme-defined title pages using a custom format.
					}{
						\verb|\title{A title}|

						\verb|\subtitle{The subtitle}| (optional)
						
						\verb|\cover{path/to/image}| (optional)
						
						\verb|\author{Dr. W. Riter} | (optional)
						
						\verb|\begin{document}|
						
						\verb|	\maketitle|
						
						\verb|	....|
						
						\verb|\end{document}|
					}{
						Calls either \verb|\RpgDrawCover| or \verb|\RpgSimpleTitle| depending on the value passed to \verb|RpgUseCoverPage|.

						If \verb|RpgUseCoverPage| has been set to true (usually by a class such as \verb|rpgbook.cls|), then the image stored in \verb|\@cover| (if there is one) is automatically used as a full-page background image. This is independent of the theme definition of \verb|RpgDrawCover|, and occurs before that function is called -- all subsequent drawing occurs over the top of the cover image.
					}
				\RpgMacro[cover]{\cover,{{m}},\@cover}
				{Saves an image path to the variable \verb|\@cover|, automatically used by \verb|\maketitle| as the background image.
				\cmdidx{{"@}cover}
					}
					{
						\cover{path/to/cover_image}				
					}
					{
						If \verb|RpgUseCoverPage| has been set to true, then the image at this path will be used as a full-page image in the background of the page created by \verb|maketitle|.

						The default value is empty (\verb|\cover{}|).
					}
				\RpgMacro[subtitle]{\subtitle,{{m}},\@subtitle}
					{
						Saves a string to the variable \verb|\@subtitle|. Themes may use this when defining their \verb|RpgDrawCover| and \verb|RpgSimpleTitle|.
						\cmdidx{{"@}subtitle}
					}
					{
						\subtitle{<string>}
					}
					{
						This command has no effect on its own (unlike \verb|cover| which is automatically included in the background).

						The default value is empty (\verb|\subtitle{}|).
					}
			\end{macrolist}
		\subsection{Theme Commands}
		\begin{macrolist}
			\RpgMacro[RpgUseCoverPage]{\RpgUseCoverPage,{{m}}}
				{
					If true, \verb|maketitle| creates a title page to populate, else the title is rendered as a heading.
				}{
					\RpgUseCoverPage{true/false}
				}
				{
					This is a \backendCommand{}. When true, \verb|maketitle| attempts to use \verb|\@cover| and then calls \verb|RpgDrawCover|. If false, it calls \verb|RpgSimpleTitle|.
				}
			\RpgMacro[RpgDrawCover]{\RpgDrawCover,{{}}}
				{
					Executes over the top of the \verb|\@cover| image to render a front cover.
				}{}
				{
					This is a \placeholderCommand{}, used by themes to customise the appearance of the title page which appears in \verb|rpgbook| class. The default value renders a single node at the centre of the page containing \verb|\@title|, \verb|\@subtitle|, \verb|\@author| and \verb|\@date| variables in the centre. More advanced themes (such as dnd or scifi) add decorative embellishments and place the text at custom locations.

					This command is executed by \verb|maketitle| if \verb|\RpgUseCoverPage{true}| has been set by the theme, class or directly by the user. 
					The command is called from within an existing tikz environment with the \verb|remember,overlay| options active, allowing for page coordinates (i.e. current page.north) to be used.

					If a \verb|\@cover| has been defined, this command is executed after the image is placed, drawing on top of it.
				}
			\RpgMacro[RpgSimpleTitle]{\RpgSimpleTitle,{{}}}
				{
					Renders a `header' title - a simple text-only title at the top of the page.
				}{}
				{
					This is a \placeholderCommand{}, used by themes to customise the appearance of the title header which appears in \verb|rpghandout| class. The default value places the title, subtitle and author at the top of the page. More advanced themes (such as dnd or scifi) add decorative embellishments and place the text at custom locations.

					The Simple Title is configured so that, in a twocolumn document, it occupies the full page width; calling \verb|centering| with the simple title therefore centers the text above both columns. 
				}
		\end{macrolist}

	\section{Part Pages}
		\begin{macrolist}
			\RpgMacro[part]{\part,{\part*},{{o m}}}
				{
					Defines a wrapper around the standard \verb|part| command that allows for tikz-based custom page formatting
				}
				{
					\part(*)[<image>]{<part-name>}
				}
				{
					\cmdidx{part,part*}
					There are three distinct behaviours that can be exhibited, depending on the presence or absence of the \verb|*|, and the presence and value of \verb|<image>|.

					\begin{RpgTable}{XX}
						Command & Behaviour
						\\
						\parbox[t]{6cm}{\verb|\part*{partname}| \\ \verb|\part*[<any text>]{partname}| \\ \verb|\part[none]{partname}|} & Uses original \verb|part| command defined by underlying class. 
						\\
						\verb|\part{partname}| & Calls \verb|RpgDrawPartPage| on a blank background.
						\\
						\parbox[t]{6cm}{\verb|\part[path/to/image]{partname}||} & Places the corresponding image as a full-page background, and then calls \verb|RpgDrawPartPage|.
					\end{RpgTable}
					\verb|RpgDrawPartPage| \RpgPage[p]{Macro:RpgDrawPartPage} is a Theme Function, which executes a series of tikz functions to place the part title according to the theme specifications. 
				}
			
			\RpgMacro[RpgDrawPartPage]{\RpgDrawPartPage,{{m}}}{Uses Tikz to draw a custom part page when activated by \verb|\part| \RpgPage[p]{Macro:part}. 
				}
				{
					\RpgDrawPartPage{<part title>}
				}
				{
					This is a \placeholderCommand{}, allowing the designed to determine where to place the part name on the page, and what embellishments accompany it. The command is called from within an existing tikz environment with the \verb|remember,overlay| options active, allowing for page coordinates (i.e. current page.north) to be used.

					The default \verb|part| command allows a user to specify a background image for their part page -- it is not necessary to provide one within the drawing command.
				}
			
		\end{macrolist}
	
	\section{Dice Commands}
		Dice are a mainstay of RPGs, and so it is important to have a standard way to report and simplify their expressions. We provide an interface for a standard `dice + modifier' expression.
		\begin{macrolist}
			\RpgMacro[RpgDice]{\RpgDice,{{m}}}
				{
					Evaluates expressions of the form $n\mathrm{d}x \pm m$, and outputs using a theme-dependent layout.
				}
				{
					\RpgDice{<dice-expression>}
				}{
					Uses regular expressions to extract and simplify the \verb|dice-expression|, which must follow the following format:
					\begin{RpgSidebar}{Dice format}
						\begin{multicols}{2}
						\begin{enumerate}
							\item It must contain either `d' or `D' (the `dice symbol')
							\item The dice symbol must be immediately followed by a single number (the `dice size')
							\item The dice symbol may optionally be prefixed by a single number (the `dice count')
							\item The first (non-whitespace) character must be either the dice count (if present) or the dice symbol
							\item The dice size must be followed by either a `+', '-', or the end of the expression.
							\item After this, any number of standard numeric expressions may follow. This expression will be evaluated into a single `modifier'.
						\end{enumerate} 
						\end{multicols}
					\end{RpgSidebar}
					The dice ignores any whitespace before the beginning of the expression, and arbitrary whitespace within the `modifier' part of the exprssion.  
					\begin{RpgTable}{XX}
						Example & Output \\
						\tabverbExample{\RpgDice{  1d6-2}}
						\tabverbExample{\RpgDice{2D6 + 3*2^2}}
						\tabverbExample{\RpgDice{1d16}}
						\tabverbExample{\RpgDice{d8-3}}
						\verb|\RpgDice{2*1d6}|, \verb|\RpgDice{1 d6}|, \verb|\RpgDice{3d 6 +3}| & (Fails to compile)
					\end{RpgTable}
				
					\verb|RpgDice| is neat, but not necessarily impressive by itself. The true power of the expression is that it calls \verb|RpgDiceFormat| to perform the output formatting (after performing the regular expression parsing), allowing designers to customise their dice formatting.

					\verb|RpgDice| is loaded in both \verb|layout| and non-\verb|layout| calls.
				}
		
			\RpgMacro[RpgDiceFormat]{\RpgDiceFormat,{{m m m}}}
				{Prints the values computed by \verb|RpgDice|
				}
				{
					\RpgDiceFormat{<dice-count>}{<dice-size>}{<added bonus>}
				}
				{
					This is a \placeholderCommand{}, used by theme designers to determine how \verb|RpgDice| is rendered. The default option is:
					\verb|\RpgDiceFormat{m m m} { #1d#2 #3}|, such that \verb|RpgDice{ndx + a + b}| gives ``ndx + c'', where c is the numerical value of a+b, with an additional check to see if \verb|#3| is equal to 0, in which case it is not printed (so as not to `1d6 + 0'). 
					
					The dnd implementation performs a more advanced operation, computing the average value of the roll, and formatting that first, to replicate the format used by monster stat blocks. 

					\RenewDocumentCommand{\RpgDiceFormat}{m m m}{\DndTempDiceFormat{#1}{#2}{#3}}

					
					\begin{RpgTable}{XX}
						Example (with \verb|\RpgSetTheme{dnd}|) & Output \\
						\tabverbExample{\RpgDice{1d6-2}}
						\tabverbExample{\RpgDice{2D6 + 3*2^2}}
						\tabverbExample{\RpgDice{1d12}}
						\tabverbExample{\RpgDice{d83-3}}
					\end{RpgTable}

				
				}
		\end{macrolist}

		
	\section{Theme Commands}

		\begin{macrolist}
			\RpgMacro[RpgLayoutOnly]{\RpgLayoutOnly,{{m}}}
				{\cmdidx{RpgLayoutOnly}
					Executes the contents of the command if \verb|layout| mode is active.
				}
				{
					\RpgLayoutOnly{<content-to-execute>}
				}
				{
					If the internal value \texttt{\textbackslash{}l\_\_rpg\_layout\_bool} is True, then \verb|content-to-execute| is run, otherwise it is ignored.

					This command is primarily used by theme developers and document class files to conditionally load or activate modules based on whether the package was loaded via a document class (layout mode active) or directly via \verb|\usepackage{rpgtex}|.
				}
			\RpgMacro{\RpgSetFont}{See \RpgPage{S:FontInterface}}{}{}
			\RpgMacro{\RpgSetFooterDecoration,{{o m}}}{Configures an image to be displayed along the bottom of a page as a `footer scroll'.}
				{
					\RpgSetFooterDecortation[<opts>]{path/to/img}
				}
				{
					When placed within a footer, (i.e. with fancypage), places the image in a node with parameters:
					
					\texttt{\textbackslash{}node[inner sep=0pt,anchor=south,nearly opaque] at (current page.south) \{\textbackslash{}includegraphics[width=\textbackslash{}paperwidth]\{path/to/img\}\};}

					If the package option \verb|bg=none| has been passed, then the image is suppressed.

					The following options modify that code as follows:
					\begin{description}
						\item[reverse] adds \verb|xscale=-1| to the node arguments, reversing the image (useful for right/left page differences)
						\item[tikz-insert={code}] inserts the code within the tikz environment after the footer scroll. This is not suppressed with \verb|bg=none| and can be used to place chaptermarks / page numbers more precisely than the standard interface allows.
						\item[height=<dimexpr>] adds \verb|height=dimexpr| to the includegraphics arguments
						\item[keepaspectratio] adds \verb|keepaspectratio| to the includegraphics arguments    
					\end{description}
				}
			\RpgMacro[RpgSetPaper]{\RpgSetPaper}
				{Sets a background image to be used as the `paper' image.}
				{
					\RpgSetPaper{path/to/image}
				}
				{
					If \verb|layout| mode is active, then this configures \rpgtex{} to use the image as the `background image' of every page with \verb|fancy, plain| or \verb|clear| pagestyle. This allows for custom `paper textures' to be loaded in in the background. 

					The pagestyle \verb|clear| is equal to \verb|empty|, with the exception of the page texture.
				}
			\RpgMacro[RpgSetTheme]{\RpgSetTheme,{{m}}}
				{
					Activates a chosen theme.
				}
				{
					\RpgSetTheme{<theme-name>}
				}
				{
					Searches for the file \verb|<theme-path>/<theme-name>/<theme-name>.cfg|, and inputs it. If this is a properly configured theme file, then it activates the chosen theme given the current global parameters. If the file does not exist, throws an error.

					If   \texttt{\textbackslash{}l\_\_rpg\_layout\_bool} is True, the command automatically inserts \verb|\clearpage|, as required to ensure the old headers are not overwritten by the new theme.

					\verb|<theme-path>| is modified via \verb|RpgSetThemePath|.
				}

			\RpgMacro[RpgSetThemeColor]{\RpgSetThemeColor,{{m}}}
				{
					Sets the \verb|themecolor|, and simultaneously updates the co-varying colors \RpgPage[p]{S:Colors}.
				}{
					\RpgSetThemeColor{color-name}
				}{
					If \verb|color-name| specifies a valid color, then the value of \verb|themecolor| is updated, as well as a number of other colors (\verb|tipcolor|, \verb|sidebarcolor| and \verb|tablecolor|) which are set to be equal to the themecolor by default.

					Of the rpg-provided colors, only \verb|narrationcolor| is unaffected by this command.
				}
			\RpgMacro[RpgSetThemePath]{\RpgSetThemePath,{{m}}}
				{
					Changes the value of the theme path searched for by \verb|RpgSetTheme|
				}
				{
					\RpgSetTheme{<path-name>}
				}
				{
					Updates an internal variable to be equal to the input value; does not check if the theme path is valid or not. Useful if you wish to create a new theme outside of the \verb|rpgtex| file structure.
				}
			
		\end{macrolist}
	\section{Utility Commands}

		\begin{macrolist}
			\RpgMacro{\RpgOrdinal,{{o m}}}
				{
					Converts a numeric value to the corresponding ordinal.
					\cmdidx{RpgOrginal}
				}
				{
					\RpgOrdinal[<command>]{<count>}
				}
				{
					The command outputs the \verb|count| followed by the english abbreviations for the corresponding ordinal. The optional \verb|command| argument is inserted between the numeral and the suffix, allowing for the customisation of appearances.
					\begin{RpgTable}{XX}
						Example & Output \\
						\tabverbExample{\RpgOrdinal{1}}
						\tabverbExample{\RpgOrdinal{2}}
						\tabverbExample{\RpgOrdinal{13}}
						\tabverbExample{\RpgOrdinal[\textsuperscript]{7}}
						\tabverbExample{\RpgOrdinal[\textbf]{133}}
						\tabverbExample{\RpgOrdinal[<arbitrary text>]{133}}
					\end{RpgTable}
					{\it Note that due to a lack of brace-capturing, it is not possible to chain multiple commands.}.

				}
			\RpgMacro[RpgPage]{\RpgPage,{{O{t} m}}}
				{
					Outputs the current page reference for a label, with an option to enclose it in specific brackets or parentheses.
				}
				{
					\RpgPage[t/p/b/c]{<label-reference>}
				}
				{
					The optional arguments wrapping of the main reference. The options are:
					\begin{description}
						\item[t (default)] No wrapping
						\item[p] (parentheses)
						\item[b] [square brackets]
						\item[c] \{curly braces\}
					\end{description}
					An invalid input resolves to \verb|?page~\pageref{<ref}?|.\label{example:current page}
					
					\begin{RpgTable}{XX}
						Example & Output \\
						\tabverbExample{\RpgPage{example:current page}}
						\tabverbExample{\RpgPage[p]{example:current page}}
						\tabverbExample{\RpgPage[b]{example:current page}}
						\tabverbExample{\RpgPage[c]{example:current page}}
						\tabverbExample{\RpgPage[(error)]{example:current page}}
					\end{RpgTable}
				}
			\RpgMacro{\RpgPlural,{{o m m}}}
				{
					Generates grammatically correct plural forms of a word based on a given count.
					\cmdidx{RpgPlural}
				}
				{
					\RpgPlural[<custom-plural>]{count}{<text>}
				}
				{
					The command outputs the count followed by the value of \verb|<text>|. For a count of 1, the command then finishes. For any other count, it appends an ``s'', pluralizing the text.

					The optional argument \verb|[<custom-plural>]| overrides the default logic, allowing for irregular plurals.


					\begin{RpgTable}{XX}
						Example & Output \\
						\tabverbExample{\RpgPlural{1}{hat}}
						\tabverbExample{\RpgPlural{2}{hat}}
						\tabverbExample{\RpgPlural[octopodes]{1}{octopus}}
						\tabverbExample{\RpgPlural[octopodes]{359}{octopus}}
					\end{RpgTable}
				}

			
		\end{macrolist}


		\chapter{Environments}
	\def\tcref{\forcelink{https://ctan.org/pkg/tcolorbox?lang=en}{tcolorbox}}
	\newcommand\labelsection[1]
	{
		\section{#1}\label{S:#1}
	}

	
	\labelsection{Rpg Boxes}

		\rpgtex{} defines three `colorbox' environments, which inherit from \tcref{}.

		\begin{macrolist}
			\RpgMacro*[RpgNarration]{RpgNarration,{{o}}}{A \tcref{} wrapper designed for text that is read aloud to players}{
				\verb|\begin{RpgNarration}[color=<color>,<tcbox-options>]|

				~~~~\verb|<text>|
				
				\verb|\end{RpgNarration}|
			}{
				RpgNarration does not (by default) set a title, using only `body text', which is typeset using the \verb|RpgFontNarration| font. The optional \verb|<tcbox-options>| argument can be a list of all the basic tcolorbox options (see that documentation). The \verb|color| argument is an alias for \verb|colback| (\verb|colbacktitle| is also set, but is ignored as the title is empty). Due to the order of processing, if both \verb|color| and \verb|colback| are set, the value of \verb|colback| is used.

				Themes may alter the appearance of the narration block using the tcb interface, calling \verb|\tcbset{rpgnarration /.append~style={...}}| to overwrite the existing instructions.
			}
			\RpgMacro*[RpgSidebar]{RpgSidebar,{{o m}}}{A decorated \tcref{} wrapper designed for information which is set outside the main text.}{
				\verb|\begin{RpgSidebar}[color=<color>,<tcbox-options>]{<title>}|

				~~~~\verb|<text>|
				
				\verb|\end{RpgSidebar}|
			}{
				RpgSidebar requires a title (using \verb|RpgFontSidebarTitle|) as well as the body text (\verb|RpgFontSidebarBody|). 	RpgSidebar is typically more highly decorated than RpgTip, and does not have the \verb|breakable| flag set. It is usually best to use one of the `float' options.
				
				The optional \verb|<tcbox-options>| argument can be a list of all the basic tcolorbox options (see that documentation). The \verb|color=x| argument is equivalent to calling both \verb|colback=x| and \verb|colbacktitle=x|. Due to the order of processing, if both \verb|color| and \verb|colback| are set, the value of \verb|colback| is used.
		 

				Themes may alter the appearance of the sidebar using the tcb interface, calling \verb|\tcbset{rpgsidebar /.append~style={...}}| to overwrite the existing instructions.
			}
			\RpgMacro*[RpgTip]{RpgTip,{{o m}}}{A simple \tcref{} wrapper designed for information which is set outside the main text.}{
				\verb|\begin{RpgTip}[color=<color>,<tcbox-options>]{<title>}|

				~~~~\verb|<text>|
				
				\verb|\end{RpgTip}|
			}{
				RpgTip is similar to RpgSidebar, requiring a title (\verb|RpgFontTipTitle|) in addition to the body text (\verb|RpgFontTipBody|). However, it is generally simpler, enabling it to safely break over page boundaries. The optional \verb|<tcbox-options>| argument can be a list of all the basic tcolorbox options (see that documentation). The \verb|color=x| argument is equivalent to calling both \verb|colback=x| and \verb|colbacktitle=x|. Due to the order of processing, if both \verb|color| and \verb|colback| are set, the value of \verb|colback| is used.

				Themes may alter the appearance of the narration block using the tcb interface, calling \verb|\tcbset{rpgnarration /.append~style={...}}| to overwrite the existing instructions.
			}
		\end{macrolist}

		

		\subsection{Which Colorbox To Use?}

			The choice between RpgSidebar and RpgTip is somewhat arbitrary -- although they have a mechanical difference by default (one being breakable, the other not) -- this can be overridden by themes. Instead, the intention is that they serve slightly different purposes:

			\begin{description}
				\item[RpgSidebar] is used for `important information' -- key rules or summaries which readers \textit{should} pay attention to.
				\item[RpgTip] is for `helpful additions' -- tips, tricks and trivia that are not necessary, but which might be useful, and are too big to fit into a footnote or parenthetical.   
			\end{description}
		\subsection{Colorbox Examples}
		\DualExampleRender{narration1}
	
	\labelsection{RpgTable}



		\begin{macrolist}
			\RpgMacro*[RpgTable]{RpgTable,{{o m}}}{Begins an environment for creating visually appealing and consistent tables. }{
				\verb|\begin{RpgTable}[<options>]{<column-specifications}|

				~~~~\verb|<table-contents>|
				
				\verb|\end{RpgTable}|
			}{
				RpgTable is a wrapper for the \forcelink{https://ctan.org/pkg/tabularx}{tabularx} (or \forcelink{https://ctan.org/pkg/xltabular}{xltabular} -- see \verb|breakable|) environment, and so accepts the standard set of column specifications:\{c,l,r,p{width},\@,\ldots{}\} and the extended set (i.e. X). It therefore acts almost identically to the standard tabular environment with a few stylistic differences.
				\macrosection{Stylistic Changes}
					The RpgTable environment makes the following changes:
					\begin{enumerate}
						\item \textbf{Title.} If the \verb|title| option is set, a title-heading is rendered above the tablular in the \verb|RpgFontTableTitle| font.
						\item \textbf{Auto-headings.} The first row of the tabular environment is automatically rendered in the \verb|RpgFontTableHeader|, allowing for trivial header labels.
						\item \textbf{Font Integration.} The main body of the table is rendered in \verb|RpgFontTableBody| font.
						\item \textbf{Auto-colouring.} The rows alternate between being transparent and being set to the \verb|tablecolor| variable \RpgPage[p]{S:Colors}. This is powered by \verb|rowcolors|.
					\end{enumerate}
				\macrosection{Optional Arguments}
				\begin{description}
					\item[width=<dimexpr>] Fixes the width of the tabular environment to the value of this argument. Default value is the current \verb|\linewidth|.
					\item[color=<color-name>] If set, uses this value instead of \verb|tablecolor| for the alternating coloration.
					\item[title=<text>] Sets the text to be rendered as the title of the table. 
					\item[breakable] If flag is present, renders using \verb|xltabular|, enabling the table to break over pages. \textbf{only available in 1-column mode (a fundamental limitation of xltabular).}
					\item[noheader] If flag is present, suppresses the autoformatting of the title. The first row is instead rendered in the body formatting.
				\end{description}
				% The available options are as follows:
			}
		\end{macrolist}
		\subsection{RpgTable Example}

		This is the standard usage of the table, showing automatic formatting of the header rows and the word-wrapping abilities of the X-column:

		\DualExampleRender{table1}

		This example adds a title, but suppresses the header formatting:

		\DualExampleRender{table2}
		


		\chapter{Variables}
	\section{Colo(u)rs}\label{S:Colors}

		\rpgtex{} by default defines five colors\footnote{Yes, I hate myself, but we're going with the code-based spelling.} which are used for different elements:
		\begin{description}
			\item[themecolor] A `basic color' which is (by default) equal to the following three colors:
			\begin{enumerate}
				\item \textbf{sidebarcolor} The background color of the \verb|RpgSidebar| environment
				\item \textbf{tablecolor} The background color of every other row in an \verb|RpgTable| 
				\item \textbf{tipcolor} The background color of the \verb|RpgTip| environment 
			\end{enumerate} 
			\item[narrationcolor]  The background color of the \verb|RpgTip| environment 
		\end{description}

		Calling \verb|\RpgSetThemeColor| \RpgPage[p]{Macro:SetThemeColor} updates the value of \verb|themecolor|, as well as the three `co-varying' colors (i.e. everything except \verb|narrationcolor|). When \verb|printmode| is active \verb|\RpgSetThemeColor{white}| is called, making environments transparent.

		\chapter{Fonts}

	\rpgtex{} allows for a high degree of customisation of the fonts and typefaces used for the elements within a document. The key interface for the designer is the \verb|\RpgSetFont| command, which accepts a wide variety of key-value inputs, detailed below. 

	
	\section{Font Interface}\label{S:FontInterface}

		\begin{macrolist}
			\RpgMacro[RpgSetFont]{\RpgSetFont,{{m}}}
			{
				Saves new font values and styles to the internal RpgFont[X] variables, which are then available for themes to use.			
			}
			{
				\RpgSetFont{<key-value-pairs>}
			}{
				Note that this interface \textbf{does not automatically change all fonts}. The SetFont-interface saves values to internal variables which populate the corresponding \verb|RpgFont[X]| macros. It does not invoke \verb|fontspec| and does not automatically assign fonts to the designated elements. \textit{However}, a theme designer may then use the \verb|RpgFont[X]| macros within their commands, thereby assigning fonts to the relevant elements. In this case (or if a user manually invokes a font), this command will act to update the font.

				To summarise: if a writer uses a theme which does not make use of the \verb|RpgFontSection| font, then calling \verb|RpgSetFont{section-style=\it}| will have no effect. They would need to set \verb|\titleformat {\section}{ \RpgFontSection }{}{}| (titlesec is loaded by default), in which case calling \verb|RpgSetFont| would change the font for all subsequent calls to \verb|\section|.
				
				The exception to this is the main-body font, which is achieved by updating \verb|\normalfont|. 
			}
		\end{macrolist}

		\subsection{Defining Fonts}

			The arguments passed to the `style' can be any form of latex formatting (i.e. \verb|\slshape\scriptsize\bfseries|, and so on). To update the typeface, however, you must define a font family:
			\begin{lstlisting}
				\newfontfamily{\myfont}{arial}
				\RpgSetFont{main-body-family=\myfont}
			\end{lstlisting}

			For custom typefaces - or where you wish to `mix and match' typefaces in different modes, you can use the full power of the \forcelink{ctan.org/pkg/fontspec?lang=en}{fontspec package}:
			\begin{lstlisting}
				\newfontfamily{\myfont}{custom-font}[
					Path=/path/to/local/font.otf,
					ItalicFont=*-Bold,
					BoldFont=Arial,
				]
				\RpgSetFont{main-body-family=\myfont}
			\end{lstlisting}
			This would define a font which used a local .otf file for the main font, but the bold typeface when \verb|\textit{}| were called, and used arial as the mock `boldface'.

		\subsection{Font Elements}
			\rpgtex{} provides 28 Font Commands, each comprised of a \textit{family} and a \textit{style}. 
			\begin{RpgSidebar}{Family vs Style}
				When defining the Font for an element, the interface allows one to specify both a \texttt{family} and a \texttt{style}. Formally speaking, \texttt{family} defines the \textbf{typeface} used by the associated element, whilst the \texttt{style} determines the options passed to that typeface (bold, italics, size etc.).

				The distinction is largely irrelevant, as the construction of the final font object is often simply the concatenation of the two:
				\begin{lstlisting}
					\def\RpgFontX
					{
						\l__rpg_x_family \l__rpg_x_style
					}
				\end{lstlisting}
				The separate definitions is therefore largely a matter of clarity and readability. It is generally safe to place commands that should be in family into the style key, as long as it doesn't conflict with other styling.
				
				\vspace{1em}
				{\RpgFontSidebarTitle \noindent Font vs Implementation}

				We generally encourage designers to place all text visualisation within the relevant Font rather than elsewhere -- if all subsections are going to be in red, then define \verb|subsection-style=\color{red}|, rather than setting it within the titlesec specification (\verb|\titleformat {\subection}{\RpgFontSubsection\color{red}}....|).

				There will naturally be some exceptions to this: we found that the \verb|RpgTitleFont| colour we wanted within \verb|RpgDrawCover| diverged so strongly from that in \verb|RpgSimpleTitle| that it made sense to define a special colour when rendering over a background image.
			\end{RpgSidebar}

			These fonts are assigned to typesetting elements by the theme designer -- what we have intended to be the section font may, within a different theme, be used for a different element. This document outlines how we have used these elements in the provided themes, though this is not prescriptive.
			
			\begin{fontlist}
				\FontElement{\RpgFontBody}{main-body}{The main body text of the document, which \verb|RpgSetFont| sets equal to \verb|\normalfont|. 

				Updating the fontsize here (i.e. using \verb|\large|) can cause some counterintuitive results since it will \textit{only} update the body text, and not adjust the other elements relatively. Adjusting the font size for the entire document should be done in the documentclass declaration.
				}
				\FontElement{\RpgFontTitle}{title}{The font used for \verb|\@title| when \verb|\maketitle| is called.}
				\FontElement{\RpgFontSubtitle}{subtitle}{The font used for the value of \verb|\@subtitle| \RpgPage[p]{Macro:subtitle}, \verb|\@author| and \verb|\@date|  when \verb|\maketitle| is called.}
				\FontElement{\RpgFontPart}{part}{The font used when \verb|\part| is called.}
				\FontElement{\RpgFontTocPart}{toc-part}{The font used for a part in the table of contents}
				\FontElement{\RpgFontChapter}{chapter}{The font used when \verb|\chapter| is called.}
				\FontElement{\RpgFontTocChapter}{toc-chapter}{The font used for a chapter in the table of contents}
				\FontElement{\RpgFontSection}{section}{The font used when \verb|\section| is called.}
				\FontElement{\RpgFontTocSection}{toc-section}{The font used for a section in the table of contents}
				\FontElement{\RpgFontSubsection}{subsection}{The font used when \verb|\subsection| is called.}
				\FontElement{\RpgFontSubsubsection}{subsubsection}{The font used when \verb|\subsubsection| is called.}
				\FontElement{\RpgFontParagraph}{paragraph}{The font used when \verb|\paragraph| is called.}
				\FontElement{\RpgFontSubparagraph}{subparagraph}{The font used when \verb|\subparagraph| is called.}
				\FontElement{\RpgFontTableTitle}{table-title}{The font used for \verb|<text>| if \verb|\RpgTable| \RpgPage[p]{Macro:RpgTable} is called with the \verb|title=<text>| option.}
				\FontElement{\RpgFontTableHeader}{table-header}{The font used for the first row of a \verb|\RpgTable|.}
				\FontElement{\RpgFontTableBody}{table-body}{The font used for the text within an \verb|\RpgTable| after the first row.}
				\FontElement{\RpgFontTipTitle}{tip-title}{The font used for the title of an \verb|RpgTip| environment \RpgPage[p]{Macro:RpgTip}.}
				\FontElement{\RpgFontTipBody}{tip-body}{The font used for the body of an \verb|RpgTip| environment \RpgPage[p]{Macro:RpgTip}.}
				\FontElement{\RpgFontSidebarTitle}{siderbar-title}{The font used for the title of an \verb|RpgSidebar| environment \RpgPage[p]{Macro:RpgSidebar}.}
				\FontElement{\RpgFontSidebarBody}{sidebar-body}{The font used for the body of an \verb|RpgSidebar| environment \RpgPage[p]{Macro:RpgSidebar}.}
				\FontElement{\RpgFontNarration}{narration}{The font used for all (since they have no title) of an \verb|RpgNarration| environment \RpgPage[p]{Macro:RpgNarration}.}

				\FontElement{\RpgStatBlockTitle}{stat-block-title}{The font used for the title of a `statblock' environment - in the dnd theme this corresponds to the \verb|monster| environment.}
				\FontElement{\RpgStatBlockSection}{stat-block-section}{The font used for sections within a `statblock' environment (should one be defined).}
				\FontElement{\RpgStatBlockBody}{stat-block-body}{The font used for text within a `statblock' environment (should one be defined).}
				\FontElement{\RpgFontFooter}{footer}{The font used for the footer text}
				\FontElement{\RpgFontPageNumber}{page-number}{The font used for the page number within the footer}
				\FontElement{\RpgFontDropCap}{drop-cap}{The font used for the large drop-cap letter created by a \verb|RpgDropCap| (see below).}
				\FontElement{\RpgFontDropCapInternal}{drop-cap-internal}{The font used for the first line of text following the drop cap.}
			\end{fontlist}

			\newpage
	\section{Decorative Text}
		In addition to the fundamental typeface alterations \rpgtex{} includes a number of commands to turn text into decorative elements.
		\begin{macrolist}
			\RpgMacro{\RpgContour,{{O{} m}}}
				{	
					Renders text with a \RpgContour[inner=red,outer=black]{contour effect}. The color and style are set through key/value pairs.
					\cmdidx{RpgContour}
				}
				{
					\RpgContour[inner=<color>,outer=<color>,style=<code>]{<text>}
				}
				{
					The \texttt{style} command is applied to the text, whilst the optional \texttt{inner} and \texttt{outer} commands set the base text colour and the external contour color respectively. If the colors are not set, the default values are the \verb|contourinnercolor| and \verb|contouroutercolor| values defined by the theme \RpgPage[p]{S:Colors}.
					
					The contour does not automatically linebreak, but can be controlled manually with a \verb|\newline| command (not \verb|\\| or \texttt{\textbackslash{}par})
					
					\begin{RpgTable}{lX}
						Example & Output \\
						\tabverbExample{\RpgContour[inner=red,outer=black]{example}}
						\tabverbExample{\RpgContour[style=\Huge\it]{example}}
						\tabverbExample{\RpgContour[]{multi\newline line\newline example}}
					\end{RpgTable}					
					~\macrosection{Quirks}

					Due to the tokenisation required for the line-splitting and space-preservation, the text inside the contour can exhibit some quirks if stylisation is applied within the \verb|<text>| argument. 

					Unbraced commands (such as \verb|\it| or \verb|\footnotesize|) will only apply to the first word in the text. Braced commands \textit{can} work, but will cause a compilation error if a \verb|\newline| is included. 

					
					\begin{RpgTable}{lX}
						\footnotesize\tabverbExample{\RpgContour[]{\Huge\it only first word changes}}
						\footnotesize\tabverbExample{\RpgContour[]{\textit{all words change}}}
						\footnotesize\verb|\RpgContour[]{\textit{all word \newline change}}| & (fails to compile)
					\end{RpgTable}
					For robustness, we therefore recommend that all stylisation be applied through the \verb|style| command, which is applied to each tokenised element, and therefore guaranteed to work as expected.
				}
			\RpgMacro{\RpgDropCap,{{O{}, m m}}}
				{	
					Creates a decorative `drop cap' letter to begin a new chapter with, and modifies the following text.
					\cmdidx{RpgDropCap}
					}
					{
						\RpgDropCap[<lettrine-args>]{<letter>}{<text>}
				}{
					This command uses \forcelink{https://texdoc.org/serve/lettrine/0}{the lettrine package} and the \forcelink{https://ctan.org/pkg/magaz?lang=en}{magaz} package to create an easy-to-use environment in which the first letter is enlarged (and stylised in the \verb|RpgFontDropCap|\index{Font!DropCap} font). The second argument formats \textit{up to the first line} of text in the \verb|RpgFontDropCapInternal|\index{Font!DropCapInternal} font (usually a simple \verb|scshape| command).

					This command can be a little fragile -- lettrine does not usually play well with the `FirstLine' command provided by magaz -- and we've used a few workarounds to allow both linebreaking, and the formatting of only the first line of text. There may need to be a small amount of manual calibration, but it is better than the default.					

					\begin{RpgTable}{lX}
						Example & Output \\
						\tabverbExample{\RpgDropCap{A}{n example: \blindtext}}
					\end{RpgTable}
				}
		\end{macrolist}





		\chapter{rpglatex Compiler}\label{C:Compiler}


	\rpgtex{} is shipped with a special compiler, \verb|rpglatex|. This is simply a python3 script which acts as a wrapper around either xelatex or luatex, but includes several quality-of-life changes to the interface to make it easier to use with \verb|rpgtex|.

	\begin{macrolist}
		\RpgMacro[rpglatex]{rpglatex}{Compiles latex documents using either xelatex or luatex}
		{
			rpglatex [options] <file> 
		}{
			\verb|rpglatex| has the following features:

			\newcommand\feature[3]
			{
				\textbf{#1} & #2 & \texttt{#3} \\
			}
			\begin{RpgTable}[width=\linewidth]{lXl}
				Feature & Description & Options \\
				\feature{Compiler Selection}{The \verb|xelatex| compiler is selected by default, but the \verb|-l, --luatex| flags set it to use luatex instead.}{-l, --luatex}
				\feature{Build Directory}{Compilation files (.aux, .log etc.) are stored in a build directory. The default is \verb|.build| in the calling location, but can be changed with the \verb|-b| flag}{-b <build dir>}
				\feature{Output Naming}{The name of the output file can be changed from the default (equal to the input tex name)}{-o <output name>}
				\feature{Multi-pass Compiling}{By default, the compiler runs twice in a row to enable references and \texttt{tikz[remember]} commands to function. A full three-compilation suite (necessary for very complex or reference-heavy documents) can be activated with the \verb|-f, --full| flag}{-f, --full}
				\feature{Volume Control}{latex is notoriously noisy, producing copius output. By default, this is suppressed and only a summary is printed. The summary can be removed (rendering it completely silent) with the \verb|-q| command, or the original output recovered in verbose mode; \verb|-v|.
				
				These outputs are always overriden if a compilation error occurs, in which case the full trace is output to the console.
				}{-q, -v}
				\feature{Auto-bibtex}{If the \verb|-r| or \verb|--ref| flag is set, \verb|bibtex| is automatically called in between the multi-compilation steps}{-r, --ref}
				\feature{Auto-visualisation}{If the \verb|--show 1| option is set (which it is by default), the compiler will call \verb|xdg-open <output-file>| upon completion of the compilation; automatically opening or context-switching to the document. This can be turned off by calling \verb|--show 0|}{--show}
				\feature{Print Mode}{A special interface for \rpgtex{}, this uses the \verb|\RpgCMD| interface \RpgPage[p]{S:CMD} to inject code into the latex document, setting the \verb|bg=print| mode and suppressing the background output.}{-p, --print}
			\end{RpgTable}
		}
	\end{macrolist}

	
	\part{rpgtex Classes}
		\chapter{rpgbook Class}\label{S:bookClass}

		\chapter{rpghandout Class}\label{S:handoutClass}


		\chapter{rpgcard Class}\label{S:cardClass}

	\part{Themes}
		\RpgSetTheme{default}
\chapter{\texttt{default} Theme}\label{Theme:Default}

		\RpgSetTheme{dnd}
\chapter{\texttt{dnd} Theme}

	The \texttt{dnd} Theme is 

	\section{RpgMonster}\cmdidx{RpgMonster}

		The dnd theme defines a special command which mimics the appearance of a monster statblock - particularly those in the more modern D\&D 2024 iteration. 

	% 	\begin{multicols}{2}
	% 	\begin{RpgMonster}{Test}
			
	% 	\end{RpgMonster}
	% 	\clearpage
	% 	\Blindtext
	% \end{multicols}

	\RpgUseCards{true}

	\begin{RpgSpell}{Hocus Pocus}{
		level = 1,
		school=Transmutation,
		casting-time={1 bonus action},
		source={Player's Handbook}
	}
		The body text of the spell

	\end{RpgSpell}

	\begin{RpgSpell}[]{Hocus Pocus 2}{
		% level = 1,
		% school=Conjuration
		casting-time={1 Reaction}
	}
		The body text of the \cardbreak spell
		% \Blindtext
	\end{RpgSpell}
	
	\begin{RpgSpell}[width=8cm,under-img=\RpgPackagePath/themes/dnd/img/paper]{Hocus Pocus 3}{
		% level = 1,
		% school=Conjuration
		casting-time={1 Reaction}
	}
		The body text of the \cardbreak spell
		% \Blindtext
	\end{RpgSpell}

	Now I test \footnote{That I didn't damage anything}.

	% \newenvironment{testit}[1]{\textit{#1}\newline}{}
	% \newenvironment{testbf}[1]{\textbf{#1}\newline}{}




	% \begin{RpgCardSwitch}{testit}{testbf}{test this!}
	% 		Some text
	% \end{RpgCardSwitch}

	% \RpgCardMode{true}

	% \begin{RpgCardSwitch}[test]{testit}{testbf}{test this!}
	% 		Some text
	% \end{RpgCardSwitch}
	

		\RpgSetTheme{scifi}
\chapter{\texttt{scifi} Theme}

\Blindtext
\clearpage
\Blindtext


	\RpgSetTheme{default}
	\printindex
\end{document}
